\section*{Discussion}
\markboth{Discussion}{Discussion}
\addcontentsline{toc}{section}{Discussion}
The analytical work presented in this thesis has provided a basic understanding of gas flow behavior within the micro-reactor assembly.
Starting with the detailed geometrical characterization in Section \ref{sec:geometry}, we simplified the reactor’s complex structure into discrete segments—reservoir, inlet/outlet nozzles, and the central reaction volume—to enable a quasi one-dimensional approximation.
This initial abstraction set the stage for applying isentropic flow theory and provided a baseline for estimating characteristic lengths and flow areas.

In Section \ref{sec:expected-flow-regimes}, the expected flow regimes were determined by calculating key dimensionless numbers.
The analysis revealed that, under typical operating conditions, the system predominantly falls within the continuum regime with slip at the walls.
This finding is crucial because it validates the use of continuum-based formulations for the majority of the flow, while also pointing to specific zones (e.g., near the outlet) where the transition to molecular flow might occur.

Section \ref{sec:one-dim-isentropic} delved into one-dimensional isentropic variable area flow, where state variables were linked to changes in cross-sectional areas along the flow path.
Although two mathematically viable solutions emerged—corresponding to subsonic and supersonic branches—the physical reasoning, supported by the geometry and boundary conditions, guided the selection of the more likely scenario.
This process underscored the importance of aligning mathematical models with realistic operating constraints.

The discussion then moved to the microscale in Section \ref{sec:micro-channels}, addressing flow behaviors in micro channels.
Here, factors such as slip effects, surface roughness, and sudden geometric expansions were discussed qualitatively.
These effects, which become significant as the characteristic lengths shrink, hint at deviations from classical isentropic behavior and suggest that empirical corrections or advanced simulation techniques may be necessary for a more accurate description.

Recognizing these limitations, Section \ref{sec:disconnected-reservoirs} extended the analysis to include non-isentropic behaviors.
By treating the reservoir and the reactor as distinct entities with independent stagnation conditions, the approach captured the impact of leaks and pressure losses on the overall mass flow.
Although this formulation introduces further approximations, it serves to bridge the gap between idealized isentropic models and the more complex reality of the system.

Finally, Section \ref{sec:outlet_plume} addressed the under-expanded nozzle plume at the outlet, where rapid expansion, shock structures, and rarefaction effects dominate.
The discussion here emphasized that while traditional one-dimensional models offer a first approximation, advanced methods such as the method of characteristics, Navier–Stokes simulations, or DSMC are required to capture the full complexity of the free expansion into vacuum.

\newpage
