\begin{table}[ht]
  \centering
  \renewcommand{\arraystretch}{1.4} % Moderate row spacing
  \begin{tabular}{|c||c||c|c|c||c|}
    \hline
    Gas & $q_{N,\;in}$ ($q_{N,\;L} = 0$) & $q_{N,\;in}$ ($q_{N,\;L} > 0$) & $q_{N,\;L}$ & $q_{N,\;out}$ & Unit\\ \hline \hline
    $\text{Ar}$ & 1.69 & 2.06 & 0.90 & 1.16 & $\cdot 10^{18} \; \frac{\text{Ar}}{s}$ \\ \hline
    $\text{N}_2$ & 1.94 & 2.36 & 1.00 & 1.36 & $\cdot 10^{18}\; \frac{\text{N}_2}{s}$ \\ \hline
    $\text{CO}_2$ & 1.48 & 1.80 & 0.74 & 1.06 & $\cdot 10^{18}\; \frac{\text{CO}_2}{s}$ \\ \hline
  \end{tabular}
   \caption[Mass‐flow rates with and without a choked leak for representative gases at Mach 1 under reservoir conditions of $p_0=1.5$\,bar and $T_0=300$\,K]{%
    \textbf{Mass‐flow rates with and without a maximum choked leak for representative gases at Mach 1 under reservoir conditions of $\mathbf{p_0=1.5}$\,bar and $\mathbf{T_0=300}$\,K:}  
    The table lists the inlet mass‐flow assuming no leak ($q_{N,\;\text{leak}}=0$), the inlet mass‐flow with the leak choked at its maximum rate, the leak flow itself ($q_{N,\;\text{leak}}$), and the resulting net outlet flow ($q_{N,\;\text{out}}$) for argon (monatomic), nitrogen (diatomic), and carbon dioxide (triatomic). All values are expressed in units of $10^{18}$ molecules s$^{-1}$.  
  }
  \label{tab:massflow_discussion}
\end{table}
