\begin{table}[ht]
  \centering
  \renewcommand{\arraystretch}{1.4} % Moderate row spacing
  \begin{tabular}{|c||c|c||c|}
    \hline
    Gas & Reynolds number & Knudsen number & Description \\ \hline \hline
    $\text{H}_2$ & 24 & 0.063 & light, diatomic \\ \hline
    $\text{N}_2$ & 452 & 0.003 & common, diatomic \\ \hline
    $\text{CO}_2$ & 632 & 0.002 & heavy, triatomic \\ \hline
    $\text{Ar}$ & 479 & 0.003 & inert, monatomic \\ \hline
  \end{tabular}
  \caption[Knudsen and Reynolds numbers for representative gases at Mach 1 under reservoir conditions of $p_0=1.5$\,bar and $T_0=300$\,K:]{%
    \textbf{Knudsen and Reynolds numbers for representative gases at Mach 1 under reservoir conditions of $\mathbf{p_0=1.5}$\,bar and $\mathbf{T_0=300}$\,K:}
    The table lists four reference gases, identified by molecular structure and weight
    (H$_2$: light, diatomic; N$_2$: common, diatomic; CO$_2$: heavy, triatomic; Ar: inert, monatomic),
    showing that even at a sonic throat the flow remains in the continuum regime and in laminar flow.
  }
  \label{tab:knudsen_reynolds_discussion}
\end{table}
