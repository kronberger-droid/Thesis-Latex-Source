The characteristic length $L_c$ essentially serves the purpose of scaling physical systems.
It is fundamental when calculating most dimensionless quantities especially, if these are connected to the scale of the system.
Therefore, these quantities usually dependent on the characteristic length scale of the system or parts of the system, to be able to describe the geometry abstractly.\\
For internal flows the characteristic length is defined as:
\begin{equation}
	L_c=\frac{4A}{P_w} \quad \text{with} \quad P_w = \sum^{\infty}_{i=0} l_i
\end{equation}
Where $A$ is the cross-sectional area and $P_w$ is the wetted perimeter, which is defined as the sum over the length of edges in the cross-section in direct contact with the fluid.
For gaseous fluids the whole perimeter of the cross-section must be considered, therefore the wetted perimeter reduces to the perimeter of the cross-section.
The following section provides the characteristic length formulas for common duct shapes.
\cite{leishman_internal_2023}
\begin{figure}[H]
    \centering
    \includegraphics[width=0.85\textwidth]{src/02_foundations/fig_wetted-perimeter-nozzle-rect.pdf}
    \label{fig:wetted-rect-nozzle}
\end{figure}
If the height $H$ of a rectangular duct is very small compared to its width $W$, it is more convenient to view it as an asymptotic case where the width approaches infinity ($W \to \infty$).
\begin{figure}[H]
    \centering
    \includegraphics[width=0.9\textwidth]{src/02_foundations/fig_wetted-perimeter-plates.pdf}
    \label{fig:wetted-plates}
\end{figure}
\newpage
