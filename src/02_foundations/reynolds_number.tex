% src/02_foundations/reynolds_number.tex

To categorize the qualitative movement of a fluid, specifically whether it exhibits smooth, layered motion or chaotic, fluctuating behavior, the Reynolds number is introduced as a dimensionless quantity characterizing the ratio of inertial to viscous forces.
This quantity is defined as:
\begin{equation}
	Re = \frac{\rho V L_c}{\mu(T)} = \frac{V L_c}{\nu}
	\label{eq:reynolds-number}
\end{equation}
Where $\rho$ is the fluid density, $V$ is the flow speed, $L_c$ is a characteristic length (such as pipe diameter as described in Section~\ref{sec:characteristic-length}), $\mu$ is the dynamic viscosity, and $\nu$ is the kinematic viscosity.
These flow behaviors can be sorted into three categories depending on the value of the Reynolds number:
\cite{Cengel2017, anderson2021modern}
\begin{itemize}
	\item Laminar flow $Re < 2000$: where fluid moves in smooth layers, with minimal mixing between those layers.
	\item Transitional flow $2000 \le Re \le 4000$: marks the transition between the main regimes.
	\item Turbulent flow $Re > 4000$: where the fluid moves chaotically and mixes irregularly due to formation of eddies.
\end{itemize}
