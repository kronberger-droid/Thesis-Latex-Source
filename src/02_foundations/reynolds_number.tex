To distinguish between different kinds of flow behaviors in fluid mechanics a dimensionless quantity is introduced, which describes the ratio between inertial forces and viscous forces inside a fluid.
This quantity is called the Reynolds number and is defined as:
\begin{equation}
	Re = \frac{\rho u L_c}{\mu} = \frac{u L}{\nu}
	\label{eq:reynolds-number}
\end{equation}
Where $\rho$ is the fluid density, $u$ is the flow speed, $L$ is a characteristic length (such as pipe diameter), $\mu$ is the dynamic viscosity, and $\nu$ is the kinematic viscosity.
These flow behaviors can be sorted into three categories depending on the value of the Reynolds number:
\begin{itemize}
	\item Laminar flow $Re < 2000$: where fluid moves in smooth layers, with minimal mixing between those layers.
	\item Transitional flow $2000 \ge Re \ge 4000$: marks the transition between the main regimes.
	\item Turbulent flow $Re > 4000$: where the fluid moves chaotic and mixes irregularly due to formation of eddies.
\end{itemize}
\cite{Cengel2017, anderson2021modern}

