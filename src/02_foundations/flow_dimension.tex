	The dimension of the flow describes the number of positional parameters needed to yield an exact solution for a given flow velocity field $V(\vec{x})$, so equals essentially $\dim(\vec{x})$. 
	The flow through a constant area duct is usually described using a one-dimensional flow field only depending on the position $x$ along the length of the duct.
	In the case of more complicated geometries the flow must be described three-dimensional and has to be calculated using all spatial coordinates.
	In the case of variable area ducts -- like nozzles and diffusers -- assuming only a slight change in area along the length of the duct the flow can be approximated using a one-dimensional flow field with enough precision.
	This is called quasi one dimensional flow. \cite{anderson2021modern}
