\section{Scope and objectives}
	The general goal of this thesis is to create a relatively simple analytical framework to be able to make predictions about the behavior of the flow through the system and approximate values at different positions in the flow to later be used as initial values for more complex numerical simulations.
	The following section will state specific questions we will then try to answer in the following sections.
	\begin{figure}[H]
	    \centering
	    \includegraphics[width=\textwidth]{src/02_scope-and-objectives/fig_technical-drawing.pdf}
	    \caption{Dimensions of the micro-reactor assembly \cite{lagin2025poster}.}
	    \label{fig:technical-drawing}
	\end{figure}
	
\subsection{Type of Flow}

	The type of flow has major implications on which mathematical formulations and simulations are applicable, as well as the way the gas particles interact with each other and the walls of the assembly. 
	The main focus here is the Knudsen number and the idealized flow regimes connected to it.
	With the main goal being to assess the most likely flow regime governing the inside of the assembly and therefore determine the equations applicable to calculate the state variables at different points in the system and the throughput of the system as a whole.

	In preparation for numerical simulations it is also important to find a way to calculate Knudsen numbers and other flow parameters using given datasets of state variables without having to rely on flow regime specific methods. This will help to analyze transient regimes, encountered when the gas expands into the vacuum, using one generally applicable method.  

\subsection{Impact of the leak}

	As described in the leading section there will be some leakage expected at the boundary between the reactor casing and the sample holder.
	This leak will inevitably lead to some mass-flow and therefore some pressure drop $\Delta P_L$ inside the reactor.
	This can lead to major changes to the steady state of the system, therefore influencing the velocity distribution at the outlet.
	In summary the goal is finding the pressure drop $\Delta P_L$ caused by the leak and the effective mass flow $\dot{m}_L$ through it and use them to predict probable behaviors.

\subsection{Behavior of the gas around the sample}

	Knowing more about the state the gas is in close to the sample is very helpful.
	These can be used to estimate values for diffusion rates to and from the surface, boundary layer thickness, and mean velocities and momentum of particles reaching the reactive surface. 
	Important metrics are the velocity, pressure and temperature of the gas and the type of flow.

\subsection{Velocity distribution at the outlet}

	After the gas leaves the outlet nozzle, it expands into a vacuum chamber where the gas atoms are ionized by an electron beam and picked up by the mass-spec, to measure the ratio of different products.
	This won't be the case for all atoms, since for an atom to be ionized it has to cross the electron beam, which is localized in space.
	The remaining gas has to be pumped out and doesn't contribute to the ratio measured by the mass-spec.
	Therefore, it is important to approximate, how much of the gas leaving the outlet actually is able to reach the region of influence of the electron beam and will contribute to the measurement of the mass-spec.
	To answer this the velocity distribution of the expansion after the gas is fully rarefied is needed.
	This distribution can then essentially be treated as a source like surface, with no interaction between gas particles, and can therefore be directly correlated to the amount of atoms reaching the sphere of influence of the electron beam.\\
	Knowing what determines the distribution of the outlet can also help identify changes to be made to the geometry or the reservoir conditions to increase the amount of atoms reaching the mass-spec.
