\section{Scope and objectives}\label{sec:scope-objectives}
	The aim of this thesis is to develop a straightforward analytical framework that predicts gas flow behavior and estimates key state variables at strategically relevant positions within a micro-reactor assembly.  
	These state variables include pressure, temperature, density, and flow velocity, which together provide a macroscopic description of the system's thermodynamic and transport conditions.  
	Accurate knowledge of these parameters is essential for understanding the overall flow dynamics and for defining reliable boundary and initial conditions for subsequent numerical simulations of microscopic or reactive flow phenomena.

	Figure~\ref{fig:technical-drawing} illustrates the reactor configuration. Reactants are first introduced into the reservoir, where they mix before entering the reaction zone through a narrow inlet.  
	Within the reaction volume, gas molecules interact with the active surface of the sample under controlled conditions.  
	The flow then continues through the outlet into the ultra-low-pressure analysis section, where residual gases are removed.  
	In parallel, a small fraction of the gas leaks through the sealing interface into the sample transfer section of the surrounding vacuum chamber, affecting local pressure distributions.

	This model focuses on the continuum flow regime and provides estimates of state variables at critical locations such as the reservoir, inlet, reaction chamber, outlet, and leak path.  
	These estimates form the foundation for assessing flow behavior under near-ambient pressures and will later be refined by high-resolution numerical methods and experimental validation.
	\begin{figure}[H]
	    \centering
	    \includegraphics[width=\textwidth]{src/01_scope-and-objectives/fig_technical-drawing.pdf}
	    \caption[Schematics of the micro-reactor assembly \cite{lagin2025poster}:]{
			\textbf{Schematics of the micro-reactor assembly \cite{lagin2025poster}:} The reactants are mixed in the reservoir, pass progressively through the inlet (green arrow), reaction volume (yellow arrow), and exhaust into the vacuum through the outlet (red arrow).
			The exhaust gas composition is analyzed via quadrupole mass spectroscopy.
			The part of the gas in the reaction volume leaks through the space between the sample and the sealing surface of the reactor (blue line).
			}
	    \label{fig:technical-drawing}
	\end{figure}

	The sections that follow address four main objectives: identifying the dominant flow regime, quantifying the impact of leakage, characterizing the gas state close to the sample, and predicting some characteristics of the velocity distribution at the outlet.
	To achieve these goals, the analysis relies on established concepts such as the Knudsen and Reynolds numbers, along with one-dimensional isentropic flow theory.
	This foundation, which is detailed in the subsequent chapter on foundational principles, not only clarifies the interaction between the reactor’s geometry and gas dynamics but also lays the groundwork for future numerical simulations and experimental studies.	

\subsection{Type of Flow}

	The type of flow has major implications on which mathematical formulations and simulations are applicable, as well as the way the gas particles interact with each other and the walls of the assembly. 
	The main focus here is on the Knudsen number and its associated idealized flow regimes, with the ultimate goal of identifying the regime that most likely governs the interior of the assembly and, in turn, selecting the appropriate equations to calculate state variables at the relevant points and the overall system throughput.

	In preparation for numerical simulations it is also important to find a way to calculate Knudsen numbers and other flow parameters using given datasets of state variables without having to rely on flow regime specific methods. This will help to analyze transient regimes, encountered when the gas expands into the vacuum, using generally applicable methods.

\subsection{Impact of the leak}

	As described in the introduction of this chapter, there will be some leakage expected at the boundary between the reactor casing and the sample.
	This leak will inevitably lead to additional mass-flow and therefore a pressure drop $\Delta P_L$ inside the reactor.
	This can lead to major changes to the steady state of the system, therefore influencing the gas-flow into the reactor as well as the velocity distribution at the outlet.
	In summary the goal is to find the maximum mass flow $\dot{m}_L$ caused by the leak, which would result in sonic speeds at the inlet throat, resulting in undefined behavior inside the micro reactor, such as sudden changes in state variables in the form of shockwaves. 

\subsection{Behavior of the gas around the sample}

	Knowing more about the state of the gas near the sample is very helpful.
	This information can be used to estimate values for diffusion rates to and from the surface, boundary layer thickness, and mean velocities and momentum of particles reaching the reactive surface. 
	Important metrics are the velocity, pressure and temperature of the gas and the type and behavior of flow.

\subsection{Velocity distribution at the outlet}

	After the gas leaves the outlet nozzle, it expands into a vacuum chamber where the gas atoms reach the ionizer and, once ionized, are analyzed by the quadrupole mass spectrometer (QMS), to measure the ratio of different products.
	This won't be the case for all atoms, since for an atom to be ionized it has to cross the ionization area, which is localized in space.
	The remaining gas has to be pumped out of the vacuum chamber and doesn't contribute to the ratio measured by the mass-spec.
	Therefore, it is important to approximate how much of the gas leaving the outlet actually is able to reach the sphere of influence of the electron beam and will contribute to the measurement of the QMS.
	To answer this the velocity distribution of the expansion after the gas is fully rarefied is needed.
	This distribution can then essentially be treated as a source like surface, with no interaction between gas particles after this point, and can therefore be directly correlated to the amount of atoms reaching the sphere of influence of the electron beam.\\
	Knowing what determines the distribution of the outlet can also help identify changes to be made to the geometry or the reservoir conditions to increase the amount of atoms reaching the QMS.
