\section*{Introduction}
	Catalysis is an inseparable part of modern society, relying on the chemical industry, solutions in ecology, or effective energy storage.
	Therefore, the development of superior catalysts has been in the scope of surface science for many years. 

	One of the approaches is the preparation and investigation of model catalysts in the form of single crystal support with well-defined active sites.
	The use of inexpensive support materials and the reduction of deposited noble metals significantly increase the effectiveness and selectivity of model catalysts.
	Such a catalyst configuration also enables a better understanding of physical processes happening on its surface. 

	However, their preparation and investigation require ultra-high vacuum (UHV) to preserve adequate inertness of the surrounding environment.
	Therefore, the laboratory pressure conditions usually do not reflect the real industrial setup. To bridge this pressure gap, the microreactor (high-pressure cell) located in the UHV chamber, enabling sample exposure to elevated gas pressure and temperature, is under development. 

	There are several similar devices in use, however, their microscopic flow description is often trivialized.
	And particularly flow conditions at the reactive surface have a significant influence on the reactant and product transportation.
	Omitting these factors can then result in a misleading read of the catalytic activity. 

	The main scientific task is to get an uncomplicated and reliable overview of the macroscopic gas flow properties inside the high-pressure cell located in the UHV chamber.
	The appropriate gas flow theory is presented initially and later applied to certain geometry and required conditions.
	The main attention was paid to selected, most-critical locations in the micro reactor.
	The obtained results, respectively the formulated approach to calculate gas properties, are essential for ongoing microscopic flow investigation, and can be easily applied on adjusted geometry of high-pressure cells. 
