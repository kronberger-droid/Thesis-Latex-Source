\section*{Abstract}
Understanding catalytic processes at the atomic scale under realistic conditions is a key challenge in surface science.
In particular, the study of single-atom catalysis at near-ambient pressures requires specialized experimental setups, such as high-pressure micro-reactor cells integrated into ultra-high vacuum (UHV) systems.
This thesis develops an analytical framework to describe gas flow behavior within such a micro-reactor, focusing on the continuum flow regime.
Critical state variables and flow velocities at relevant locations are estimated to provide a reliable macroscopic flow description.
While effects such as leakage and outlet flow remain to be quantified in detail and should incorporate advanced numerical simulations and experimental validation to improve the model accuracy and applicability.
The presented approach lays the groundwork for future microscopic flow studies.
\blankpage

% LTeX: language=de-DE
\section*{Kurzfassung}
Das Verständnis katalytischer Prozesse auf atomarer Ebene unter realistischen Bedingungen stellt eine zentrale Herausforderung in der Oberflächenwissenschaft dar.
Insbesondere die Untersuchung der Einzelatomkatalyse bei nahezu Umgebungsdruck erfordert spezialisierte experimentelle Aufbauten, wie Hochdruck-Mikroreaktorzellen, die in Ultrahochvakuum-(UHV)-Systeme integriert sind.
Diese Arbeit entwickelt einen analytischen Rahmen zur Beschreibung des Gasströmungsverhaltens innerhalb eines solchen Mikroreaktors, mit Fokus auf das Kontinuumsströmungsregime.
Kritische Zustandsgrößen und Strömungsgeschwindigkeiten an relevanten Stellen werden abgeschätzt, um eine verlässliche makroskopische Strömungsbeschreibung bereitzustellen.
Effekte wie Leckagen und der Auslassstrom müssen hingegen noch im Detail quantifiziert werden und sollten durch weiterführende numerische Simulationen und experimentelle Validierung in das Modell integriert werden, um dessen Genauigkeit und Anwendbarkeit zu verbessern.
Dennoch bildet der vorgestellte Ansatz die Grundlage für zukünftige mikroskopische Strömungsuntersuchungen.
\blankpage
