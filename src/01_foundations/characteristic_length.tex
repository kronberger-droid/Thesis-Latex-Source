The characteristic length essentially serves the purpose of scaling physical systems.
In dimensional analysis the goal is to find dimensionless quantities which describe the behavior of the system.
These quantities are usually dependent on some characteristic scale, in our case the length scale, which describes the geometry of the model abstractly.\\
For internal flows the characteristic length is defined as:
$$
	L_c=\frac{4A}{P_w} \quad \text{with} \quad P_w = \sum^{\infty}_{i=0} l_i
$$
Where $A$ is the cross-sectional area and $P_w$ is the wetted perimeter, which is defined as the sum over the lengths of all surfaces in direct contact with the fluid.
For gaseous fluids the whole perimeter of the cross-section must be considered, therefore the wetted perimeter reduces to the perimeter of the cross-section.
The following section provides the characteristic length formulas for common duct shapes. 
% src/01_foundations/fig_wetted-perimeter.tex
\subsubsection*{Circular duct or Nozzle}

\begin{figure}[htbp]
\centering 
	\begin{minipage}[c]{0.45\textwidth}
	    \begin{tikzpicture}
	        \pgfmathsetmacro{\radius}{1.2}
			% \pdfmathsetmacro{\xshift}{2}
	        % Draw the circle
	        \filldraw[fill=blueColor!10, draw=black, thick] (0,0) circle [radius=\radius];
	        % radius annotation
	        \draw[{Stealth}-{Stealth}] (-\radius,0) -- (\radius,0) node[midway, below] {$D$};
	    \end{tikzpicture}
	\end{minipage}
	\begin{minipage}[c]{0.45\textwidth}
		\raggedleft
		$$
			P_{w_\circ} = \pi D \quad \rightarrow \quad L_{c_\circ} = D
		$$
	\end{minipage}
\end{figure}


\subsubsection*{Rectangular duct}

\begin{figure}[htbp]
\centering 
	\begin{minipage}[c]{0.45\textwidth}
	    \begin{tikzpicture}
	        % Draw the rectangle
	        \fill[blueColor!10] (0,0) rectangle (4,2);
	        \draw[thick] (0,0) rectangle (4,2);
	        % Width annotation
	        \draw[{Stealth}-{Stealth}] (0,1) -- (4,1) node[midway, below] {$W$};
	        % Height annotation
		    \draw[{Stealth}-{Stealth}] (4.5,0) -- (4.5,2) node[midway, right] {$H$};
	    \end{tikzpicture}
	\end{minipage}
	\begin{minipage}[c]{0.45\textwidth}
		\centering
		$$
			P_{w_\square} = 2H + W \quad \rightarrow \quad L_{c_\square} = \frac{2HW}{H+W}
		$$
	\end{minipage}
\end{figure}


\subsubsection*{Two parallel plates}

If the height $H$ of a rectangular duct is very small compared to its width $W$, it is more convenient to view it as an asymptotic case where the width approaches infinity ($W \to \infty$).

\begin{figure}[htbp]
\centering
	\begin{minipage}[c]{0.45\textwidth}
		\begin{tikzpicture}
		    % Draw the parallel lines
	        \fill[blueColor!10] (0,0) rectangle (4,2);
		    \draw[thick] (0,0) -- (4,0);
		    \draw[thick] (0,2) -- (4,2);
		    % Distance annotation
	    \draw[{Stealth}-{Stealth}] (4.5,0) -- (4.5,2) node[midway, right] {$H$};
		\end{tikzpicture}
	\end{minipage}
	\begin{minipage}[c]{0.45\textwidth}
	\centering
		$$
			\lim_{W \to \infty} L_{c_\square} =  \lim_{W \to \infty} \frac{2HW}{H+W} = 2H
		$$
	\end{minipage}
\end{figure}




