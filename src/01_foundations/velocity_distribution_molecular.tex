
\begin{figure}[H]
\centering
	\begin{tikzpicture}[scale=1.2]

% Pipe background with border
\fill[baseLightGray] (-3,-1) rectangle (3,1);
\draw[baseDarkerGray, thick] (-3,-1) rectangle (3,1);

% Horizontal centerline
\draw[dashed, baseDarkGray] (-3.2,0) -- (3.2,0);

% Boundary lines
\draw[redColor, thick] (-3,1) -- (3,1);
\draw[redColor, thick] (-3,-1) -- (3,-1);

% r-axis
\draw[-{Stealth[length=4pt]}, baseDarkerGray] (-3,-1.2) -- (-3,1.5) node[above] {$r$};
\draw (-3.2,0) node[left, baseDarkerGray] {0};

% Seed for random number generator
\pgfmathsetseed{12345}

% Parameters
\def\particleRadius{0.05} % Radius of particles
\def\vectorLength{0.3}    % Length of velocity vectors
\def\pipeLeft{-3}         % Left boundary of the pipe
\def\pipeRight{1}         % Right boundary of the pipe
\def\pipeBottom{-1}       % Bottom boundary of the pipe
\def\pipeTop{1}           % Top boundary of the pipe
\def\numParticles{20}     % Number of particles

% Add particles with velocity vectors
\foreach \i in {1,...,\numParticles} {
    % Random position within the pipe, accounting for particle radius and vector length
    \pgfmathsetmacro{\xpos}{\pipeLeft + \particleRadius + abs(rand*(\pipeRight - \pipeLeft - 2*\particleRadius))}
    \pgfmathsetmacro{\ypos}{\pipeBottom + 2*\particleRadius + abs(rand*(\pipeBottom - \pipeTop + 4*\particleRadius))}
    % Random angle for velocity direction
    \pgfmathsetmacro{\angle}{360*rand}
    % Particle
    \fill[blueColor] (\xpos,\ypos) circle (\particleRadius);
    % Velocity vector
    \draw[blueColor, -{Stealth[length=2pt]}] (\xpos,\ypos) -- ++({\vectorLength*cos(\angle)},{\vectorLength*sin(\angle)});
}

\draw[dashed, baseDarkGray] (-1,1.2) -- (-1,-1.2) node[below] {$x$} (-1,1.4);

\draw[greenColor, thick, -{Stealth[length=5pt]}] (-1,0) -- (0,0);
\node[greenColor] at (-0.4,0.2) {\large $u_{\text{avg}}$};

\end{tikzpicture}
\caption{Mean velocity parallel to the flow at a point $x$ inside a constant area duct in molecular flow.}
\label{fig:flow}
\end{figure}
