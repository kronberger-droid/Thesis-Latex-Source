\subsubsection{Mach regimes}

	The Mach number is defined as the ratio between the local velocity $u$ and the local speed of sound $a$.
	$$
		Ma = \frac{u}{a}
	$$
	It is a very important metric when analyzing isentropic flow.

\paragraph{Low subsonic regime (\(Ma < 0.3\))}

	For low Mach numbers, compressibility effects of a gas can be neglected, and the gas can be treated as an incompressible fluid.

\paragraph{Subsonic regime (\(0.3 < Ma < 1.0\))} 

	Inside system of variable area ducts the gas flow generally stays subsonic.
	Once sonic speed is reached in a converging duct, the behavior reverses, and the velocity decreases, limiting the flow to subsonic or sonic speeds within converging ducts.\\
	
	\import{src/01_foundations}{subsonic.tex}	
\paragraph{Sonic regime (\(Ma = 1\))}
	
	Sonic flow occurs at the exit of a converging duct, if a certain critical pressure ratio between the two reservoirs connected by the duct is reached.
	This ratio is defined as:
	$$
		\frac{P^*}{P_t}=\left(\frac{2}{\gamma + 1}\right)^{\gamma/(\gamma - 1)}
	$$ 
	It is derived from isentropic flow relations and can be expressed in any state variable.
\paragraph{Supersonic regime (\(Ma > 1\))} 

	If there are critical conditions at the end of a converging duct and a diverging duct follows.
	The flow continues to accelerate and reaches supersonic speeds.
	The location where the flow reaches critical condition is called throat and represents the minimal diameter of the duct.\\
	In supersonic flows, state variables change rapidly causing phenomenons like shock waves and expansion fans.
	
	\import{src/01_foundations}{supersonic.tex}

{\color{greenColor}\itshape
	Add references! There should be two good ones!
}
