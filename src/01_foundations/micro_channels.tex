	Going from macro scale channels to microscales has some major implications for the behavior of the gas.
	The primary factor for these differences is slipping at the boundary of the surfaces.
	This is due to the fact that at small characteristic length scales the Knudsen number ($Kn$), whose value describes the interaction of the molecules in the gas and its boundaries, becomes relatively high ($Kn > 0.001$).
	Which usually puts the gas flow in the category of compressible flow with slip at the boundaries.

	Most of these behaviors have to be studied using complex simulations or experimental results and are sometimes not fully explained.
	Therefore, this section is not intended to give concrete definitions or formulations, but to provide as many relevant references regarding behaviors in microfluidics as possible.
	It will be mainly based on the review study conducted by Amit Agrawal.
	\cite{agrawal_comprehensive_2011}

\subsubsection*{Phenomenon of Slip}
	Slip, in comparison to non-slip, refers to the fact that the tangential velocity close to a surface is non-zero.
	Maxwell suggested that on a control surface $s$, at distance half the mean free path away from the surface, one half of the molecules come in from one mean free path away with the tangential velocity $u_\lambda$, the other half is reflected from the surface.
	Assuming a fraction $\sigma$ of the molecules are reflected diffusively (average velocity corresponds to velocity at the wall $u_w$)at the walls, with the remainder $(1-\sigma)$ being reflected specularly (no change in their impinging velocity $u_\lambda$).
	When expanding $u_\lambda$ in a second order Taylor series this yields the second order slip boundary condition used in continuum analysis:
	$$
		u_g - u_w =
		\left[
			\frac{2-\sigma}{\sigma}Kn\left(\frac{\partial u}{\partial n}\right)_s
			+ \frac{Kn^2}{2} \left(\frac{\partial^2 u}{\partial n^2}\right)_s
		\right]
		\qquad \eqref{}
	$$ 
	Where $u$ stands for the streamwise velocities, where the subscripts $g$, $w$ and $s$ refer to gas, wall and control surface, with $n$ being the normal to the control surface.
	And most importantly $\sigma$ is the tangential momentum accommodation coefficient, or short TMAC.

	Determining the TMAC for a specific application is one of the most critical aspects when dealing with slip conditions, as it directly influences the velocity slip at the gas-wall interface and, consequently, the overall behavior of rarefied gas flows.
	This is usually achieved through empirical studies or simulations like direct simulation Monte Carlo which will be discussed in more detail in the next section.

\subsubsection*{Surface Roughness}

\subsubsection*{Sudden expansion or contraction}

\note{
	Add segway over to why making the reactor a reservoir solves some of the problems while still not ignoring them.
}
