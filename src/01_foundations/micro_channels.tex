Going from macro scale channels to microscales has some major implications for the behavior of the gas.
The primary factor for these differences is slipping at the boundary of the surfaces.
This is due to the fact that at small characteristic length scales the Knudsen number ($Kn$), whose value describes the interaction of the molecules in the gas and its boundaries, becomes relatively high ($Kn > 0.001$).
Which usually puts the gas flow in the category of compressible flow with slip at the boundaries.

Most of these behaviors have to be studied using complex simulations or experimental results and are sometimes not fully explained.
Therefore, this section is not intended to give concrete definitions or formulations, but to provide as many relevant references regarding behaviors in microfluidics as possible.
It will be mainly based on the review study conducted by Amit Agrawal.
\cite{agrawal_comprehensive_2011}

\subsubsection*{Phenomenon of Slip}
Slip, in comparison to non-slip, refers to the fact that the tangential velocity close to a surface is non-zero.
Determining this velocity at the boundary is the main focus when slip is to be included in future calculations.

\paragraph*{Slip models}
Maxwell suggested that on a control surface $s$, at distance half the mean free path away from the surface, one half of the molecules come in from one mean free path away with the tangential velocity $u_\lambda$, the other half is reflected from the surface.
Assuming a fraction $\sigma$ of the molecules are reflected diffusively (average velocity corresponds to velocity at the wall $u_w$)at the walls, with the remainder $(1-\sigma)$ being reflected specularly (no change in their impinging velocity $u_\lambda$).
When expanding $u_\lambda$ in a second order Taylor series this yields:

$$
	u_g - u_w =
	\left[
		\frac{2-\sigma}{\sigma}Kn\left(\frac{\partial u}{\partial n}\right)_s
		+ \frac{Kn^2}{2} \left(\frac{\partial^2 u}{\partial n^2}\right)_s
	\right]
$$ 

Where $u$ stands for the streamwise velocities, where the subscripts $g$, $w$ and $s$ refer to gas, wall and control surface, with $n$ being the normal to the control surface.
And most importantly $\sigma$ is the tangential momentum accommodation coefficient, or short TMAC.

The TMAC can be expressed in terms of tangential momentum of the incident molecules $\tau_i$ and the tangential momentum of the reflected ones $\tau_r$:

$$
	\sigma = \frac{\tau_i - \tau_r}{\tau_i}
$$

\paragraph*{The tangential momentum accommodation coefficient}
\subsubsection*{Surface Roughness}

\subsubsection*{Numerical simulations}

\subsubsection*{Sudden expansion or contraction}

