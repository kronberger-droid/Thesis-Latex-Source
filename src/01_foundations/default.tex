\section{Foundational Principles}
	When dealing with advanced fluid dynamical systems, before being able to formulate models to solve them, they have to be categorized.
	This is usually done by determining dimensionless numbers which values ought to describe more the behavior of the gas and its interaction with its surrounding.
	Essentially the values of these dimensionless numbers give a clue on which formulations are applicable and if additional boundary conditions or considerations have to be taken into account.
	This first chapter will go into the details of how to calculate some of the most important dimensionless numbers and give insight on what the values of them imply.

\subsection{Characteristic Length}  
The characteristic length essentially serves the purpose of scaling physical systems.
In dimensional analysis the goal is to find dimensionless quantities which describe the behavior of the system.
These quantities are usually dependent on some characteristic scale, in our case the length scale, which describes the geometry of the model abstractly.\\
For internal flows the characteristic length is defined as:
$$
	L_c=\frac{4A}{P_w} \quad \text{with} \quad P_w = \sum^{\infty}_{i=0} l_i
$$
Where $A$ is the cross-sectional area and $P_w$ is the wetted perimeter, which is defined as the sum over the lengths of all surfaces in direct contact with the fluid.
For gaseous fluids the whole perimeter of the cross-section must be considered, therefore the wetted perimeter reduces to the perimeter of the cross-section.
The following section provides the characteristic length formulas for common duct shapes. 
% src/01_foundations/fig_wetted-perimeter.tex
\subsubsection*{Circular duct or Nozzle}

\begin{figure}[htbp]
\centering 
	\begin{minipage}[c]{0.45\textwidth}
	    \begin{tikzpicture}
	        \pgfmathsetmacro{\radius}{1.2}
			% \pdfmathsetmacro{\xshift}{2}
	        % Draw the circle
	        \filldraw[fill=blueColor!10, draw=black, thick] (0,0) circle [radius=\radius];
	        % radius annotation
	        \draw[{Stealth}-{Stealth}] (-\radius,0) -- (\radius,0) node[midway, below] {$D$};
	    \end{tikzpicture}
	\end{minipage}
	\begin{minipage}[c]{0.45\textwidth}
		\raggedleft
		$$
			P_{w_\circ} = \pi D \quad \rightarrow \quad L_{c_\circ} = D
		$$
	\end{minipage}
\end{figure}


\subsubsection*{Rectangular duct}

\begin{figure}[htbp]
\centering 
	\begin{minipage}[c]{0.45\textwidth}
	    \begin{tikzpicture}
	        % Draw the rectangle
	        \fill[blueColor!10] (0,0) rectangle (4,2);
	        \draw[thick] (0,0) rectangle (4,2);
	        % Width annotation
	        \draw[{Stealth}-{Stealth}] (0,1) -- (4,1) node[midway, below] {$W$};
	        % Height annotation
		    \draw[{Stealth}-{Stealth}] (4.5,0) -- (4.5,2) node[midway, right] {$H$};
	    \end{tikzpicture}
	\end{minipage}
	\begin{minipage}[c]{0.45\textwidth}
		\centering
		$$
			P_{w_\square} = 2H + W \quad \rightarrow \quad L_{c_\square} = \frac{2HW}{H+W}
		$$
	\end{minipage}
\end{figure}


\subsubsection*{Two parallel plates}

If the height $H$ of a rectangular duct is very small compared to its width $W$, it is more convenient to view it as an asymptotic case where the width approaches infinity ($W \to \infty$).

\begin{figure}[htbp]
\centering
	\begin{minipage}[c]{0.45\textwidth}
		\begin{tikzpicture}
		    % Draw the parallel lines
	        \fill[blueColor!10] (0,0) rectangle (4,2);
		    \draw[thick] (0,0) -- (4,0);
		    \draw[thick] (0,2) -- (4,2);
		    % Distance annotation
	    \draw[{Stealth}-{Stealth}] (4.5,0) -- (4.5,2) node[midway, right] {$H$};
		\end{tikzpicture}
	\end{minipage}
	\begin{minipage}[c]{0.45\textwidth}
	\centering
		$$
			\lim_{W \to \infty} L_{c_\square} =  \lim_{W \to \infty} \frac{2HW}{H+W} = 2H
		$$
	\end{minipage}
\end{figure}






\subsection{Turbulence and the Reynolds number}
% src/02_foundations/reynolds_number.tex

To categorize the qualitative movement of a fluid, specifically whether it exhibits smooth, layered motion or chaotic, fluctuating behavior, the Reynolds number is introduced as a dimensionless quantity characterizing the ratio of inertial to viscous forces.
This quantity is defined as:
\begin{equation}
	Re = \frac{\rho V L_c}{\mu(T)} = \frac{V L_c}{\nu}
	\label{eq:reynolds-number}
\end{equation}
Where $\rho$ is the fluid density, $V$ is the flow speed, $L_c$ is a characteristic length (such as pipe diameter as described in Section~\ref{sec:characteristic-length}), $\mu$ is the dynamic viscosity, and $\nu$ is the kinematic viscosity.
These flow behaviors can be sorted into three categories depending on the value of the Reynolds number:
\cite{Cengel2017, anderson2021modern}
\begin{itemize}
	\item Laminar flow $Re < 2000$: where fluid moves in smooth layers, with minimal mixing between those layers.
	\item Transitional flow $2000 \le Re \le 4000$: marks the transition between the main regimes.
	\item Turbulent flow $Re > 4000$: where the fluid moves chaotically and mixes irregularly due to formation of eddies.
\end{itemize}


\subsection{Rarefaction and the Knudsen number}
	In gas dynamics a flow can be categorized by its particle interaction using the Knudsen number, which represents the ratio between the mean-free-path $\lambda$ of the gas and some characteristic length $L_c$.
	$$
		Kn=\frac{\lambda}{L_c}
	$$
	The characteristic length is usually chosen to be the smallest linear length in the system.
	For example the throat diameter of a nozzle. \cite{putignano2012supersonic}

\subsubsection*{Molecular regime (\(Kn \geq 10\))}

	In this regime, the mean free path is much larger than the dimensions of boundaries.
	This leads to particle interactions themselves becoming negligible in comparison to the interaction of particles with the boundary.
	\cite{rapp2017microfluidics}
	\input{src/01_foundations/fig_velocity-distribution-molecular.tex}

\subsubsection*{Transition regime (\(0.1 \leq Kn \leq 10\))}
	
	This regime is a middle ground between continuum and fully molecular flow.
	Neither the continuum assumptions of fluid dynamics nor the free molecular flow assumptions hold completely.
	The interactions between the gas molecules and the boundaries are significant, and the flow characteristics may vary widely.

\subsubsection*{Slip regime (\(0.001 \leq Kn \leq 0.1\))}

	For increasing Knudsen numbers the mean free path becomes comparable to the characteristic length scale of the system.
	In this regime, the assumptions for continuum flow still hold, but there are deviations, especially near the boundaries.
	While continuum mechanics assumes no-slip conditions on the boundary, in this regime, slip on the boundary must be factored in.

	\input{src/01_foundations/fig_velocity-distribution-slip.tex}

\subsubsection*{Continuum regime (\(Kn \leq 0.001\))}
	
	In this regime, the interactions of particles in the medium are much more frequent than the interactions of particles with the boundaries of the duct.
	This makes it possible to describe the fluid itself as a continuous medium with the assumption of non-slip boundary conditions.
	The Navier-Stokes equations govern the calculations in this regime.
	\input{src/01_foundations/fig_velocity-distribution-continuum.tex}


\subsection{Dimension of the flow}
	The dimensionality of the flow describes how many spatial coordinates the values in a velocity field depend on.
	The flow through a constant area duct is usually described as a one-dimensional flow field only depending on the position $x$ along the length of the duct.
	In the case of variable area ducts the flow will be three-dimensional and has to be calculated for all spatial coordinates.
	But assuming only a slight change in area along the length of the duct the flow can be approximated using a one-dimensional flow field with enough precision.
	This is called quasi one dimensional flow. \cite{anderson2021modern}


\subsection{Isentropic one-dimensional flow}
	Isentropic varying-area flow is one of the most idealized models to describe the behavior of gases flowing through a confined space. The following assumptions are made:
	\begin{itemize}
		\item steady, one-dimensional flow
		\item adiabatic: $\delta q = 0, ds_e = 0$
		\item no shaft work: $\delta w_s = 0$
		\item negligible change in potential energy: dz = 0
		\item reversible: $ds_i = 0$
	\end{itemize}
	Being reversible as well as adiabatic, the flow is therefore isentropic.

	\note{
		Mach number is an important tool since all state variables can be calculated using it plus some reference states in the system.
	}	

	The Mach number is defined as the ratio between the local velocity $u$ and the local speed of sound $a$.
	$$
		Ma = \frac{u}{a}
	$$
	It is a very important metric when analyzing isentropic flow.

\paragraph{Low subsonic regime (\(Ma < 0.3\))}

	For low Mach numbers, compressibility effects of a gas can be neglected, and the gas can be treated as an incompressible fluid.

\paragraph{Subsonic regime (\(0.3 < Ma < 1.0\))} 

	Inside system of variable area ducts the gas flow generally stays subsonic.
	Once sonic speed is reached in a converging duct, the behavior reverses, and the velocity decreases, limiting the flow to subsonic or sonic speeds within converging ducts.\\
	
	% src/01_foundations/fig_subsonic.tex
\begin{figure}[H]
	\centering
	\begin{tikzpicture}[font=\small, scale=1.4]

		% Titles
		\node[baseBlack] at (1,1.3) {\large Subsonic Nozzle};
		\node[baseBlack] at (6,1.3) {\large Subsonic Diffuser};

		% Nozzle on the left
		\draw[baseBlack, thick] (0,1) -- (2,0.5) -- (2,-0.5) -- (0,-1) -- cycle;

		% Flow arrow into nozzle
		\draw[line width=1.2pt, blueColor, -{Stealth[length=6pt]}] (-1.2,0) -- (0,0) 
		  node[midway, above, sloped, baseBlack]{subsonic};

		% Parameters in the nozzle
		% Green parameters (top line)
		\node[greenColor] at (1.0,0.3) {Ma $\uparrow$ \quad V $\uparrow$};

		% Red parameters (below)
		\node[redColor] at (1.0,-0.05) {P $\downarrow$ \quad T $\downarrow$ \quad $\rho \downarrow$};
		% Diffuser on the right
		\draw[baseBlack, thick] (5,0.5) -- (7,1) -- (7,-1) -- (5,-0.5) -- cycle;

		% Flow arrow into diffuser
		\draw[line width=1.2pt, blueColor, -{Stealth[length=6pt]}] (3.8,0) -- (5,0) 
		  node[midway, above, sloped, baseBlack]{subsonic};

		% Parameters in the diffuser
		% Green parameters (top line)
		\node[greenColor] at (6.0,0.25) {T $\uparrow$ \quad P $\uparrow$ \quad $\rho \uparrow$};

		% Red parameters (below)
		\node[redColor] at (6.0,-0.05) {V $\downarrow$ \quad Ma $\downarrow$};

	\end{tikzpicture}

	\caption{Change of flow properties in subsonic nozzles and diffusers}
	\label{fig:nozzle_diffuser_super}
\end{figure}
	
\paragraph{Sonic regime (\(Ma = 1\))}
	
	Sonic flow occurs at the exit of a converging duct, if a certain critical pressure ratio between the two reservoirs connected by the duct is reached.
	This ratio is defined as:
	$$
		\frac{P^*}{P_t}=\left(\frac{2}{\gamma + 1}\right)^{\gamma/(\gamma - 1)}
	$$ 
	It is derived from isentropic flow relations and can be expressed in any state variable.
\paragraph{Supersonic regime (\(Ma > 1\))} 

	If there are critical conditions at the end of a converging duct and a diverging duct follows.
	The flow continues to accelerate and reaches supersonic speeds.
	The location where the flow reaches critical condition is called throat and represents the minimal diameter of the duct.\\
	In supersonic flows, state variables change rapidly causing phenomenons like shock waves and expansion fans.
	
	% src/01_foundations/fig_supersonic.tex 
\begin{figure}[H]
\centering
	\begin{tikzpicture}[font=\small, scale=1.4]
		% Titles
		\node[baseBlack] at (1,1.3) {\large Supersonic Nozzle};
		\node[baseBlack] at (6,1.3) {\large Supersonic Diffuser};

		%%%%%%%%%%%%%%%%%%%%%%%%
		% Supersonic Nozzle on the left (diverging duct)
		%%%%%%%%%%%%%%%%%%%%%%%%
		% For a diverging nozzle:
		% Left side narrower, right side wider
		\draw[baseBlack, thick] (0,0.5) -- (2,1) -- (2,-1) -- (0,-0.5) -- cycle;

		% Flow arrow into nozzle
		\draw[line width=1.2pt, blueColor, -{Stealth[length=6pt]}] (-1.2,0) -- (0,0) 
		  node[midway, above, sloped, baseBlack]{supersonic};

		% Parameters in the supersonic nozzle
		% Green parameters (top line)
		\node[greenColor] at (1.0,0.3) {Ma $\uparrow$ \quad V $\uparrow$};

		% Red parameters (below)
		\node[redColor] at (1.0,-0.05) {P $\downarrow$ \quad T $\downarrow$ \quad $\rho \downarrow$};

		%%%%%%%%%%%%%%%%%%%%%%%%
		% Supersonic Diffuser on the right (converging duct)
		%%%%%%%%%%%%%%%%%%%%%%%%
		% For a converging diffuser:
		% Left side wider, right side narrower
		\draw[baseBlack, thick] (5,1) -- (7,0.5) -- (7,-0.5) -- (5,-1) -- cycle;

		% Flow arrow into diffuser
		\draw[line width=1.2pt, blueColor, -{Stealth[length=6pt]}] (3.8,0) -- (5,0) 
		  node[midway, above, sloped, baseBlack]{supersonic};

		% Parameters in the supersonic diffuser
		% Green parameters (top line)
		\node[greenColor] at (6.0,0.25) {T $\uparrow$ \quad P $\uparrow$ \quad $\rho \uparrow$};

		% Red parameters (below)
		\node[redColor] at (6.0,-0.05) {V $\downarrow$ \quad Ma $\downarrow$};
	\end{tikzpicture}
\caption{Change in flow properties in supersonic nozzles and diffusers \cite{Cengel2017}}
\label{fig:supersonic_nozzle_diffuser}
\end{figure}


{\color{greenColor}\itshape
	Add references! There should be two good ones!
	Go more into derivation? We at least need a explaination for: adiabatic, 1-dimensional, steady, reversibility, perfect gas, 
	Reversibility is very important since flow entering and leaving the chamber won't be fully reversible.
	Make a stark distinction between isentropic and isothermal flow.
}

