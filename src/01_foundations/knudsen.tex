	In gas dynamics a flow can be categorized by its particle interaction using the Knudsen number, which represents the ratio between the mean-free-path $\lambda$ of the gas and some characteristic length $L_c$.
	$$
		Kn=\frac{\lambda}{L_c}
	$$
	The characteristic length is usually chosen to be the smallest linear length in the system.
	For example the throat diameter of a nozzle. \cite{putignano2012supersonic}

\subsubsection*{Molecular regime (\(Kn \geq 10\))}

	In this regime, the mean free path is much larger than the dimensions of boundaries.
	This leads to particle interactions themselves becoming negligible in comparison to the interaction of particles with the boundary.
	\cite{rapp2017microfluidics}
	\input{src/01_foundations/fig_velocity-distribution-molecular.tex}

\subsubsection*{Transition regime (\(0.1 \leq Kn \leq 10\))}
	
	This regime is a middle ground between continuum and fully molecular flow.
	Neither the continuum assumptions of fluid dynamics nor the free molecular flow assumptions hold completely.
	The interactions between the gas molecules and the boundaries are significant, and the flow characteristics may vary widely.

\subsubsection*{Slip regime (\(0.001 \leq Kn \leq 0.1\))}

	For increasing Knudsen numbers the mean free path becomes comparable to the characteristic length scale of the system.
	In this regime, the assumptions for continuum flow still hold, but there are deviations, especially near the boundaries.
	While continuum mechanics assumes no-slip conditions on the boundary, in this regime, slip on the boundary must be factored in.

	\input{src/01_foundations/fig_velocity-distribution-slip.tex}

\subsubsection*{Continuum regime (\(Kn \leq 0.001\))}
	
	In this regime, the interactions of particles in the medium are much more frequent than the interactions of particles with the boundaries of the duct.
	This makes it possible to describe the fluid itself as a continuous medium with the assumption of non-slip boundary conditions.
	The Navier-Stokes equations govern the calculations in this regime.
	\input{src/01_foundations/fig_velocity-distribution-continuum.tex}
