	Isentropic varying-area flow is one of the most idealized models to describe the behavior of gases flowing through a confined space. The following assumptions are made:
	\begin{itemize}
		\item steady, one-dimensional flow
		\item adiabatic: $\delta q = 0, ds_e = 0$
		\item no shaft work: $\delta w_s = 0$
		\item negligible change in potential energy: dz = 0
		\item reversible: $ds_i = 0$
	\end{itemize}
	Being reversible as well as adiabatic, the flow is therefore isentropic.
	The Mach number is defined as the ratio between the local velocity $u$ and the local speed of sound $a$.
	$$
		Ma = \frac{u}{a}
	$$
	It is a very important metric when analyzing isentropic flow, since state variables are uniquely defined through the mach number at the corresponding location, as long as either the stagnation, or critical conditions of the flow are known.

\paragraph{Low subsonic regime (\(Ma < 0.3\))}

	For low Mach numbers, compressibility effects of a gas can be neglected, and the gas can be treated as an incompressible fluid.

\paragraph{Subsonic regime (\(0.3 < Ma < 1.0\))} 

	Inside system of variable area ducts the gas flow generally stays subsonic.
	Once sonic speed is reached in a converging duct, the behavior reverses, and the velocity decreases, limiting the flow to subsonic or sonic speeds within converging ducts.\\
	
	% src/01_foundations/fig_subsonic.tex
\begin{figure}[H]
	\centering
	\begin{tikzpicture}[font=\small, scale=1.4]

		% Titles
		\node[baseBlack] at (1,1.3) {\large Subsonic Nozzle};
		\node[baseBlack] at (6,1.3) {\large Subsonic Diffuser};

		% Nozzle on the left
		\draw[baseBlack, thick] (0,1) -- (2,0.5) -- (2,-0.5) -- (0,-1) -- cycle;

		% Flow arrow into nozzle
		\draw[line width=1.2pt, blueColor, -{Stealth[length=6pt]}] (-1.2,0) -- (0,0) 
		  node[midway, above, sloped, baseBlack]{subsonic};

		% Parameters in the nozzle
		% Green parameters (top line)
		\node[greenColor] at (1.0,0.3) {Ma $\uparrow$ \quad V $\uparrow$};

		% Red parameters (below)
		\node[redColor] at (1.0,-0.05) {P $\downarrow$ \quad T $\downarrow$ \quad $\rho \downarrow$};
		% Diffuser on the right
		\draw[baseBlack, thick] (5,0.5) -- (7,1) -- (7,-1) -- (5,-0.5) -- cycle;

		% Flow arrow into diffuser
		\draw[line width=1.2pt, blueColor, -{Stealth[length=6pt]}] (3.8,0) -- (5,0) 
		  node[midway, above, sloped, baseBlack]{subsonic};

		% Parameters in the diffuser
		% Green parameters (top line)
		\node[greenColor] at (6.0,0.25) {T $\uparrow$ \quad P $\uparrow$ \quad $\rho \uparrow$};

		% Red parameters (below)
		\node[redColor] at (6.0,-0.05) {V $\downarrow$ \quad Ma $\downarrow$};

	\end{tikzpicture}

	\caption{Change of flow properties in subsonic nozzles and diffusers}
	\label{fig:nozzle_diffuser_super}
\end{figure}
	
\paragraph{Sonic regime (\(Ma = 1\))}
	
	Sonic flow occurs at the exit of a converging duct, if the pressure ratio between two reservoirs becomes smaller than the following critical ratio.
	Which is called chocked flow and constitutes the maximum mass-flow for given stagnation conditions. 
	This ratio is defined as:
	$$
		\frac{P^*}{P_t}=\left(\frac{2}{\gamma + 1}\right)^{\gamma/(\gamma - 1)}
	$$ 
	Where $P_t$ is the stagnation condition, $P^*$ the critical back-pressure and $\gamma$ the specific heat ratio.
	The ratio is derived from the isentropic flow relation \eqref{eq:total_relation_T} and can be expressed for any state variable.
\paragraph{Supersonic regime (\(Ma > 1\))} 

	If there are critical conditions at the end of a converging duct and a diverging duct follows.
	The flow continues to accelerate and reaches supersonic speeds.
	The location where the flow reaches critical condition is called throat and represents the minimal diameter of the duct.\\
	In supersonic flows, state variables change rapidly causing phenomenons like shock waves and expansion fans.
	
	% src/01_foundations/fig_supersonic.tex 
\begin{figure}[H]
\centering
	\begin{tikzpicture}[font=\small, scale=1.4]
		% Titles
		\node[baseBlack] at (1,1.3) {\large Supersonic Nozzle};
		\node[baseBlack] at (6,1.3) {\large Supersonic Diffuser};

		%%%%%%%%%%%%%%%%%%%%%%%%
		% Supersonic Nozzle on the left (diverging duct)
		%%%%%%%%%%%%%%%%%%%%%%%%
		% For a diverging nozzle:
		% Left side narrower, right side wider
		\draw[baseBlack, thick] (0,0.5) -- (2,1) -- (2,-1) -- (0,-0.5) -- cycle;

		% Flow arrow into nozzle
		\draw[line width=1.2pt, blueColor, -{Stealth[length=6pt]}] (-1.2,0) -- (0,0) 
		  node[midway, above, sloped, baseBlack]{supersonic};

		% Parameters in the supersonic nozzle
		% Green parameters (top line)
		\node[greenColor] at (1.0,0.3) {Ma $\uparrow$ \quad V $\uparrow$};

		% Red parameters (below)
		\node[redColor] at (1.0,-0.05) {P $\downarrow$ \quad T $\downarrow$ \quad $\rho \downarrow$};

		%%%%%%%%%%%%%%%%%%%%%%%%
		% Supersonic Diffuser on the right (converging duct)
		%%%%%%%%%%%%%%%%%%%%%%%%
		% For a converging diffuser:
		% Left side wider, right side narrower
		\draw[baseBlack, thick] (5,1) -- (7,0.5) -- (7,-0.5) -- (5,-1) -- cycle;

		% Flow arrow into diffuser
		\draw[line width=1.2pt, blueColor, -{Stealth[length=6pt]}] (3.8,0) -- (5,0) 
		  node[midway, above, sloped, baseBlack]{supersonic};

		% Parameters in the supersonic diffuser
		% Green parameters (top line)
		\node[greenColor] at (6.0,0.25) {T $\uparrow$ \quad P $\uparrow$ \quad $\rho \uparrow$};

		% Red parameters (below)
		\node[redColor] at (6.0,-0.05) {V $\downarrow$ \quad Ma $\downarrow$};
	\end{tikzpicture}
\caption{Change in flow properties in supersonic nozzles and diffusers \cite{Cengel2017}}
\label{fig:supersonic_nozzle_diffuser}
\end{figure}

\subsubsection*{Relation of dimensionless numbers in continuum flow}
	It is actually possible to relate the three dimensionless numbers mentioned in the preceding chapters.
	Leading to following relation:
	$$
		Kn = \frac{Ma}{Re}\sqrt{\frac{\gamma \pi}{2}}
	$$
