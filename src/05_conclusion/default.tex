\section*{Conclusion}
\markboth{Conclusion}{Conclusion}
\addcontentsline{toc}{section}{Conclusion}
The primary goal of this thesis was to develop an accessible analytical framework to predict gas flow behavior within high-pressure micro-reactor cells, emphasizing continuum gas flows.
Through analytical modeling, the study successfully characterized the dominant flow regimes, identified critical state variables, and provided estimations of flow velocities at key reactor points.
Although the initial objectives—quantifying the impact of leakage and determining the velocity distribution after the reactor outlet—were not fully realized, the research established a solid foundational understanding of these factors and their influence on reactor behavior.

The findings demonstrated that isentropic flow assumptions could effectively capture macroscopic gas behavior, especially when considering microscale effects such as slip conditions and non-isentropic phenomena.
Nonetheless, further refinement is required, particularly in accurately modeling leak dynamics and the complex transitional behaviors observed at the reactor outlet as the gas expands into vacuum conditions.

Looking forward, detailed numerical simulations, including methods like Direct Simulation Monte Carlo (DSMC), should be pursued to better represent rarefied flow conditions and explicitly quantify leakage effects.
Additionally, empirical validation through experimental investigations is essential to refine these analytical models, thereby improving their reliability and extending their applicability.
