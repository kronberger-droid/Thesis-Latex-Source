In Section \ref{sec:one-dim-isentropic}, the flow within the reactor is modeled under the assumption of being fully isentropic and one-dimensional. This approach simplifies the reactor geometry to that of a variable area duct, resulting in a gradual change of state variables along the flow path from inlet to outlet. However, this formulation does not accurately capture the real geometry of the reactor, particularly the presence of a leak located at the perimeter. Since this location is not explicitly represented in the one-dimensional model, its influence on the flow cannot be accounted for without further simplifications.

Section \ref{sec:micro-channels} introduces several phenomena relevant to micro-scale flows that are difficult to incorporate analytically. To enable an analytical treatment of the leak and its effects, this section proposes an alternative system formulation. Here, both the inlet reservoir and the reactor chamber are modeled as separate reservoirs, each defined by their respective stagnation conditions. This simplification decouples the system into two independent problems: one describing the gas expansion from a reservoir into a discharge region, and the other relating to the reactor outlet. This approach effectively models the change in stagnation conditions across the reactor as the result of non-isentropic processes, such as those occurring when the gas enters and exits the reactor.

To reconnect the two reservoirs analytically, initially without including the leak, additional assumptions are made.
Stagnation conditions for the reactor are prescribed so that the resulting mass flow at the outlet can be balanced by the inlet conditions, without imposing sonic flow at the inlet throat.
To solve this system, a connection between the inlet reservoir conditions and those within the reactor must be established.
Two approaches are considered: First, the back pressure and temperature of the discharge region are assumed to correspond to the reactor chamber conditions and are linked isentropically to the inlet reservoir.
Second, the temperature is assumed constant, while the pressure follows the same isentropic relation.
This simplification effectively models possible heat exchange within the reactor.

Finally, the influence of an additional mass flow rate $\dot{m}_L$ leaving the reactor through the leak is considered. The impact of this additional flow on the inlet Mach number $M_{in}$ is analyzed, aiming to identify an upper limit for the leak mass flow. This limit is defined by the point at which the inlet is again forced to reach sonic conditions. Notably, this analysis does not attempt to derive an explicit formula for $\dot{m}_L$ but instead focuses on the qualitative influence of the leak on the inlet flow conditions.

\subsubsection*{Formulations without including the leak}
	In the absence of a leak, the system reduces to a single flow path from the inlet reservoir, through the reactor volume, and out to the vacuum.
	All mass entering the reactor must exit via the outlet nozzle, eliminating any additional flow paths.
	Consequently, the boundary conditions become simpler, and standard relationships between the inlet and reactor stagnation states can be applied directly.
	This section formulates those relationships and illustrates how isentropic or isothermal assumptions may be used without the complexities introduced by a leak.
	\begin{figure}[H]
	    \centering
	    \includegraphics[width=0.9\textwidth]{src/03_analytical-work/fig_disconnected-reservoirs.pdf}
	    \caption[Idealized system separating the reactor from the inlet and outlet regions to account for non-isentropic effects.]{
			\textbf{Idealized system separating the reactor from the inlet and outlet regions to account for non-isentropic effects:}
			The yellow regions represent domains where one-dimensional flow is assumed.
			Grey indicates stagnant gas, green arrow indicates subsonic flow, and the red arrow indicates supersonic flow at the outlet.
			An entropy change $\Delta s$ occurs between the gas flowing into the discharge region and the reactor, and the converging section at the outlet represents the flow constraining itself as it accelerates as mentioned in \ref{sec:geometry}.
		}
	    \label{fig:disconnected-reservoirs}
	\end{figure}
	\noindent Since this still constitutes a closed system, mass flow must be conserved, thus:
	$$
		\dot{m}_{in} = \dot{m}_r = const.
	$$
	The reactor still discharges into vacuum, this therefore must lead to chocked flow:
	$$
		M_r = 1
	$$
	From the geometry in section \ref{sec:geometry} it is given that:
	$$
		A_{in} = A_r = A
	$$
	The mass flow rates from the reactor into vacuum, can be calculated using the isentropic mass flow equation \eqref{eq:1-d-massflow}\\
	Reservoir $\to$ Reactor
	\begin{equation}
		\dot{m}_{in} = A\, p_0\, \sqrt{\frac{\gamma}{R\,T}}\, M_{in}\,\left(1+\frac{\gamma-1}{2}\,M_{in}^2\right)^{-\frac{\gamma+1}{2(\gamma-1)}}
		\label{eq:massflow-inlet}
	\end{equation}
	Reactor $\to$ Vacuum
	\begin{equation}
		\dot{m}_r = A\, p_r\, \sqrt{\frac{\gamma}{R\,T}}\,\left(1+\frac{\gamma-1}{2}\right)^{-\frac{\gamma+1}{2(\gamma-1)}}
		\label{eq:massflow-outlet}
	\end{equation}
	Thus from ($\dot{m}_r = \dot{m}_{in} = const.$)
	$$
		A\, p_r\, \sqrt{\frac{\gamma}{R\,T}}\,\left(1+\frac{\gamma-1}{2}\right)^{-\frac{\gamma+1}{2(\gamma-1)}}
		=  A\, p_0\, \sqrt{\frac{\gamma}{R\,T}}\, M_{in}\,\left(1+\frac{\gamma-1}{2}\,M_{in}^2\right)^{-\frac{\gamma+1}{2(\gamma-1)}}\\\\
	$$
	Which constitutes the general conservation of mass equation for the system without connecting the state variables of the reservoirs in any way.
	This will be the starting point for both of the following approaches.\\
\paragraph{Constant temperature approach}
	The isothermal approach similarly separates the reactor as an independent reservoir, but enforces a uniform temperature shared with the inlet reservoir.
	Despite holding temperature constant, a pressure difference can still arise, which inherently indicates a net entropy change.
	This scenario is plausible if the reactor walls maintain a uniform thermal environment, allowing the gas to remain at the same temperature while the pressure drops.
	The resulting non-isentropic behavior is thus represented by a constant-temperature condition, and simpler local flow equations remain usable, provided the irreversibilities are captured in the overall pressure imbalance between inlet and reactor. 
	$$
		T_r = T_0 = T \neq T_b
	$$
	This approach is going contrary to isentropic formulation, but still seems somehow plausible due to the fact that the gas in the reservoir just like the whole assembly will be preheated to a certain temperature.
	Since for small cavities the surface area of its walls is much larger in proportion to its internal volume, heat transfer from the walls will be significant even without high mixing due to low Reynolds numbers present in microfluidics.
	$$
		p_r\, \left(\frac{2}{\gamma+1}\right)^{\frac{\gamma+1}{2(\gamma-1)}}
		=  p_0\, M_{in}\,\left(1+\frac{\gamma-1}{2}\,M_{in}^2\right)^{-\frac{\gamma+1}{2(\gamma-1)}}
	$$
	Again using the isentropic relation \eqref{eq:total_relation_p} for $p_0$
	$$
		p_r\, \left(\frac{2}{\gamma+1}\right)^{\frac{\gamma+1}{2(\gamma-1)}}
		=  p_r \left(1 + \frac{\gamma - 1}{2}M_{in}^2 \right)^{\frac{\gamma}{\gamma - 1}}\, M_{in}\,\left(1+\frac{\gamma-1}{2}\,M^2\right)^{-\frac{\gamma+1}{2(\gamma-1)}}
	$$
	Rearranging leads to an equation only dependent on $\gamma$ and the mach number $M_{in}$ at the inlet.
	$$
		\left(\frac{2}{\gamma+1}\right)^{\frac{\gamma+1}{2(\gamma-1)}}
		=  M_{in}\,\left(1+\frac{\gamma-1}{2}\,M_{in}^2\right)^{\frac{1}{2}}
	$$
	Which can be solved analytically, since by squaring both sides and rearranging it resembles a simple quadratic equation:
	\begin{equation}
		M_{in}^4 + \frac{2}{\gamma - 1}M_{in}^2 - \frac{2}{\gamma -1}\left(\frac{2}{\gamma + 1}\right)^{\frac{\gamma + 1}{\gamma - 1}} = 0
	\end{equation}
	Now substituting $M_{in}^2 = f \quad \to \quad M_{in} = \sqrt{f}$
	\begin{equation}
		f = -\frac{1}{\gamma - 1} \pm \sqrt{\frac{1}{(\gamma - 1)^2}
		+ \frac{2}{\gamma -1}\left(\frac{2}{\gamma + 1}\right)^{\frac{\gamma + 1}{\gamma - 1}}}
	\end{equation}
	There is only one real solution for this equation using the positive square root.
	For $\gamma = 1.47$ we get the solution:
	$$
		f = 0.31
			\quad \to \quad
		M_{in} = 0.55
	$$
	Which corresponds to the following ration between the conditions in the reactor and the reservoir, which results in the following mass-flow:
	$$
		\frac{T_r}{T_0} = 1 \quad \neq \quad \frac{T_b}{T_0} = 0.93\;,
			\quad
		\frac{p_r}{p_0} = 0.79\;,
			\quad
		\frac{\rho_r}{\rho_0} \neq \frac{\rho_b}{\rho_0} = 0.85
	$$
	$$
		\dot{m}_{iso} = 8.84 \cdot 10^{-8} \; \frac{\text{kg}}{\text{s}}
			\quad \rightarrow \quad
		q_{m,\;iso} = 1.90 \cdot 10^{18} \; \frac{\text{N}_2}{\text{s}}
	$$
	at reservoir conditions of $T_{0_{max}} = 600\;\text{K}$ and $p_{0_{ref}} = 1.5\;\text{bar}$.
	Which is in the same order of magnitude as the mass-flow for the fully isentropic formulation in Section \ref{sec:one-dim-isentropic}.
	The discrepancy in mass-flow can be attributed to the change in entropy, indicated by the change in stagnation conditions.
	Also, now that the temperature of the discharge area and the reactor don't match anymore, heat transfer has to be assumed.

\paragraph{Isentropic approach}
	In this approach, the reactor is treated as a downstream reservoir with distinct stagnation conditions, implying that irreversibilities have occurred somewhere in the flow path.
	Instead of modeling the associated entropy generation, local isentropic relations are used to connect the inlet reservoir’s stagnation state to the reactor boundary.
	This method effectively embeds any non-isentropic effects into a “shift” in stagnation values, so the reactor’s reduced pressure and/or temperature serve as an isentropic back condition.
	As a result, flow properties in the nozzles or ducts can still be obtained from standard one-dimensional isentropic equations, even though the actual flow inside the reactor may be more complex.
	$$
		T_b = T_r
		\quad \to \quad
		T_0 = T_r \left(1 + \frac{\gamma - 1}{2}M_{in}^2 \right)
	$$
	and
	$$
		p_b = p_r
		\quad \to \quad
		p_0 = p_r \left(1 + \frac{\gamma - 1}{2}M_{in}^2 \right)^{\frac{\gamma}{\gamma - 1}}
	$$
	\newpage
	Assuming the conditions of the discharge region to match the values in the reactor and using the isentropic relations to express $T_0$ and $p_0$ in terms of $T_b$ and $P_b$ the previously defined conservation of mass equation yields:
	$$
		A\, p_r\, \sqrt{\frac{\gamma}{R\,T_r}}\,\left(1+\frac{\gamma-1}{2}\right)^{-\frac{\gamma+1}{2(\gamma-1)}}\\
	$$
	$$
		=
	$$
	$$
		A\, p_r\left(1 + \frac{\gamma - 1}{2}M_{in}^2\right)^{\frac{\gamma}{\gamma-1}}\, \sqrt{\frac{\gamma}{R\,T_r}}\left(1 + \frac{\gamma - 1}{2}M_{in}\right)^{-\frac{1}{2}}\, M_{in}\,\left(1+\frac{\gamma-1}{2}\,M_{in}^2\right)^{-\frac{\gamma+1}{2(\gamma-1)}}
	$$
	Canceling out variables present on both sides and combining the potential expressions on the right side it results a function of $M_{in}$ only dependent on $\gamma$, since the summing the exponents equals zero.
	$$
		M_{in} = \left(1 + \frac{\gamma - 1}{2}\right)^{-\frac{\gamma + 1}{2(\gamma - 1)}}= 0.57
		\quad \text{for} \quad \gamma = 1.47
	$$
	Thus resulting in following ratios between the conditions in the reactor and the reservoir and a corresponding mass flow.
	$$
		\frac{T_b}{T_0} = \frac{T_r}{T_0} = 0.93\;,
			\quad
		\frac{p_b}{p_0} = \frac{p_r}{p_0} = 0.79\;,
			\quad
		\frac{\rho_b}{\rho_0} = \frac{\rho_r}{\rho_0} = 0.85
	$$
	$$
		\dot{m}_{isen} = 9.04 \cdot 10^{-8}\; \frac{\text{kg}}{\text{s}}
			\quad \rightarrow \quad
		q_{m,\;isen} = 1.94 \cdot 10^{18}\; \frac{\text{N}_2}{\text{s}} 
	$$
	At reservoir conditions of $T_{0_{max}} = 600\;\text{K}$ and $p_{0_{ref}} = 1.5\;\text{bar}$.

\subsubsection*{Formulations Including the Leak}
	Including the leak can be accounted for by assuming an additional mass flow leaving the reactor, which the inlet must compensate:
	\begin{equation}
	    \dot{m}_{in} = \dot{m}_r + \dot{m}_L 
	    \quad \rightarrow \quad 
	    \dot{m}_L = \dot{m}_{in} - \dot{m}_r \label{eq:massflow-leak}
	\end{equation}
	This requires an increase in the total conditions of the reservoir to maintain constant conditions inside the reactor.
	As a consequence, the Mach number increases until the critical pressure ratio \eqref{eq:critical-pressure} is reached, resulting in sonic flow at the inlet throat.

	\begin{figure}[H]
	    \centering
	    \includegraphics[width=0.9\textwidth]{src/03_analytical-work/fig_disconnected-reservoirs-with-leak.pdf}
	    \caption[Idealized system separating the reactor from the inlet and outlet regions to account for non-isentropic effects, now including a leak path from the reactor.]{
	        \textbf{Idealized system separating the reactor from the inlet and outlet regions to account for non-isentropic effects, including a leak path from the reactor:}
	        The yellow regions represent domains where one-dimensional flow is assumed, gray indicates stagnant gas, and red arrows indicate supersonic flow.
	        An entropy change $\Delta s$ occurs between the discharge region and the reactor, while the converging section at the outlet represents the constriction of the flow as it accelerates, as discussed in Section \ref{sec:geometry}.
	    }
	    \label{fig:disconnected-reservoirs-leak}
	\end{figure}

	Using the isentropic mass flow Equations~\eqref{eq:massflow-inlet} and \eqref{eq:massflow-outlet}, the effect of the leak can be described by assuming $M_{in} = 1$ and substituting into Equation~\eqref{eq:massflow-leak}.
	Yielding:
	$$
	    \dot{m}_L =
	    A\, p_0\, \sqrt{\frac{\gamma}{R\,T}}\, \left(1+\frac{\gamma-1}{2}\right)^{-\frac{\gamma+1}{2(\gamma-1)}}
	    - A\, p_r\, \sqrt{\frac{\gamma}{R\,T}}\, \left(1+\frac{\gamma-1}{2}\right)^{-\frac{\gamma+1}{2(\gamma-1)}}
	$$
	To relate the reservoir temperature $T_0$ and pressure $p_0$ to the reactor conditions, it is again assumed that the conditions in the discharge region and the reactor are identical, thus connecting them through Equations~\eqref{eq:total_relation_T} and \eqref{eq:critical-pressure}.
	This formulation is chosen because it has lead to a higher Mach number at the inlet and therefore reaches sonic conditions sooner.
	It represents a "worst-case" scenario, as it results in the lowest additional mass flow required to choke the inlet.
	After substituting and rearranging the equation yields:
	\begin{equation}
	    \dot{m}_L =
	    A\, p_r\, \sqrt{\frac{\gamma}{R\, T_r}}\,
	    \left(1 + \frac{\gamma - 1}{2}\right)^{-\frac{\gamma + 1}{2(\gamma - 1)}}
	\end{equation}
	Assuming reservoir conditions of $p_{0_{ref}} = 1.5\;\text{bar}$ and $T_{0_{max}} = 600\;\text{K}$, the corresponding reservoir state can be calculated using the critical ratios discussed in Section \ref{sec:isentropic-1D-foundations}.
	The additional mass flow through the leak required to cause sonic conditions at the inlet is then:
	$$
	    \dot{m}_L = 4.68 \cdot 10^{-8} \; \frac{\text{kg}}{\text{s}}
			\quad \rightarrow \quad
		q_{m,L} = 1.00 \cdot 10^{18} \; \frac{(\text{N}_2)}{\text{s}}
	$$
	This value is approximately half of the mass flow calculated for the disconnected reservoir model.
	It can thus be concluded that if the mass flow through the leak is comparable to the mass flow through the outlet, this will force the inlet throat to go sonic.
	This can induce significant changes in the gas behavior downstream of the inlet, such as shock waves and rapid variations in state variables.

	Since this setup still constitutes a coupled system, where the leak influences the outlet, for instance, via pressure losses, and vice versa, the exact distribution of the mass flow between the two, as well as the reservoir conditions at which the inlet becomes sonic, must be further investigated through numerical simulation.
