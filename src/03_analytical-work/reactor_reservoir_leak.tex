In section \ref{sec:one-dim-isentropic} the flow is assumed to be fully isentropic and one-dimensional.
This led to the simplification of the reactor being a variable area duct, which leads to a change in state-variables along the path of the gas flowing from the inlet to the outlet.
Since this change is based on the variable area duct formulation, it won't represent an accurate solution for the given geometry.
The leak is located at the perimeter of the reactor, which is not an actual location in the current formulation.
To be able to include the influence of the leak analytically, without using numerical tools, further simplification is needed.
Thus, this section will explore a formulation of the system, where both the inlet reservoir and the reactor are assumed to be reservoirs, with stagnation conditions associated with them.
Representing a change in stagnation conditions which can be attributed to non-isentropic processes happening when the gas enters or leaves the reactor.

\begin{figure}[H]
    \centering
    \includegraphics[width=0.9\textwidth]{src/03_analytical-work/fig_disconnected-reservoirs.pdf}
    \caption{A descriptive caption for the figure.}
    \label{fig:disconnected-reservoirs}
\end{figure}




For an isentropic flow, the mass flow rate is given by

$$
	\dot{m} = A_i\, p_t\, \sqrt{\frac{\gamma}{R\,T_t}}\, M\,\left(1+\frac{\gamma-1}{2}\,M^2\right)^{-\frac{\gamma+1}{2(\gamma-1)}}
$$

From Reactor to Vacuum

$$
	\dot{m}_r = A_r\, p_r\, \sqrt{\frac{\gamma}{R\,T_r}}\, M\,\left(1+\frac{\gamma-1}{2}\,M^2\right)^{-\frac{\gamma+1}{2(\gamma-1)}}
$$

From Reservoir to Reactor

$$
	\dot{m}_0 = A_0\, p_0\, \sqrt{\frac{\gamma}{R\,T_0}}\, M_0\,\left(1+\frac{\gamma-1}{2}\,M_0^2\right)^{-\frac{\gamma+1}{2(\gamma-1)}}
$$

Assumptions

$$
	\dot{m}_0 = \dot{m}_r = const.\quad T_0 = T_r,\quad M_r = 1, \quad A_0 = A_r
$$

Thus

$$
	p_r\, \left(\frac{2}{\gamma+1}\right)^{\frac{\gamma+1}{2(\gamma-1)}}
	= p_0\, M_0\,\left(1+\frac{\gamma-1}{2}\,M_0^2\right)^{-\frac{\gamma+1}{2(\gamma-1)}}.
$$

Using the isentropic relation

$$
	p_0 = p_r \left(1 + \frac{\gamma - 1}{2} M_0^2\right)^{\frac{\gamma}{\gamma - 1}}
$$

Then

$$
	0 = 
	M_0\,\left(1+\frac{\gamma-1}{2}\,M_0^2\right)^{\frac{1}{2}}
	- \left(\frac{2}{\gamma+1}\right)^{\frac{\gamma+1}{2(\gamma-1)}}
$$

If we assume not just the outlet contributes to the mass-flow but also the leak we have to add 

