In section \ref{sec:one-dim-isentropic} the flow is assumed to be fully isentropic and one-dimensional.
This led to the simplification of the reactor being a variable area duct, which leads to a change in state-variables along the path of the gas flowing from the inlet to the outlet.
Since this change is based on the variable area duct formulation, it won't represent an accurate solution for the given geometry.
The leak is located at the perimeter of the reactor, which is not an actual location in this formulation.
To be able to include the influence of the leak analytically, without using numerical tools, further simplification is needed.

Thus, this section will explore a formulation of the system, where both the inlet reservoir and the reactor are assumed to be reservoirs, with stagnation conditions associated with them.
In turn disconnecting the two reservoirs and resulting in two independent problems, describing gas from a reservoir flowing into a discharge region with a given back pressure. 
Representing a change in stagnation conditions which can be attributed to non-isentropic processes happening when the gas enters or leaves the reactor.

Now proper assumptions have to be taken to connect both reservoirs again, first without inclusion of the leak.
Stagnation conditions for the reactor are chosen, resulting in a mass-flow at the outlet.
Which in turn have to be compensated by the inlet, thus mach number and the conditions inside the reservoir can be derived.  
Afterward two possible formulations including the leak will be presented.

\subsubsection*{Formulations without including the leak}

\begin{figure}[H]
    \centering
    \includegraphics[width=0.9\textwidth]{src/03_analytical-work/fig_disconnected-reservoirs.pdf}
    \caption{A descriptive caption for the figure.}
    \label{fig:disconnected-reservoirs}
\end{figure}

\textbf{Assumptions}\\
Since this still constitutes a closed system, mass flow must be conserved, thus:
$$
	\dot{m}_0 = \dot{m}_r = const.
$$
The reactor still discharges into vacuum, this therefore must lead to chocked flow:
$$
	M_r = 1
$$
From the geometry it is given that:
$$
	A_0 = A_r = A
$$
Lastly, to be able to reduce this problem to one unknown, the stagnation conditions of the reservoir and the reactor have to be related, thus reducing the unknowns to just $M_0$.
One obvious possibility is to use the isentropic relations (equations \eqref{eq:total_relation_T} - \eqref{eq:total_relation_rho}), for both Temperature and Pressure:
$$
	T_b = T_r
	\quad \to \quad
	T_0 = T_r \left(1 + \frac{\gamma - 1}{2}M_0^2 \right)
$$
and
$$
	p_b = p_r
	\quad \to \quad
	p_0 = p_r \left(1 + \frac{\gamma - 1}{2}M_0^2 \right)^{\frac{\gamma}{\gamma - 1}}
$$
Disconnecting the inlet and outlet gives the ability to define non-isentropic relations between the two reservoirs.
Thus, as the second approach Temperature will be held constant between both reservoirs. 
$$
	T_r = T_0 = T
$$
This approach is going contrary to isentropic formulation, but still seems somehow plausible due to the fact that the gas in the reservoir just like the whole assembly will be preheated to a certain temperature.
Since for small cavities the surface area of its walls is much larger in proportion to its internal volume, heat transfer from the walls will be significant even without high mixing due to low Reynolds numbers present in microfluidics.\\
\textbf{Calculation}\\
The mass flow rates from the reactor into vacuum, can be calculated using the isentropic mass flow equation:\\
Reservoir $\to$ Reactor
$$
	\dot{m}_0 = A\, p_0\, \sqrt{\frac{\gamma}{R\,T}}\, M_0\,\left(1+\frac{\gamma-1}{2}\,M_0^2\right)^{-\frac{\gamma+1}{2(\gamma-1)}}
$$
Reactor $\to$ Vacuum
$$
	\dot{m}_r = A\, p_r\, \sqrt{\frac{\gamma}{R\,T}}\,\left(1+\frac{\gamma-1}{2}\right)^{-\frac{\gamma+1}{2(\gamma-1)}}
$$
Thus from ($\dot{m}_r = \dot{m}_0 = const.$)
$$
	A\, p_r\, \sqrt{\frac{\gamma}{R\,T}}\,\left(1+\frac{\gamma-1}{2}\right)^{-\frac{\gamma+1}{2(\gamma-1)}}
	=  A\, p_0\, \sqrt{\frac{\gamma}{R\,T}}\, M_0\,\left(1+\frac{\gamma-1}{2}\,M_0^2\right)^{-\frac{\gamma+1}{2(\gamma-1)}}\\\\
$$
Which constitutes the general conservation of mass equation for the system without connecting the state variables of the reservoirs in any way.
This will be the starting point for both of the following approaches.\\
\textbf{Isentropic approach}\\
Using the isentropic relations to express $T_0$ and $p_0$ in terms of $T_r$ and $P_r$ the previously defined conservation of mass equation yields:
\begin{align*}
	&A\, p_r\, \sqrt{\frac{\gamma}{R\,T_r}}\,\left(1+\frac{\gamma-1}{2}\right)^{-\frac{\gamma+1}{2(\gamma-1)}}\\
	&= A\, p_r\left(1 + \frac{\gamma - 1}{2}M_0^2\right)^{\frac{\gamma}{\gamma-1}}\, \sqrt{\frac{\gamma}{R\,T_r}}\left(1 + \frac{\gamma - 1}{2}M_0\right)^{-\frac{1}{2}}\, M_0\,\left(1+\frac{\gamma-1}{2}\,M_0^2\right)^{-\frac{\gamma+1}{2(\gamma-1)}}\\\\
\end{align*}
Canceling out variables present on both sides and combining the potential expressions on the right side it results a function of $M_0$ only dependent on $\gamma$, since the summing the exponents equals zero.
$$
	M_0 = (1 + \frac{\gamma - 1}{2})^{-\frac{\gamma + 1}{2(\gamma - 1)}}
	\quad \text{with} \quad \gamma = 1.47 \quad \to \quad
	M_0 = 0.57
$$
Thus resulting in following ratios between the conditions in the reactor and the reservoir.
$$
	\frac{T_r}{T_0} = ... , \quad \frac{p_r}{T_0} = ... , \quad \frac{\rho_r}{\rho_0} = ...
$$
\textbf{Constant temperature approach}\\
Starting from the general conservation of mass equation, but choosing the temperatures to be equal yields:
$$
	p_r\, \left(\frac{2}{\gamma+1}\right)^{\frac{\gamma+1}{2(\gamma-1)}}
	=  p_0\, M_0\,\left(1+\frac{\gamma-1}{2}\,M_0^2\right)^{-\frac{\gamma+1}{2(\gamma-1)}}
$$
Again using the isentropic relation \eqref{eq:total_relation_p} for $p_0$
$$
	p_r\, \left(\frac{2}{\gamma+1}\right)^{\frac{\gamma+1}{2(\gamma-1)}}
	=  p_r \left(1 + \frac{\gamma - 1}{2}M_0^2 \right)^{\frac{\gamma}{\gamma - 1}}\, M_0\,\left(1+\frac{\gamma-1}{2}\,M_0^2\right)^{-\frac{\gamma+1}{2(\gamma-1)}}
$$
Rearranging leads to an equation only dependent on $\gamma$ and the mach number $M_0$ at the inlet.
$$
	\left(\frac{2}{\gamma+1}\right)^{\frac{\gamma+1}{2(\gamma-1)}}
	=  M_0\,\left(1+\frac{\gamma-1}{2}\,M_0^2\right)^{\frac{1}{2}}
$$
Which can be solved analytically, since by squaring both sides and rearranging it resembles a simple quadratic equation:
$$
	M_0^4 + \frac{2}{\gamma - 1}M_0^2 - \frac{2}{\gamma -1}\left(\frac{2}{\gamma + 1}\right)^{\frac{\gamma + 1}{\gamma - 1}} = 0
$$
Now substituting $M_0^2 = f \quad \to \quad M_0 = \sqrt{f}$
$$
	f = -\frac{1}{\gamma - 1} \pm \sqrt{\frac{1}{(\gamma - 1)^2}
	+ \frac{2}{\gamma -1}\left(\frac{2}{\gamma + 1}\right)^{\frac{\gamma + 1}{\gamma - 1}}}
$$
There is only one real solution for this equation using the positive square root.
For $\gamma = 1.47$ we get the solution:
$$
	f \approx 0.31 \quad \to \quad M_0 \approx 0.55
$$

\subsubsection*{Formulations including the leak}

\begin{figure}[H]
    \centering
    \includegraphics[width=0.9\textwidth]{src/03_analytical-work/fig_disconnected-reservoirs-with-leak.pdf}
    \caption{A descriptive caption for the figure.}
    \label{fig:disconnected-reservoirs-leak}
\end{figure}

