Going from macroscale channels to microscales introduces significant changes in the behavior of gas flows.
A primary factor causing these differences is the phenomenon of velocity slip at surfaces.
At small characteristic length scales, the Knudsen number ($Kn$)—which characterizes the ratio of molecular mean free path to the characteristic dimension of the flow—becomes relatively high ($Kn > 0.001$), placing the flow in the compressible regime with slip boundary conditions.

Most microscale flow behaviors require detailed studies through complex simulations or experiments, as the underlying mechanisms are often not fully explained analytically.
This section, therefore, does not provide definitive formulations but instead aims to summarize and reference critical phenomena relevant to microfluidics, primarily drawing from the comprehensive review by Amit Agrawal \cite{agrawal_comprehensive_2011}.

\subsubsection*{Phenomenon of Slip}
	In contrast to the classical no-slip condition, the slip boundary condition assumes a non-zero tangential velocity of gas molecules at the wall.
	Maxwell proposed a theoretical model for slip by considering a hypothetical control surface $s$ located half a mean free path from the wall.
	At this surface, half the gas molecules originate from one mean free path away with a tangential velocity $u_\lambda$, while the other half reflect from the wall surface.
	A fraction $\sigma$ of the reflected molecules undergo diffuse reflection (taking on the wall velocity $u_w$), whereas the remaining fraction $(1 - \sigma)$ is reflected specularly (retaining their incoming velocity $u_\lambda$).

	Expanding $u_\lambda$ as a second-order Taylor series yields the second-order slip boundary condition commonly employed in continuum analyses:
	\begin{equation}
		u_g - u_w =
		\left[
			\frac{2-\sigma}{\sigma}Kn\left(\frac{\partial u}{\partial n}\right)_s
			+ \frac{Kn^2}{2} \left(\frac{\partial^2 u}{\partial n^2}\right)_s
		\right]
	\end{equation}

	Here, $u$ denotes the streamwise velocity, and subscripts $g$, $w$, and $s$ indicate gas, wall, and control surface, respectively, while $n$ represents the direction normal to the surface.
	Crucially, $\sigma$ denotes the tangential momentum accommodation coefficient (TMAC).

	Determining the TMAC accurately for a specific application is critical, as it directly impacts velocity slip at the gas-wall interface, thus significantly influencing rarefied flow behavior.
	Typically, TMAC values are obtained experimentally or from computational methods such as direct simulation Monte Carlo (DSMC), as discussed further in the next section.

	Additionally, near-wall flow at high Knudsen numbers leads to the formation of a thin region called the Knudsen layer, extending approximately one mean free path from the wall.
	Within this layer, non-equilibrium effects dominate, and continuum assumptions—including traditional slip conditions—become increasingly inaccurate.
	Consequently, explicit modeling of the Knudsen layer is necessary for highly rarefied conditions, typically by introducing modified boundary conditions or employing higher-order molecular models.

\subsubsection*{Sudden Expansion or Contraction}
	When flow exits from a confined duct into the reactor, the geometry abruptly transitions, resembling a sudden expansion or contraction.
	As shown by Agrawal \cite{agrawal_comprehensive_2011}, under these conditions, the assumption of strict isentropic behavior is no longer valid.
	The rapid geometric changes induce phenomena such as flow separation, formation of vena contracta regions, and associated pressure and momentum losses.

	These non-isentropic effects result in local changes to the stagnation conditions downstream of the geometric discontinuities, meaning static pressure and temperature become influenced by the complex expansion dynamics rather than solely by inlet conditions.
	Treating the reactor as a reservoir with distinct stagnation conditions allows the analytical framework to accommodate these additional effects.
	Within this approach, mass flow through the reactor becomes influenced by both geometric configuration and the additional losses due to sudden expansions or contractions.

\subsubsection*{Surface Roughness}
	Surface roughness significantly influences effective slip conditions at microscale channel walls, as detailed by Agrawal \cite{agrawal_comprehensive_2011}.
	At small scales, even slight surface imperfections enhance momentum exchange between gas molecules and the channel wall.
	This increased interaction effectively reduces the slip length relative to idealized smooth surfaces predicted by Maxwell's model.

	Enhanced wall friction due to roughness impacts the pressure distribution and alters local stagnation conditions within the flow.
	Within the reactor-reservoir modeling approach, surface roughness must therefore be considered through appropriate modifications to boundary conditions, typically by adjusting the TMAC or employing higher-order slip corrections.
	Including such roughness effects enables more realistic predictions of flow behavior, particularly in regions with significant deviations from classical isentropic relationships.
	\cite{agrawal_comprehensive_2011, wang_analyses_2008}
