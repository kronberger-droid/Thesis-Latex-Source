Once the gas emerges from the reactor and enters the outlet nozzle, it undergoes a rapid expansion into a low-pressure or near-vacuum environment.
This regime differs significantly from the upstream flow: velocity increases dramatically, density drops, and shock waves can appear.
The one-dimensional isentropic models used until now cannot fully capture these effects, especially if there is phase change (e.g., condensation) or a transition to free-molecular flow.
Hence, more advanced modeling approaches are required.
\cite{jousten_handbook_2016, anderson2021modern}

\subsubsection*{Method of characteristics}

	\paragraph*{Ideal nozzle design}
		The method of characteristics is a mathematical technique used to design supersonic nozzles so that gas flows expand smoothly from sonic to supersonic speeds without generating internal shocks.
		It works by tracing characteristic lines—paths along which flow properties remain constant—through the nozzle region.
		By aligning the walls with these lines it ensures that each incremental flow turn occurs through a series of controlled expansion waves, rather than abrupt angle changes that could cause shocks.
		If fully expanded this leads to a straight column of gas leaving the nozzle, where all of its energy is converted into kinetic energy without significant losses due to shocks.
		This approach is not feasible in this case since the length of the nozzle grows as the back pressure gets lower, the ideal nozzle for the high vacuum found in the surrounding chamber would not be able to fit due to space constraints.
		\cite{khare_rocket_2021, fernandes_shape_2023}

	\paragraph*{Determining flow field for free expansion}
		Mach lines are a concept usually encountered in two-dimensional supersonic flows. 
		These lines occur in pairs and are oriented at an angle $\mu = \arcsin\!\bigl(\frac{1}{M}\bigr)$ called the Mach angle, with respect to the direction of motion.
		In a 3-D flow field, these lines form a Mach cone, whose half-angle is the Mach angle itself.
		In an under-expanded jet, the method of characteristics can also be applied immediately downstream of the nozzle exit to predict the initial spread of the flow.
		It does this by mapping local Mach numbers and density profiles along characteristic lines, thus giving 
		a first approximation for velocity and pressure distributions.
		In supersonic expansions, Prandtl–Meyer expansion fans define the maximum turning angle $\theta_{max}$ the gas can undergo around a sharp corner.
		Which can be calculated using the following equation:
		\begin{equation*}
			\theta_{max} = \nu_{max} - \nu(Ma)
		\end{equation*}
		\begin{equation}
			\theta_{max} = \frac{\pi}{2} \left(\sqrt{\frac{\gamma + 1}{\gamma - 1}} - 1\right) - \sqrt{\frac{\gamma + 1}{\gamma - 1}}\arctan{\sqrt{\frac{\gamma - 1}{\gamma + 1}(Ma^2 - 1)}} + \arctan{\sqrt{Ma^2 -1}}
			\label{eq:max-turning-angle}
		\end{equation}
		Where $\nu_{max},\nu(Ma)$ are the Prandl-Meyer functions, $Ma$ is the mach number before the turn and $\gamma$ is the ratio of heats for the given gas.
		By incorporating these fans, the method of characteristics predicts the outer boundary of the flow—effectively identifying the angle outside which the density and mass flux become negligible.
		This angle is critical for determining the lateral extent of the under-expanded plume and for estimating where essentially all of mass flow is confined.
		Using equation \eqref{eq:max-turning-angle} for the mach number at the outlet nozzle exit calculated in section \ref{sec:one-dim-isentropic} the maximum turning angle yields:
		$$
			\theta_{max} = 68.9^\circ
			\qquad \text{with} \qquad
			Ma = 3.06\;,\quad \gamma = 1.47
		$$
		Notably, findings from Cassanova and Stephenson \cite{Cassanova1965} — even though the nozzle half-angle does not match — still offer valuable reference points regarding the resulting velocity field and associated flow features.
		However, a key limitation arises if condensation or other non-ideal processes occur.
		Experiments show that for working fluids like nitrogen, condensation in the expansion region can alter the thermodynamic state and produce deviations from these idealized solutions.
		Consequently, while the method of characteristics remains valuable for nozzle design and near-exit expansions, it cannot fully capture condensation effects or strong rarefaction phenomena further downstream.
		\cite{jousten_handbook_2016, robertson_investigation_1970, noauthor_zucrow_nodate}

\subsubsection*{Navier-Stokes Equations}
	A second option is to solve the Navier–Stokes equations numerically, using no-slip or first/second-order slip boundary conditions as appropriate.
	Near the nozzle exit—where the flow is still predominantly continuum—Navier–Stokes simulations can provide detailed velocity and pressure fields, including shock structures and boundary-layer effects.
	As the gas expands and its density decreases, however, continuum assumptions start to break down.
	While slip or transitional models can extend the validity range, ultimately they cannot fully describe the molecular-level interactions that dominate in highly rarefied regions.
	Consequently, beyond a certain distance from the nozzle exit or under conditions favorable to partial condensation, Navier–Stokes solutions must be supplemented by more advanced kinetic-based methods.
	\cite{anderson_fundamentals_2017, anderson2021modern}

\subsubsection*{Direct simulation Monte Carlo (DSMC)}
	To address the shortcomings of both the method of characteristics (under non-ideal conditions) and Navier–Stokes (when rarefaction becomes significant), the Direct Simulation Monte Carlo (DSMC) method is indispensable.
	DSMC performs a stochastic simulation of molecular motion and collisions, making it particularly well suited for:
	\begin{itemize}
		\item Transitional Regimes: Where the mean free path is comparable to or larger than the nozzle scale, DSMC accurately captures Knudsen-layer effects and velocity-slip phenomena.
		\item Condensation Phenomena: By including collision physics and intermolecular potentials, DSMC can handle the onset of condensation or two-phase flow more reliably than continuum or purely isentropic approaches. \cite{EMMONS1958}
		\item Free Expansion: As the jet spreads into the vacuum and transitions toward free-molecular flow, DSMC seamlessly handles extremely high Mach numbers and very low densities, maintaining physical fidelity across a wide range of Knudsen numbers.
	\end{itemize}
	Although more computationally expensive than continuum-based methods, DSMC provides the most robust depiction of outlet flows—particularly for condensable gases like nitrogen—and ensures high accuracy from near-nozzle regions all the way to the fully rarefied state.
	\cite{putignano2012supersonic, liu_study_2006}
	\newpage
\subsubsection*{Summary}
	Overall, while the method of characteristics remains a cornerstone of ideal nozzle design and near-exit expansions (including the prediction of Prandtl–Meyer fans and maximum turning angles), condensation and severe rarefaction invalidate its purely isentropic assumption further downstream.
	Navier–Stokes simulations can capture intermediate regimes near the exit, but eventually fail in the highly rarefied or partially condensed flow domain.
	For a comprehensive treatment of under-expanded plumes—particularly with nitrogen or other condensable working fluids—the DSMC technique offers the necessary level of detail, bridging the gap from continuum flow at the nozzle exit to free-molecular flow far downstream.
