After the gas reaches sonic speeds at the throat of the outlet nozzle, it expands in the diverging outlet nozzle.
This leads to a steady pressure drop as the gas reaches the exit plane of the nozzle.
There are three distinctive expansion patterns characterized by the difference in pressure between the exit of the nozzle and the back pressure of the chamber.

If these values match, this is called fully expanded, which leads to a straight column of gas leaving the nozzle and no shock waves being created.
Is the pressure of the gas leaving the nozzle lower than the back-pressure.
The gas inside gives way to the back-pressure, leading to oblique shockwaves forming at the exit, compressing the column of gas, this is called under-expanded.
If the back-pressure is now lower than the pressure of the gas, in this over-expanded state the gas leaving the nozzle further expands over the edges of the exit, creating what is called expansion fans.

These fans are often called prandl-meyer fans and occur when super-sonic flow has to turn around a sharp edge.
\note{
	brandl-meyer angle and maximum angle. More information about connection with caracteristic lines, also explanation on the process of rarification and transistion regime. 
}

\subsubsection*{Method of characteristics}

\paragraph*{Use for nozzle design}

The method of characteristics is a mathematical technique used to design supersonic nozzles so that gas flows expand smoothly from sonic to supersonic speeds without generating internal shocks.
It works by tracing characteristic lines—paths along which flow properties remain constant—through the nozzle region.
By aligning the walls with these lines it ensures that each incremental flow turn occurs through a series of controlled expansion waves, rather than abrupt angle changes that could cause shocks.
If fully expanded this leads to a straight column of gas leaving the nozzle, where all of its energy is converted into kinetic energy without significant losses due to shocks.

\paragraph*{}
\subsubsection*{Navier-Stokes Equations}
\note{
	Fundamental equations for solving continuum flow problems. No information about boundary layer thickness and boundary conditions. These need simulations like DSMC to define boundary layer  
}
\subsubsection*{Direct simulation Monte Carlo (DSMC)}
\note{
	Fundamentals of montecarlo simulations in general. How does it solve continuum regime even tho it uses molecular simulations. Use for boundary layer approximations.
}
\cite{putignano2012supersonic}
\cite{liu_study_2006}
\cite{robertson_investigation_1970}
