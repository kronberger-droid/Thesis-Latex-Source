Once the gas emerges from the reactor and enters the outlet nozzle, it undergoes a rapid expansion into a low-pressure or near-vacuum environment.
This regime differs significantly from the upstream flow: velocity increases dramatically, density drops, and shock waves can appear.
The one-dimensional isentropic models used until now cannot fully capture these effects, especially if there is phase change (e.g., condensation) or a transition to free-molecular flow.
Hence, more advanced modeling approaches are required.
\cite{jousten_handbook_2016, anderson2021modern}

\subsubsection*{Method of characteristics}

	\paragraph*{Ideal nozzle design}
		The method of characteristics is a mathematical technique used to design supersonic nozzles so that gas flows expand smoothly from sonic to supersonic speeds without generating internal shocks.
		It works by tracing characteristic lines—paths along which flow properties remain constant—through the nozzle region.
		By aligning the walls with these lines it ensures that each incremental flow turn occurs through a series of controlled expansion waves, rather than abrupt angle changes that could cause shocks.
		If fully expanded this leads to a straight column of gas leaving the nozzle, where all of its energy is converted into kinetic energy without significant losses due to shocks.
		This approach is not feasible in this case since the length of the nozzle grows as the back pressure gets lower, the ideal nozzle for the high vacuum found in the surrounding chamber would not be able to fit due to space constraints.
		\cite{khare_rocket_2021, fernandes_shape_2023}

	\paragraph*{Determining Flow Field for Free Expansion}
		In supersonic flows, Mach lines form at a specific angle called the Mach angle, given by $\mu = \arcsin\left(\frac{1}{M}\right)$, relative to the flow direction.
		In three-dimensional flows, these Mach lines form a Mach cone with a half-angle equal to the Mach angle.
		The method of characteristics is a useful analytical tool for predicting the initial flow properties, including local Mach numbers, densities, velocities, and pressures, immediately after the nozzle exit.

		During supersonic expansions, gas flows experience changes in direction described by Prandtl–Meyer expansion fans.
		These fans set the maximum angle, $\theta_{max}$, that a gas can turn around a sharp corner, defining the outer boundary beyond which gas density and mass flow become very small.
		The maximum turning angle is calculated by:
		\begin{equation}
			\theta_{max} = \frac{\pi}{2} \left(\sqrt{\frac{\gamma + 1}{\gamma - 1}} - 1\right) - \sqrt{\frac{\gamma + 1}{\gamma - 1}}\arctan{\sqrt{\frac{\gamma - 1}{\gamma + 1}(M^2 - 1)}} + \arctan{\sqrt{M^2 -1}}
			\label{eq:max-turning-angle}
		\end{equation}

		Using Equation \eqref{eq:max-turning-angle} with a Mach number of $M = 3.06$ and $\gamma = 1.47$, we find a maximum turning angle of $\theta_{max} = 68.9^\circ$.
		This angle marks the theoretical limit for gas expansion after leaving the nozzle.

		\begin{figure}[H]
			\centering
		    \includegraphics[width=0.9\textwidth]{src/03_analytical-work/fig_velocity-distribution-reduced.pdf}
			\caption[Method of Characteristics Flow Field Solution \cite{Cassanova1965}]{
				\textbf{Method of Characteristics Flow Field Solution \cite{Cassanova1965}:}
				for values of $A_0/A^* = 3.73,\;\gamma=1.4,M_e =2.88$ a half-angle of $\theta = 15^\circ$.
				The black curves represent isochor lines, the red curve shows rotational freezing for reservoir pressure $~1\;\text{atm}$ -- which corresponds to the pressure in the reactor volume in our case -- and the blue curves indicate fractions of the total mass flow enclosed between each curve and the $x$-axis. 
				This figure has been reduced for readability; see Appendix~\ref{apx:flow-field} for more details.
			}
		\end{figure}

		Rotational freezing occurs in rapidly expanding gases when molecules no longer exchange rotational energy effectively, due to fewer collisions.
		Typically, rotational energy stops changing after about ten collisions, compared to just one collision for translational energy.
		When rotational freezing happens, the rotational temperature separates from the translational temperature, affecting the gas's pressure and density profiles significantly.
		This separation means that predictions near the nozzle exit become less accurate if rotational freezing isn't considered.
		Rotational freezing happens before translational freezing, meaning rotational energy modes stop adjusting earlier, while translational motion continues to change further downstream.

		Boundary layer effects, caused by friction near nozzle walls, also influence gas flow.
		These effects decrease the exit Mach number and reduce density at the edges of the jet.
		The boundary layer can create differences between theoretical predictions and experimental results, particularly at lower Knudsen numbers, by redistributing momentum within the gas flow.
		Adjustments like those provided by the Cohen-Reshotko method help correct predictions by accounting for these boundary layer impacts.

		Experiments, such as those by Cassanova and Stephenson, support the method of characteristics for analyzing free expansions and confirm it provides accurate predictions of gas density changes from continuous flow conditions to rarefied conditions.
		However, these studies also highlight important limitations due to condensation and boundary layer effects.
		Condensation, especially at pressures between 1–7 atm for nitrogen, significantly alters the gas's thermodynamic properties, limiting the method's accuracy.

		In summary, while the method of characteristics effectively predicts initial flow conditions, fully accurate modeling of supersonic expansions must also consider rotational freezing, boundary layer effects, and potential condensation.
		\cite{jousten_handbook_2016, robertson_investigation_1970, noauthor_zucrow_nodate}

\subsubsection*{Navier-Stokes Equations}
	A second option is to solve the Navier–Stokes equations numerically, using no-slip or first/second-order slip boundary conditions as appropriate.
	Near the nozzle exit—where the flow is still predominantly continuum—Navier–Stokes simulations can provide detailed velocity and pressure fields, including shock structures and boundary-layer effects.
	As the gas expands and its density decreases, however, continuum assumptions start to break down.
	While slip or transitional models can extend the validity range, ultimately they cannot fully describe the molecular-level interactions that dominate in highly rarefied regions.
	Consequently, beyond a certain distance from the nozzle exit or under conditions favorable to partial condensation, Navier–Stokes solutions must be supplemented by more advanced kinetic-based methods.
	\cite{anderson_fundamentals_2017, anderson2021modern}

\subsubsection*{Direct simulation Monte Carlo (DSMC)}
	To address the shortcomings of both the method of characteristics (under non-ideal conditions) and Navier–Stokes (when rarefaction becomes significant), the Direct Simulation Monte Carlo (DSMC) method is indispensable.
	DSMC performs a stochastic simulation of molecular motion and collisions, making it particularly well suited for:
	\begin{itemize}
		\item Transitional Regimes: Where the mean free path is comparable to or larger than the nozzle scale, DSMC accurately captures Knudsen-layer effects and velocity-slip phenomena.
		\item Condensation Phenomena: By including collision physics and intermolecular potentials, DSMC can handle the onset of condensation or two-phase flow more reliably than continuum or purely isentropic approaches. \cite{EMMONS1958}
		\item Free Expansion: As the jet spreads into the vacuum and transitions toward free-molecular flow, DSMC seamlessly handles extremely high Mach numbers and very low densities, maintaining physical fidelity across a wide range of Knudsen numbers.
	\end{itemize}
	Although more computationally expensive than continuum-based methods, DSMC provides the most robust depiction of outlet flows—particularly for condensable gases like nitrogen—and ensures high accuracy from near-nozzle regions all the way to the fully rarefied state.
	\cite{putignano2012supersonic, liu_study_2006}

\subsubsection*{Summary}
	Overall, while the method of characteristics remains a cornerstone of ideal nozzle design and near-exit expansions (including the prediction of Prandtl–Meyer fans and maximum turning angles), condensation and severe rarefaction invalidate its purely isentropic assumption further downstream.
	Navier–Stokes simulations can capture intermediate regimes near the exit, but eventually fail in the highly rarefied or partially condensed flow domain.
	For a comprehensive treatment of under-expanded plumes—particularly with nitrogen or other condensable working fluids—the DSMC technique offers the necessary level of detail, bridging the gap from continuum flow at the nozzle exit to free-molecular flow far downstream.
