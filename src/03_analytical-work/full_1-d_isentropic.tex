	By assuming the flow through the assembly will be fully isentropic and pseudo one-dimensional it is possible to calculate the state variables at every point knowing the stagnation conditions and the ratio between the cross-sectional area at the point of interest $A_i$ and the at throat of the assembly $A^*$.\\
	It must be noted that this is a very radical approximation since for the flow to be  considered pseudo one-dimensional the problem must be reduced to consist purely of variable area ducts.
	This clearly overlooks the fact that when entering and leaving the reactor the gas has to perform a right angle turn to follow the flow path.
	Another constraint on the duct geometry to achieve reasonable solutions assuming pseudo one-dimensional flow is that the duct must change its cross-sectional area gradually.
	\cite{anderson2021modern}
	This won't be the case inside the reactor, since there is no way of slicing the reactor chamber to achieve a gradual change in cross-section, especially around the inlet and outlet.
	\begin{figure}[H]
	    \centering
	    \includegraphics[width=0.9\textwidth]{src/03_analytical-work/fig_1d-flow-geometry}
	    \caption{A descriptive caption for the figure.}
	\end{figure}
	This geometry can now be characterized as a double throat, and therefore isentropic flow can result in one of two fundamental solutions.
	One predicting subsonic flow and one predicting supersonic flow after the first throat (2).
	\cite{SALAS1986193, EMMONS1958}

\subsubsection*{Calculations}

	The first step is to define the critical locations, where the flow fill be chocked.
	Since the outlet is expanding into vacuum, resulting in a pressure ratio tending towards zero, therefore the flow therefore must be chocked and can be recognized as a critical point.
	Maximum mass-flow occurs if the flow is chocked. Therefore, to keep up with the mass-flow of the outlet, the inlet must also be chocked: 
	$$
		A_{2,4},\;p_{2,4},\;\rho_{2,4},\;T_{2,4}\quad\xrightarrow{M=1}\quad A^*,\;p^*,\;\rho^*,\;T^*
	$$
	The second reference location corresponding to the stagnation or total conditions is at the entry of the inlet nozzle (1), which can be defined afterward to get quantitative solutions.
	$$
		A_1,\;p_1,\;\rho_1,\;T_1\quad\xrightarrow{M=0}\quad A_t,\;p_t,\;\rho_t,\;T_t
	$$
	Next step is to calculate the cross-sectional areas for every location:
	$$
		A_i = \pi \left(\frac{D_i}{2}\right)^2
			\quad \text{for} \quad
		i=\{1,2,4,5\}
			\quad \text{and} \quad
		A_3 = H_\text{reactor}\cdot D_\text{reactor}
	$$
	Followed by solving the equation for the ratio of cross-sectional area for M numerically which, should yield one subsonic and one supersonic solution.
	$$
		\frac{A}{A^*} = \frac{1}{M} \left[ \frac{2}{\gamma + 1} \left( 1 + \frac{\gamma - 1}{2} M^2 \right) \right]^{\frac{\gamma + 1}{2(\gamma - 1)}}
		\qquad \eqref{eq:area_ratio_mach}
	$$
	Afterward the ratios of state variables can be determined, which after defining the total conditions can be used to calculate the local variables for every point.
	\cite{hall_isentropic_nodate}

	\begin{alignat*}{2}
	    \frac{T}{T_t}   & = \left( 1 + \frac{\gamma - 1}{2} M^2 \right)^{-1}
	    & \qquad & \eqref{eq:total_relation_T} \\
	    \frac{p}{p_t}   & = \left( 1 + \frac{\gamma - 1}{2} M^2 \right)^{-\frac{\gamma}{\gamma - 1}}
	    & \qquad & \eqref{eq:total_relation_p} \\
	    \frac{\rho}{\rho_t} & = \left( 1 + \frac{\gamma - 1}{2} M^2 \right)^{-\frac{1}{\gamma - 1}}
	    & \qquad & \eqref{eq:total_relation_rho}
	\end{alignat*}
	Where $p,\; \rho,\; T$ are the local gas conditions, $p_t,\; \rho_t,\; T_t$ the total gas conditions, $\gamma$ the specific heat ratio and $M$ the local mach number.
	This leads to the following solutions:
	\begin{table}[H]
\centering
\renewcommand{\arraystretch}{1.4} % Moderate row spacing
\begin{tabular}{|c|c|c|c|c|c|}
\hline
$i$                  & $\frac{A_i}{A^*}$       & M         & $\frac{p_i}{p_t}$        & $\frac{\rho_i}{\rho_t}$   & $\frac{T_i}{T_t}$      \\ \hline
1                    & 4                       & 0         & 1                        & 1                         & 1                      \\ \hline
2                    & 1                       & 1         & 0.52                     & 0.64                      & 0.81                   \\ \hline
\multirow{2}{*}{3}   & \multirow{2}{*}{318.31} & $\sim$0   & $\sim$1                  & $\sim$1                   & $\sim$1                \\ \cline{3-6} 
                     &                         & 10.55     & $3.28 \cdot 10^{-5}$     & $8.90 \cdot 10^{-4}$      & $3.68 \cdot 10^{-2}$   \\ \hline
4                    & 1                       & 1         & 0.52                     & 0.64                      & 0.81                   \\ \hline
\multirow{2}{*}{5}   & \multirow{2}{*}{4}      & 0.15      & 0.983                    & 0.988                     & 0.994                  \\ \cline{3-6} 
                     &                         & 3.06      & 0.02629                  & 0.08415                   & 0.31245                \\ \hline
\end{tabular}
\caption{Isentropic flow properties for different conditions.}
\label{tab:isentropic_flow}
\end{table}

	Mass flow must be conserved along the flow and can be calculated using following equation, which derives from the general equation for mass-flow, the isentropic relations and the ideal gas law.
	\cite{benson_mass_nodate}
	$$
		\dot{m} = A \cdot P_t \cdot \sqrt{\frac{\gamma}{R T_t}} \cdot M \cdot \left(1 + \frac{\gamma - 1}{2} M^2\right)^{-\frac{\gamma + 1}{2(\gamma - 1)}}
	$$
	Where $A$ is the local cross-sectional area, $P_t$ the total pressure, $T_t$ the total temperature, $\gamma$ the specific heat ratio, $R$ the specific gas constant and $M$ the mach number.
	\note{
		Add mass flow for every point to show that mass flow is conserved.
		Also include calculation of dimensionless numbers!
	}
	\cite{Cantwell_AA210A}

\subsubsection*{Interpretation}
\note{
	Go more into the solutions. Especially into the non physicality of the high mach numbers inside. Also mention non isentropic processes. This is a idealized situation this will always lead to maximum massflow in comparison to other formulation.
}
