By assuming fully isentropic and pseudo one-dimensional flow throughout the assembly, it is possible to determine the state variables at any point, provided that the stagnation conditions and the area ratio between the cross-sectional area at the point of interest $A_i$ and the throat area $A^*$ are known.

However, it must be emphasized that this represents a considerable simplification, as pseudo one-dimensional flow strictly requires the geometry to be approximated by smoothly varying ducts.
This assumption neglects the abrupt directional changes that occur as the gas enters and exits the reactor.
Moreover, pseudo one-dimensional formulations typically demand gradual changes in cross-sectional area, which clearly is not the case at the sharp transitions near the reactor inlet and outlet.
\cite{anderson2021modern}

\begin{figure}[H]
    \centering
    \includegraphics[width=0.9\textwidth]{src/03_analytical-work/fig_1d-flow-geometry}
    \caption[Idealized geometry of the reactor assembly modeled as a one-dimensional variable-area duct.]{
        \textbf{Idealized geometry of the reactor assembly modeled as a one-dimensional variable-area duct.}
        The yellow region represents the domain where quasi one-dimensional isentropic flow is assumed (as described in section~\ref{sec:geometry}).
        Grey indicates stagnant gas, the green arrow indicates subsonic flow at the inlet, and the red arrow indicates supersonic flow at the outlet.
        Flow within the central reactor region is most likely slow-moving.
    }
\end{figure}

The simplified geometry can now be recognized as a double-throat configuration, and one-dimensional isentropic flow theory provides two fundamental solutions for this setup.
Depending on how steady-state conditions are established, the flow either remains subsonic (with velocity decreasing upon entry into the reactor), or it reaches sonic conditions at the inlet throat (2) and continues to accelerate downstream.
Other non-trivial solutions will not be considered.
\cite{SALAS1986193, EMMONS1958}

\subsubsection*{Calculations}
	The first step is identifying critical locations where the flow becomes choked.
	Given the outlet expansion into vacuum, the pressure ratio approaches zero, ensuring choked flow at this location.
	Because maximum mass-flow through the assembly occurs under choked conditions, the inlet must also be choked to match the mass-flow rate:
	$$
		A_{2,4},\;p_{2,4},\;\rho_{2,4},\;T_{2,4}
			\quad \xrightarrow {M = 1} \quad
		A^*,\;p^*,\;\rho^*,\;T^*
	$$
	The second critical location corresponds to the stagnation (total) conditions at the inlet nozzle entry (1), defined by zero velocity:
	$$
		A_1,\;p_1,\;\rho_1,\;T_1
			\quad \xrightarrow{M=0} \quad
		A_0,\;p_0,\;\rho_0,\;T_0
	$$
	Next, cross-sectional areas $A_i$ at all points $i$ defined in chapter~\ref{sec:geometry} are used to calculate their respective area ratios relative to the throat area $A_{2,4}$.
	These area ratios allow for solving equation~\eqref{eq:area_ratio_mach}, which numerically yields both subsonic and supersonic solutions depending on initial conditions, as detailed in appendix~\ref{apx:mach-py}.

	Subsequently, local state variables relative to stagnation conditions can be computed using equations~\eqref{eq:total_relation_T}–\eqref{eq:total_relation_rho}.
	Once stagnation conditions are specified, the local state variables at each location in the assembly are directly determined.

	\begin{table}[H]
\centering
\renewcommand{\arraystretch}{1.4} % Moderate row spacing
\begin{tabular}{|c|c|c|c|c|c|}
\hline
$i$                  & $\frac{A_i}{A^*}$       & M         & $\frac{p_i}{p_t}$        & $\frac{\rho_i}{\rho_t}$   & $\frac{T_i}{T_t}$      \\ \hline
1                    & 4                       & 0         & 1                        & 1                         & 1                      \\ \hline
2                    & 1                       & 1         & 0.52                     & 0.64                      & 0.81                   \\ \hline
\multirow{2}{*}{3}   & \multirow{2}{*}{318.31} & $\sim$0   & $\sim$1                  & $\sim$1                   & $\sim$1                \\ \cline{3-6} 
                     &                         & 10.55     & $3.28 \cdot 10^{-5}$     & $8.90 \cdot 10^{-4}$      & $3.68 \cdot 10^{-2}$   \\ \hline
4                    & 1                       & 1         & 0.52                     & 0.64                      & 0.81                   \\ \hline
\multirow{2}{*}{5}   & \multirow{2}{*}{4}      & 0.15      & 0.983                    & 0.988                     & 0.994                  \\ \cline{3-6} 
                     &                         & 3.06      & 0.02629                  & 0.08415                   & 0.31245                \\ \hline
\end{tabular}
\caption{Isentropic flow properties for different conditions.}
\label{tab:isentropic_flow}
\end{table}


	With approximate state variables known throughout the assembly, assumptions regarding Reynolds and Knudsen numbers made in chapter~\ref{sec:expected-flow-regimes} can now be tested.
	Given that mass flow is conserved, it can be calculated from equation~\eqref{eq:1-d-massflow} using stagnation conditions and any given pair of cross-sectional area and Mach number at a location $i$:
	\begin{align*}
		p_{0_{ref}} = 1.5\;\text{bar}
			&\qquad T_{0_{min}} = 300\;\text{K}
				\qquad \text{yields} \qquad
			\dot{m}_1 \approx 1.1 \cdot 10^{-7} \; \frac{\text{kg}}{\text{s}}\\
		p_{0_{ref}} = 1.5\;\text{bar}
			&\qquad T_{0_{max}} = 600\;\text{K}
				\qquad \text{yields} \qquad
			\dot{m}_2 \approx 7.8 \cdot 10^{-8} \; \frac{\text{kg}}{\text{s}}
	\end{align*}

	The following table summarizes the calculated Reynolds and Knudsen numbers at various points within the assembly for two stagnation temperatures ($T_{0_{min}} = 300 \text{K}$ and $T_{0_{max}} = 500 \text{K}$) and a stagnation pressure of $p_{0_{ref}} = 1.5\,\text{bar}$.

	\begin{table}[H]
    \centering
    \begin{tabular}{|c|c|c|c|c|c|c|c|c|}
    \hline
    \multirow{2}{*}{} & \multirow{2}{*}{$T_t$}
    & \multirow{2}{*}{1} & \multirow{2}{*}{2} & \multirow{2}{*}{\color{greenColor} 3 (sub)}
    & \multirow{2}{*}{\color{redColor} 3 (sup)} & \multirow{2}{*}{4}
    & \multirow{2}{*}{\color{redColor} 5 (sub)} & \multirow{2}{*}{\color{greenColor}5 (sup)} \\
    & & & & & & & & \\ \hline

    \multirow{2}{*}{Re}
      & 300 K & - & 0.0001 & 0.4 & $\sim 0$ & 0.0001 & 0.005 & 0.0002 \\
      & 500 K & - & 0.001  & 0.8 & 0.0001   & 0.001  & 0.01  & 0.0005 \\ \hline

    \multirow{2}{*}{Kn}
      & 300 K & 0.001 & 0.002 & 0.001 & \cellcolor[HTML]{FFADA8}0.2
              & 0.002 & 0.001 & 0.009 \\
      & 500 K & 0.002 & 0.003 & 0.002 & 0.02
              & 0.003 & 0.002 & 0.02 \\ \hline
    \end{tabular}
    \caption{Knudsen and Reynolds numbers for $p_0 = 1.5\,\mathrm{bar}$ 
        and $T_0 = 300, 500\,\mathrm{K}$ using values from 
        table~\ref{tab:isentropic_flow}. The cell highlighted {\color{redColor}red} indicates 
        a transitional regime, whereas the titles colored {\color{redColor} red} and {\color{greenColor} green} correspond to the less and more likely solutions from table \ref{tab:isentropic_flow} respectively}
    \label{tab:test-knudsen-reynolds-isentropic}
\end{table}


	\newpage

\subsubsection*{Interpretation}

	The equation relating the area ratio to Mach number typically yields two mathematically consistent solutions at every location (excluding reference points with fixed conditions).
	These correspond to subsonic and supersonic flows, respectively, and their applicability depends on physical constraints.

	For the outlet nozzle, the correct physical solution is supersonic, since the flow exhausts into a vacuum.
	This holds even in the absence of a clearly defined converging section downstream of the reactor.
	As discussed in Section~\ref{sec:expected-flow-regimes}, the flow itself may effectively create a converging geometry as it passes over a sudden geometric contraction, enabling acceleration to sonic conditions and beyond.

	While the above analysis establishes a baseline using one-dimensional isentropic flow theory, real micro-channels often exhibit additional phenomena such as slip effects at the boundaries and significant viscous losses.
	These effects are particularly important at small scales and can substantially alter the actual flow behavior.
	The next section therefore investigates these microchannel-specific effects and their implications for the validity of the assumptions made thus far.
