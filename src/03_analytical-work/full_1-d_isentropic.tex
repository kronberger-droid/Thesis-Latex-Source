	By assuming the flow through the assembly will be fully isentropic and pseudo one-dimensional it is possible to calculate the state variables at every point knowing the stagnation conditions and the ratio between the cross-sectional area at the point of interest $A_i$ and the at throat of the assembly $A^*$.\\
	It must be noted that this is a very radical approximation since for the flow to be  considered pseudo one-dimensional the problem must be reduced to consist purely of variable area ducts.
	This clearly overlooks the fact that when entering and leaving the reactor the gas has to perform a right angle turn to follow the flow path.
	Another constraint on the duct geometry to achieve reasonable solutions assuming pseudo one-dimensional flow is that the duct must change its cross-sectional area gradually.
	\cite{anderson2021modern}
	This won't be the case inside the reactor, since there is no way of slicing the reactor chamber to achieve a gradual change in cross-section, especially around the inlet and outlet.
	\begin{figure}[H]
	    \centering
	    \includegraphics[width=0.9\textwidth]{src/03_analytical-work/fig_1d-flow-geometry}
	    \caption{A descriptive caption for the figure.}
	\end{figure}
	This geometry can now be characterized as a double throat, and therefore one-dimensional isentropic flow will yield one of in two fundamental solutions.
	Depending on how the steady state in the system is reached, flow velocity of the stream will either stay subsonic and therefore decrease when entering the reactor or become sonic at the throat (2) and continue to accelerate.
	Other non-trivial solutions won't be discussed.
	\cite{SALAS1986193, EMMONS1958}

\subsubsection*{Calculations}

	The first step is to define the critical locations, where the flow fill be chocked.
	Since the outlet is expanding into vacuum, resulting in a pressure ratio tending towards zero, therefore the flow therefore must be chocked and can be recognized as a critical point.
	Maximum mass-flow occurs if the flow is chocked. Therefore, to keep up with the mass-flow of the outlet, the inlet must also be chocked: 
	$$
		A_{2,4},\;p_{2,4},\;\rho_{2,4},\;T_{2,4}\quad\xrightarrow{M=1}\quad A^*,\;p^*,\;\rho^*,\;T^*
	$$
	The second reference location corresponding to the stagnation or total conditions is at the entry of the inlet nozzle (1), which can be defined afterward to get quantitative solutions.
	$$
		A_1,\;p_1,\;\rho_1,\;T_1\quad\xrightarrow{M=0}\quad A_0,\;p_0,\;\rho_0,\;T_t
	$$
	Next step is to take the cross-sectional areas at every point defined in chapter \ref{sec:geometry}, and calculate their ratios to the throat areas.
	To use them for solving the following equation relating them to M, numerically yielding one subsonic and one supersonic solution depending on their initial conditions.
	$$
		\frac{A_i}{A^*} = \frac{1}{M_i} \left[ \frac{2}{\gamma + 1} \left( 1 + \frac{\gamma - 1}{2} M^2 \right) \right]^{\frac{\gamma + 1}{2(\gamma - 1)}}
		\qquad \eqref{eq:area_ratio_mach}
	$$
	Afterward the ratio of the local state variables to the total conditions can be determined, which can be used after, defining total conditions for the system, to calculate the local variables for every point.
	\cite{hall_isentropic_nodate}

	\begin{alignat*}{2}
	    \frac{T_i}{T_0}   & = \left( 1 + \frac{\gamma - 1}{2} M_i^2 \right)^{-1}
	    & \qquad & \eqref{eq:total_relation_T} \\
	    \frac{p_i}{p_0}   & = \left( 1 + \frac{\gamma - 1}{2} M_i^2 \right)^{-\frac{\gamma}{\gamma - 1}}
	    & \qquad & \eqref{eq:total_relation_p} \\
	    \frac{\rho_i}{\rho_0} & = \left( 1 + \frac{\gamma - 1}{2} M_i^2 \right)^{-\frac{1}{\gamma - 1}}
	    & \qquad & \eqref{eq:total_relation_rho}
	\end{alignat*}
	Where $p,\; \rho,\; T$ are the local gas conditions, $p_0,\; \rho_0,\; T_0$ the total gas conditions, $\gamma$ the specific heat ratio and $M_i$ the local mach number.
	This leads to the following solutions:
	\begin{table}[H]
\centering
\renewcommand{\arraystretch}{1.4} % Moderate row spacing
\begin{tabular}{|c|c|c|c|c|c|}
\hline
$i$                  & $\frac{A_i}{A^*}$       & M         & $\frac{p_i}{p_t}$        & $\frac{\rho_i}{\rho_t}$   & $\frac{T_i}{T_t}$      \\ \hline
1                    & 4                       & 0         & 1                        & 1                         & 1                      \\ \hline
2                    & 1                       & 1         & 0.52                     & 0.64                      & 0.81                   \\ \hline
\multirow{2}{*}{3}   & \multirow{2}{*}{318.31} & $\sim$0   & $\sim$1                  & $\sim$1                   & $\sim$1                \\ \cline{3-6} 
                     &                         & 10.55     & $3.28 \cdot 10^{-5}$     & $8.90 \cdot 10^{-4}$      & $3.68 \cdot 10^{-2}$   \\ \hline
4                    & 1                       & 1         & 0.52                     & 0.64                      & 0.81                   \\ \hline
\multirow{2}{*}{5}   & \multirow{2}{*}{4}      & 0.15      & 0.983                    & 0.988                     & 0.994                  \\ \cline{3-6} 
                     &                         & 3.06      & 0.02629                  & 0.08415                   & 0.31245                \\ \hline
\end{tabular}
\caption{Isentropic flow properties for different conditions.}
\label{tab:isentropic_flow}
\end{table}

	Mass flow is conserved along the flow and can be calculated using following equation, which derives from the equation for mass-flow in a steady one-dimensional flow \eqref{eq:1-d-massflow}, the isentropic relations [\eqref{eq:total_relation_T} - \eqref{eq:total_relation_p}] and the ideal gas law \eqref{eq:ideal_gas}.
	\cite{benson_mass_nodate}
	$$
		\dot{m} = A \cdot P_0 \cdot \sqrt{\frac{\gamma}{R T_0}} \cdot M \cdot \left(1 + \frac{\gamma - 1}{2} M^2\right)^{-\frac{\gamma + 1}{2(\gamma - 1)}}
	$$
	Where $A$ is the local cross-sectional area, $P_0$ the total pressure, $T_0$ the total temperature, $\gamma$ the specific heat ratio, $R$ the specific gas constant and $M$ the mach number.
	\cite{Cantwell_AA210A}

	Now knowing the approximate way state variables change inside the system, the assumption taken in chapter \ref{sec:expected-flow-regimes} regarding the Reynolds and Knudsen number can be tested for the following values:
	\begin{align*}
		p_{0,\;1} = 1.5\;\text{bar} &\qquad T_{0,\;1} = 300\;\text{K}\\
		p_{0,\;2} = 1.5\;\text{bar} &\qquad T_{0,\;2} = 500\;\text{K}
	\end{align*}
	Additionally, the lower limit for the reservoir pressure $p_{0,\;min}$ where some part of the system transitions to the molecular regime, will be determined.
	Again for temperatures $T_{0,\;1},\;T_{0,\;2}$
	

\subsubsection*{Interpretation}
\note{
	Go more into the solutions. Especially into the non physicality of the high mach numbers inside. Also mention non-isentropic processes. This is a idealized situation this will always lead to maximum mass flow in comparison to other formulation.
}
