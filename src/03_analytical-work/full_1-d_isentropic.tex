	By assuming the flow through the assembly will be fully isentropic and pseudo one-dimensional it is possible to calculate the state variables at every point knowing the stagnation conditions and the ratio between the cross-sectional area at the point of interest $A_i$ and the at throat of the assembly $A^*$.\\
	It must be noted that this is a very radical approximation since for the flow to be  considered pseudo one-dimensional the problem must be reduced to consist purely of variable area ducts.
	This clearly overlooks the fact that when entering and leaving the reactor the gas has to perform a right angle turn to follow the flow path.
	Another constraint on the duct geometry to achieve reasonable solutions assuming pseudo one-dimensional flow is that the duct must change its cross-sectional area gradually.
	\cite{anderson2021modern}
	This won't be the case inside the reactor, since there is no way of slicing the reactor chamber to achieve a gradual change in cross-section, especially around the inlet and outlet.
	\begin{figure}[H]
	    \centering
	    \includegraphics[width=0.9\textwidth]{src/03_analytical-work/fig_1d-flow-geometry}
	    \caption{A descriptive caption for the figure.}
	\end{figure}
	This geometry can now be characterized as a double throat, and therefore one-dimensional isentropic flow will yield one of in two fundamental solutions.
	Depending on how the steady state in the system is reached, flow velocity of the stream will either stay subsonic and therefore decrease when entering the reactor or become sonic at the throat (2) and continue to accelerate.
	Other non-trivial solutions won't be discussed.
	\cite{SALAS1986193, EMMONS1958}

\subsubsection*{Calculations}

	The first step is to define the critical locations, where the flow fill be chocked.
	Since the outlet is expanding into vacuum, resulting in a pressure ratio tending towards zero, therefore the flow therefore must be chocked and can be recognized as a critical point.
	Maximum mass-flow occurs if the flow is chocked. Therefore, to keep up with the mass-flow of the outlet, the inlet must also be chocked: 
	$$
		A_{2,4},\;p_{2,4},\;\rho_{2,4},\;T_{2,4}\quad\xrightarrow{M=1}\quad A^*,\;p^*,\;\rho^*,\;T^*
	$$
	The second reference location corresponding to the stagnation or total conditions is at the entry of the inlet nozzle (1), which can be defined afterward to get quantitative solutions.
	$$
		A_1,\;p_1,\;\rho_1,\;T_1\quad\xrightarrow{M=0}\quad A_0,\;p_0,\;\rho_0,\;T_t
	$$
	Next step is to take the cross-sectional areas at every point defined in chapter \ref{sec:geometry}, and calculate their ratios to the throat areas.
	To use them for solving the following equation relating them to M, numerically yielding one subsonic and one supersonic solution depending on their initial conditions.
	$$
		\frac{A_i}{A^*} = \frac{1}{M_i} \left[ \frac{2}{\gamma + 1} \left( 1 + \frac{\gamma - 1}{2} M^2 \right) \right]^{\frac{\gamma + 1}{2(\gamma - 1)}}
		\qquad \eqref{eq:area_ratio_mach}
	$$
	Afterward the ratio of the local state variables to the total conditions can be determined using equations \eqref{eq:total_relation_T} - \eqref{eq:total_relation_rho}, which can be used after, defining the total conditions of the system, to calculate the local variables for every point.
	\cite{hall_isentropic_nodate}
	\begin{alignat*}{2}
	    \frac{T}{T_0}   & = \left( 1 + \frac{\gamma - 1}{2} Ma_i^2 \right)^{-1}\qquad &\eqref{eq:total_relation_T}\\
	    \frac{p}{p_0}   & = \left( 1 + \frac{\gamma - 1}{2} Ma_i^2 \right)^{-\frac{\gamma}{\gamma - 1}}&\eqref{eq:total_relation_T}\\
	    \frac{\rho}{\rho_0} & = \left( 1 + \frac{\gamma - 1}{2} Ma_i^2 \right)^{-\frac{1}{\gamma - 1}}&\eqref{eq:total_relation_T}
	\end{alignat*}
	Where $p,\; \rho,\; T$ are the local gas conditions, $p_0,\; \rho_0,\; T_0$ the total gas conditions, $\gamma$ the specific heat ratio and $M_i$ the local mach number.
	This leads to the following solutions:
	\begin{table}[H]
\centering
\renewcommand{\arraystretch}{1.4} % Moderate row spacing
\begin{tabular}{|c|c|c|c|c|c|}
\hline
$i$                  & $\frac{A_i}{A^*}$       & M         & $\frac{p_i}{p_t}$        & $\frac{\rho_i}{\rho_t}$   & $\frac{T_i}{T_t}$      \\ \hline
1                    & 4                       & 0         & 1                        & 1                         & 1                      \\ \hline
2                    & 1                       & 1         & 0.52                     & 0.64                      & 0.81                   \\ \hline
\multirow{2}{*}{3}   & \multirow{2}{*}{318.31} & $\sim$0   & $\sim$1                  & $\sim$1                   & $\sim$1                \\ \cline{3-6} 
                     &                         & 10.55     & $3.28 \cdot 10^{-5}$     & $8.90 \cdot 10^{-4}$      & $3.68 \cdot 10^{-2}$   \\ \hline
4                    & 1                       & 1         & 0.52                     & 0.64                      & 0.81                   \\ \hline
\multirow{2}{*}{5}   & \multirow{2}{*}{4}      & 0.15      & 0.983                    & 0.988                     & 0.994                  \\ \cline{3-6} 
                     &                         & 3.06      & 0.02629                  & 0.08415                   & 0.31245                \\ \hline
\end{tabular}
\caption{Isentropic flow properties for different conditions.}
\label{tab:isentropic_flow}
\end{table}

	Mass flow is conserved along the flow and can be calculated using following equation, which derives from the equation for mass-flow in a steady one-dimensional flow \eqref{eq:1-d-massflow}, the isentropic relations [\eqref{eq:total_relation_T} - \eqref{eq:total_relation_p}] and the ideal gas law \eqref{eq:ideal_gas}.
	\cite{benson_mass_nodate}
	$$
		\dot{m} = A \cdot P_0 \cdot \sqrt{\frac{\gamma}{R T_0}} \cdot M \cdot \left(1 + \frac{\gamma - 1}{2} M^2\right)^{-\frac{\gamma + 1}{2(\gamma - 1)}}
		\qquad \eqref{}
	$$
	Where $A$ is the local cross-sectional area, $P_0$ the total pressure, $T_0$ the total temperature, $\gamma$ the specific heat ratio, $R$ the specific gas constant and $M$ the mach number.
	\cite{Cantwell_AA210A}

	Now knowing the approximate way state variables change inside the system, the assumption taken in chapter \ref{sec:expected-flow-regimes} regarding the Reynolds and Knudsen number can be tested for the following values:
	\begin{align*}
		p_{0,\;1} = 1.5\;\text{bar} &\qquad T_{0,\;1} = 300\;\text{K}\\
		p_{0,\;2} = 1.5\;\text{bar} &\qquad T_{0,\;2} = 500\;\text{K}
	\end{align*}
	Additionally, the lower limit for the reservoir pressure $p_{0,\;min}$ where some part of the system transitions to the molecular regime, will be determined.
	Again for temperatures $T_{0,\;1},\;T_{0,\;2}$
	\begin{table}[H]
		\centering
		\begin{tabular}{|c|c|c|c|c|c|c|c|c|}
		\hline
		 & $T_t$ & 1 & 2 & 3 (sub) & 3 (sup) & 4 & 5 (sub) & 5 (sup) \\ \hline
		 & 300 K & 0.001 & 0.002 & 0.001 & \cellcolor[HTML]{FFADA8}0.2 & 0.002 & 0.001 & 0.009 \\
		\multirow{-2}{*}{Kn} & 500 K & 0.002 & 0.003 & 0.002 & 0.02 & 0.003 & 0.002 & 0.02 \\ \hline
	     & 300 K & - & 0.0001 & 0.4 & $\sim$0 & 0.0001 & 0.005 & 0.0002  \\
		\multirow{-2}{*}{Re} & 500 K & - & 0.001 & 0.8 & 0.0001 & 0.001 & 0.01 & 0.0005 \\ \hline 
		\end{tabular}
		\caption{Knudsen and Reynolds number for total pressure $p_0 = 1.5 \text{bar}$ and total temperatures of $T_0 = 300, 500 \text{K}$ using values from table \ref{tab:isentropic_flow}. Where red color signifies transition possible transitional regime.}
		\label{tab:knudsen-reynolds-isentropic}
	\end{table}
	\noindent It now becomes clear that the assumptions made in chapter \ref{sec:expected-flow-regimes} are accurate:
	\begin{itemize}
		\item Slip on the boundaries is to expect.
		\item The Mach number has major influence on the value of the Knudsen number, thus high Mach numbers point to possible transition to molecular regime.
		\item The outlet always corresponds to the highest Knudsen number with exception to the hypersonic solution found in the middle of the reactor. Which will be discussed more in the following paragraph.
		\item Reynolds numbers are very low, therefore no turbulences are expected.
	\end{itemize}
\subsubsection*{Interpretation}
	The equation relating area ratio and Mach number has in general two solutions.
	This leads to the two solutions for every point which is not defined as a (total or critical) reference point will have mathematically true solutions
	One corresponding to subsonic flow and the other one to supersonic flow, which of those are applicable in the situation is to decide.

	For the outlet it is easier to discern since when exausting into a vacuum the flow will be forced supersonic.
	This is expected even tho there is actually no specific converging section when leaving the reactor, especially if the reactor is seen as a reservoir by itself, like it will in chapter \ref{sec:disconnected-reservoirs}.
	In short, the flow itself when flowing over the edge of a sudden contraction in the flow geometry, can itself form a kind of converging section and still accelerate to sonic conditions.
	This will be discussed in the next chapter in more detail.

	\note{
		Mention non-isentropic behavior non viability of the hypersonic solution!
	}	 
