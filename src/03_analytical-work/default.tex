\section{Analytical work}
	Now that the fundamental principles used are set, this chapter focuses on answering as many questions from chapter \ref{sec:scope-objectives}.
	First the geometry is defined in a more rigorous way, to be able to define the characteristic lengths for every important point inside the assembly.
	While also defining the conditions inside the reservoir and the vacuum and how they influence the flow.
	Afterward, the expected flow regimes will be assumed, to confirm that the isentropic continuum flow regime is applicable.
	These assumptions will be confirmed by the following analytical calculations.
	Where first the assembly will be approximated by a variable area duct, without any leakage in the system, where one-dimensional isentropic flow formulations can be applied, and the assumptions made in the preceding chapter can be tested.
	Since this approximation come with a cost, mainly by ignoring the geometry and possible non-isentropic behavior in the reactor.
	Therefore, a short section going into behaviors in micro-channels and the problem of slip on the boundaries follows, which paves the way for a more elegant formulation, where the reactor is seen as its own reservoir with given stagnation conditions.
	Since a change in stagnation conditions corresponds to non-isentropic flow behavior, is trying to include those while still treating the parts isentropically in isolation.
	This makes it possible to include mass-flow of the leak in the reactor, whose influence on the Mach number of the inlet throat will be explored.
	Ending with a short argument on why using the given formulations are not applicable to solve the velocity distribution at the outlet and therefore possible numerical solutions are presented.

\subsection{Geometry and flow characteristics of the components}\label{sec:geometry}
		The geometry can be explained in three simple sections: gas from a reservoir flows over the inlet into the micro-reactor where it leaves through the outlet into a vacuum. 
	This is a stark simplification, but for a great part of this thesis, this is how we will imagine our flow path.
	This is because the only thing we left out is any kind of leak in the system.
	Those leaks will be the most influential around the reactor, since this is the only part that is not held at a constant pressure by any external part of the system.\\

\subsubsection*{Reservoir}
	It is kept at constant pressure $P_0$ and constant temperature $T_0$, out of which the density $\rho$ can be calculated.
	These values will represent the stagnation or total conditions of the system, where the flow velocity is zero. The parts of the flow where stagnation conditions are assumed will be represented using a gray hue.
	For simplicity, it contains only one gas which is defined by its specific heat ratio $\gamma$ and by its molar mass $M_m$.
	The Formulations used yield the changes in pressure, temperature and density as ratios between the local conditions and the stagnation conditions
	Therefore, there is no need to define stagnation conditions explicitly beforehand, since local conditions can be calculated for by multiplying chosen stagnation conditions with the calculated ratios.
	Therefore, only the parameters of the gas have to be defined up front, which will be defined as:
	$$
		\gamma = 1.47\;, \qquad M_m = 28.013\; \frac{\text{g}}{\text{mol}}
	$$
	representing nitrogen gas ($\text{N}_2$), since it is comparable to the gases planned to use in experiments.
	\newpage

\subsubsection*{Inlet/Outlet Nozzles}
	\paragraph{Inlet}
		The duct connecting the inlet reservoir with the reactor will be a slightly converging duct, due to production constraints.
		Therefore, it will act like a Nozzle, accelerating the gas until it expands into the reactor.
	
	\paragraph{Outlet}
		With the same geometry as the inlet, but the gas flowing in opposite directions, one would suspect the outlet to act as a subsonic nozzle, which could logically be the case, since without a converging section in front it would be impossible to reach sonic velocities and therefore will choke the flow and keep them at subsonic velocities.
		However, it is actually possible for the flow to create a converging section by itself, which will force the flow to be sonic at the beginning of the outlet and will further accelerate into the supersonic regimes, creating a supersonic nozzle.
		\cite{jousten_handbook_2016}
	\begin{figure}[H]
	    \centering
	    \includegraphics[width=\textwidth]{src/03_analytical-work/fig_nozzle-geometries.pdf}
	    \caption{
			Schematic of the inlet and outlet nozzle assembly.
			Yellow-shaded regions denote domains where one-dimensional flow is assumed.
			Flow is indicated by colored arrows: green arrows represent subsonic flow and red arrows indicate supersonic flow.
			In the downstream expansion region, the flow is modeled as two-dimensionally rotationally symmetric (green-shaded).
			This schematic is not drawn to scale
		}
	    \label{fig:geometry-nozzles}
	\end{figure}

	\paragraph{Nozzle flow - pseudo 1D}
		In these sections of the flow a simplification of the two-dimensional flow through a duct will be used called "quasi one-dimensional" flow.
		This is possible by reducing the velocity distribution present at any point along the length of a duct to its mean velocity.
		Therefore, reducing the velocity from a distribution $V(r)$ at every point in the duct to a scalar value $V$ at every point of the duct.
		This is a general simplification which can be made when using continuum flow analysis inside of ducts with reasonable small change in cross-sectional area and is only bound by the applicability of continuum flow as described in chapter \ref{sec:flow-dimension-foundations}.
		One-dimensional flow will be represented using a yellow hue, as seen in the volume of the nozzles in figure \ref{fig:geometry-nozzles}.
		\cite{anderson2021modern}
		\newpage
\subsubsection*{Micro-Reactor Volume}
	The reactor resembles a very small but broad cylinder shape which is opened at the bottom.
	The sample will be pressed into the opening, which will lead to some leakage out of the system.
	\begin{figure}[H]
	    \centering
	    \includegraphics[width=\textwidth]{src/03_analytical-work/fig_reactor-geometry.pdf}
	    \caption{
			Schematic of the reactor geometry.
			The reactor volume is shown in blue to represent the region where three-dimensional flow occurs.
			The sample surface is indicated in red, and the leak is marked by a blue outline along the reactor perimeter.
			Flow characteristics are depicted by colored arrows: yellow arrows for slow-moving flow, green arrows for subsonic flow, and red arrows for supersonic flow.
			This diagram is not drawn to scale.
		}
	    \label{fig:geometry-reactor}
	\end{figure}

	\paragraph*{Flow through the reactor - 3D}
		After the gas flow leaves the inlet nozzle and streams into the reactor chamber, the geometry of the assembly has no symmetry to reduce the dimension of the velocity field.
		Additionally, there is rapid expansion of the gas since the constraints given by the walls of the nozzle are now gone and the gas flows over a sharp corner.
		Therefore, the flow field has to be dependent on all spacial dimensions and essentially means the only way to get an accurate representation of the flow field the Navier-Stokes equations have to be solved.
		This won't be possible using only analytical tools, therefore in the following sections two simplifications of the flow through the reactor will be used.
		At first, assuming pseudo one-dimensional flow throughout the reactor, which is definitely wrong, since the flux area is neither expanding very slowly, nor the flow will be isentropic, due to the expansion into free space.
		To address this issue and still be able to use analytical tools to solve the system, in chapter \ref{sec:disconnected-reservoirs} the reactor will be assumed to be a reservoir.
		Three-dimensional flow will be represented using a blue hue as seen in the middle of the reactor in figure \ref{fig:geometry-reactor}.

\subsubsection*{Vacuum}
	After leaving the outlet, the gas will first expand into a small cylindrical section after which it will expand freely into the vacuum.
	The exact pressure left in the vacuum chamber does not influence the gas flow inside the assembly since the large pressure ratio between reactor and vacuum will force the flow to be chocked, and thus the back pressure looses its influence.
	Due to the sharp change in pressure and the high velocity of the gas particles the gas will transition to the molecular regime after leaving the outlet nozzle.

	\paragraph{Free jet into vacuum - 2D}
		Similar to the flow into the reactor, the flow out of the system into the vacuum chamber won't be reducible to pseudo one-dimensional flow anymore.
		But this time the fact that the outlet nozzle is radially symmetric and there is no further geometry constraining the flow after it leaves the outlet.
		The flow can be at least assumed to inherit the radial symmetry from the outlet nozzle, thus reducing the spacial parameters of the flow field to the distance from the nozzle and the distance from the axis of symmetry of the outlet nozzle $r$.
		Furthermore, due to the gas expanding into vacuum and therefore the pressure dropping rapidly the gas will go through the process of rarefication, thus after some distance from the nozzle even continuum formulations break down.
		This will be discussed in more detail in chapter \ref{sec:outlet_plume} where possible ways to formulate solutions using numerical techniques.
		\cite{anderson2021modern}


\subsection{Expected flow regimes}\label{sec:expected-flow-regimes}
	\subsubsection*{Continuum Regime}
	The theoretical framework developed in this thesis relies heavily on the continuum flow assumption.
	For every location along the flow path, except in the vacuum region, it is assumed that the state variables ($T$, $p$, $\rho$) can be approximated using the continuum model introduced in Chapter~\ref{sec:one-dim-isentropic}.
	These state variables will then be used to evaluate the Knudsen number.

	A sensible question at this point is: Where in the flow is the Knudsen number expected to be highest?
	Identifying this location helps to limit the analysis to specific regions when assessing the validity of the continuum assumption and simplifies the evaluation.

	To answer this, we start from the definition of the Knudsen number.
	\cite{halwidl_development_2016, anderson2021modern}
	$$
		Kn(p,T) = \frac{\lambda}{L_c} = \frac{\mu(T)}{pL_c}\sqrt{\frac{\pi R T}{2}}
	$$
	Here, $\lambda$ is the mean free path, $L_c$ is the characteristic length, $R$ is the specific gas constant, $T$ is the fluid temperature, $p$ is the fluid pressure, $M_m$ is the molar mass, and $\mu$ is the dynamic viscosity.

	The dynamic viscosity is evaluated using Sutherland's formula.
	\cite{Hirschfelder1954MolecularTO}
	\begin{equation}
		\mu(T) = \mu_0 \left(\frac{T}{T_0}\right)^{3/2} \frac{T_0 + S_\mu}{T + S_\mu}
		\label{eq:sutherland}
	\end{equation}
	Here, $\mu_0$ is the reference viscosity at the reference temperature $T_0$, and $S_\mu$ is the Sutherland constant, both dependent on the gas species.
	For nitrogen, the following values apply:
	\cite{kim2004numericalanalysisflowcharacteristics}
	$$
		S_\mu = 111\;\text{K}, \quad T_0 = 300.55\;\text{K}, \quad \mu_0 = 17.81\; \text{sPa}
	$$
	The following plot shows the dynamic viscosity over temperature using Sutherland's formula, along with two linear approximations over the temperature range of 200–600~K.
	One of the approximations is constrained to pass through the origin.
	While this shifts the transition point to molecular flow slightly, it does not significantly affect the general behavior of the Knudsen number.
	For simplicity, the best-fit line with zero intercept will be used in the following argument.
	\newpage
	\begin{figure}[H]
\centering
    \begin{tikzpicture}[scale=0.85]
        \begin{axis}[
            width=\textwidth,
            height=0.6\textwidth,
            xlabel={Temperature $T$ (K)},
            ylabel={Dynamic Viscosity $\mu$ (N·s/m$^2$)},
            xmin=200, xmax=600,
            ymin=0, ymax=4e-5,
            domain=200:600,
            grid=both,
            legend pos=north west,
        ]
            % Sutherland's formula for Air
            \addplot [
                thick,
                blueColor,
                samples=200
            ]
            {(1.716e-5) * (x/273)^(3/2) * (273+111)/(x+111)};
            \addlegendentry{Nitrogen ($\text{N}^2$) using Sutherlands formula}

            % Linear interpolation of the above function (dashed red)
            \addplot [
                thick,
                dashed,
                redColor
            ]
            {(4.18e-8)*x + 5.75e-6};
            \addlegendentry{\shortstack{Best fit: $\mu(T)\approx 4.18\cdot 10^{-8}\; T + 5.75\cdot 10^{-6}$}}

            % Linear interpolation of the above function (dashed red)
            \addplot [
                thick,
                dashed,
                greenColor
            ]
            {(5.51e-8)*x};
            \addlegendentry{\shortstack{Best fit forcing intercept zero: $\mu(T)\approx 5.51\cdot 10^{-8}\; T$}}
        \end{axis}
    \end{tikzpicture}
    \caption[Values for the dynamic viscosity of nitrogen using the Sutherland formula:]{
        \textbf{Values for the dynamic viscosity of nitrogen using the Sutherland formula:}
        with addition of two square-error, best-fits of the Sutherland formula in the range of $200 < T < 600$, one with intercept being forced to zero.
    }
    \label{plt:sutherland}
\end{figure}

	
	To determine the location of highest Knudsen number, the characteristic length is assumed to be the throat diameter of the nozzles: $L_c = d_{2,4} = 2 \cdot 10^{-6}\;\text{m}$.
	
	This choice yields the largest Knudsen number in the system.
	Any part of the system with a larger $L_c$ will reach the continuum limit later.

	Additionally, by using the best-fit line with zero intercept and a slope of $k = 5.51 \cdot10^{-8}\; \frac{\text{sPa}}{\text{K}}$ from Figure~\ref{plt:sutherland}, the Knudsen number can be simplified to a proportionality relation:
	\begin{equation}
		Kn(p,T) \approx
		\frac{ k \cdot T }{ L_c } \sqrt{ \frac{ \pi R }{ 2 } } \cdot \frac{ \sqrt{ T } }{ p }
		\coloneqq \alpha \cdot \frac{ T^{ 3/2 } }{ p }
		\quad \rightarrow \quad
		Kn \propto \frac{ T^{ 3/2 } }{ p }
	\end{equation}

	This relation allows for an approximate evaluation of the constant $\alpha$ for the present case:
	$$
		\alpha = \frac{k}{L_c}\sqrt{\frac{\pi R}{2}}
	$$
	Using the system parameters:
	$$
		\alpha = \frac{5.51 \cdot 10^{-8}\; \frac{\text{sPa}}{\text{K}}}{2.0 \cdot 10^{-6}\;\text{m}} \sqrt{\frac{\pi \cdot 296\; \frac{\text{J}}{\text{kgK}}}{2}}
		\approx 0.06\; \frac{\text{Pa}}{\text{K}^{3/2}}
	$$

	With $\alpha$ known, the pressure at which the flow transitions to molecular behavior can be estimated for a given temperature:
	$$
		p_\text{trans} = \alpha \frac{T^{3/2}}{Kn} \quad \text{where} \quad Kn = 0.1
	$$
	For two representative temperatures:
	\begin{align*}
		T_{0,1} &= 300\;\text{K} \quad \to \quad p_{\text{trans},1} = 0.03\;\text{bar}\\
		T_{0,2} &= 500\;\text{K} \quad \to \quad p_{\text{trans},2} = 0.07\;\text{bar}
	\end{align*}

	These results show that extremely low pressures are required to force the flow into the molecular regime.
	Since the reactor is intended to operate at higher pressures, close to ambient conditions, the continuum assumption can be considered valid throughout the assembly.

	However, this analysis does not yet identify the specific location within the assembly where the Knudsen number is expected to be highest.
	The simplified relation already suggests that, for a given temperature, the Knudsen number increases as the local pressure decreases.
	To pinpoint the most probable location for a transition to molecular flow, the relationship between pressure, temperature, and Mach number must be examined.

	Since the flow is assumed to be isentropic and in the continuum regime, the pressure and temperature distribution can be related to the local Mach number.
	This is illustrated in the following plot:

	\begin{figure}[ht]
\centering
\begin{tikzpicture}[scale=0.85]
    \begin{axis}[
        width=\textwidth,
        height=0.6\textwidth,
        xlabel={Mach number $Ma$},
        ylabel={Ratio},
        xmin=0, xmax=3.5,
        ymin=0, ymax=1.2,
        grid=both,
        legend style={legend pos=north east}
    ]
    % Parameter: gamma
    \def\gamma{1.47}

    % T_i/T_0
    \addplot[
        domain=0:3.5,
        samples=200,
        thick,
        blueColor
    ]
    {
        (1 + 0.5*(\gamma-1)*x^2)^(-1)
    };
    \addlegendentry{$T/T_0$}

    % p_i/p_0
    \addplot[
        domain=0:3.5,
        samples=200,
        thick,
        redColor
    ]
    {
        (1 + 0.5*(\gamma-1)*x^2)^(-\gamma/(\gamma-1))
    };
    \addlegendentry{$p/p_0$}

    % rho_i/rho_0
    \addplot[
        domain=0:3.5,
        samples=200,
        thick,
        greenColor
    ]
    {
        (1 + 0.5*(\gamma-1)*x^2)^(-1/(\gamma-1))
    };
    \addlegendentry{$\rho/\rho_0$}

    \end{axis}
\end{tikzpicture}
\caption{Isentropic temperature, pressure, and density ratios (equations \eqref{eq:total_relation_T} - \eqref{eq:total_relation_rho}) as a function of Mach number for $\gamma=1.47$.}
\end{figure}


	From Figure~\ref{plt:isentropic-change-over-Ma}, it becomes clear that temperature decreases less rapidly with increasing Mach number compared to pressure.
	Consequently, in isentropic continuum flow, a reduction in pressure—and thus an increase in Knudsen number—is primarily driven by acceleration of the flow.

	In confined geometries, the flow is typically subsonic, only reaching sonic or supersonic velocities at the outlet, when discharging into a sufficiently low ambient pressure.
	Therefore, the location where the Knudsen number is highest, and where a transition to molecular behavior is most probable, is after the exit plane of the outlet nozzle.
	As stagnation pressure decreases, this transition may propagate further upstream into the assembly.

	\paragraph{Knudsen Number in Low-Pressure Zones}
		As the gas leaves the outlet nozzle and expands into the vacuum, the concept of a characteristic length loses its significance.
		This is because the walls of the vacuum chamber are far away compared to the geometric length scales inside the assembly.
		During expansion, the gas will continuously lose pressure to conform to the vacuum environment, eventually transitioning into free molecular flow.
		Consequently, formulations relying on the Mach number also become inapplicable.

		For this reason, a more general and elegant expression for the local Knudsen number $Kn_L$ is introduced, which is better suited for describing flow behavior in this regime:
		\begin{equation}
			Kn_L = \frac{\lambda}{\phi} \left| \frac{d\phi}{dx} \right|
		\end{equation}
		Here, $\lambda$ is the local mean free path, and $\phi$ is an arbitrary state variable of the flow.
		This formulation allows for the calculation of the Knudsen number throughout the expansion and can be used to identify contours where the transition from continuum to molecular flow occurs.
		\cite{bird_dsmc_2013, Grabe2008, LiLam1964}

\subsubsection*{Laminar Flow}
	While the Reynolds number does not directly affect the validity of the analytical formulations used, it plays a crucial role in shaping the actual flow behavior.
	In particular, it influences mixing and the development of turbulence, both of which affect key state variables as well as interactions of the gas with the reactive surface.

	For isentropic expansions, typical Reynolds numbers per unit length fall within the range $10^{-2} < Re/l < 1$.
	Given the small characteristic length of $L_c = 20 \cdot 10^{-6}\;\text{m}$, the flow is strongly constrained to low Reynolds numbers.
	\cite{ames1953compressible}
	As a result, the flow remains laminar throughout the assembly, with mixing primarily governed by molecular diffusion.
	\cite{comsol_microfluidics_guide}

\subsubsection*{Steady Flow}
	The assumption of steady flow is justified by the boundary conditions of the system.
	The reservoir pressure $p_0$ and temperature $T_0$ are held constant during operation.
	Thus, the flow is driven solely by the pressure differential between the reservoir and the vacuum.

	Once the initial unsteady effects have decayed, the system reaches an equilibrium state in which the flow remains steady over time.
	This assumption is valid both within the assembly and in the downstream expansion into the vacuum.
	\cite{LiLam1964}

	\newpage

\subsection{One-dimensional isentropic variable area flow}\label{sec:one-dim-isentropic}
		By assuming the flow through the assembly will be fully isentropic and pseudo one-dimensional it is possible to calculate the state variables at every point knowing the stagnation conditions and the ratio between the cross-sectional area at the point of interest $A_i$ and the at throat of the assembly $A^*$.\\
	It must be noted that this is a very radical approximation since for the flow to be  considered pseudo one-dimensional the problem must be reduced to consist purely of variable area ducts.
	This clearly overlooks the fact that when entering and leaving the reactor the gas has to perform a right angle turn to follow the flow path.
	Another constraint on the duct geometry to achieve reasonable solutions assuming pseudo one-dimensional flow is that the duct must change its cross-sectional area gradually.
	\cite{anderson2021modern}
	This won't be the case inside the reactor, since there is no way of slicing the reactor chamber to achieve a gradual change in cross-section, especially around the inlet and outlet.
	\begin{figure}[H]
	    \centering
	    \includegraphics[width=0.9\textwidth]{src/03_analytical-work/fig_1d-flow-geometry}
	    \caption{A descriptive caption for the figure.}
	\end{figure}
	This geometry can now be characterized as a double throat, and therefore one-dimensional isentropic flow will yield one of in two fundamental solutions.
	Depending on how the steady state in the system is reached, flow velocity of the stream will either stay subsonic and therefore decrease when entering the reactor or become sonic at the throat (2) and continue to accelerate.
	Other non-trivial solutions won't be discussed.
	\cite{SALAS1986193, EMMONS1958}

\subsubsection*{Calculations}

	The first step is to define the critical locations, where the flow fill be chocked.
	Since the outlet is expanding into vacuum, resulting in a pressure ratio tending towards zero, therefore the flow therefore must be chocked and can be recognized as a critical point.
	Maximum mass-flow occurs if the flow is chocked. Therefore, to keep up with the mass-flow of the outlet, the inlet must also be chocked: 
	$$
		A_{2,4},\;p_{2,4},\;\rho_{2,4},\;T_{2,4}\quad\xrightarrow{M=1}\quad A^*,\;p^*,\;\rho^*,\;T^*
	$$
	The second reference location corresponding to the stagnation or total conditions is at the entry of the inlet nozzle (1), which can be defined afterward to get quantitative solutions.
	$$
		A_1,\;p_1,\;\rho_1,\;T_1\quad\xrightarrow{M=0}\quad A_0,\;p_0,\;\rho_0,\;T_t
	$$
	Next step is to take the cross-sectional areas at every point defined in chapter \ref{sec:geometry}, and calculate their ratios to the throat areas.
	To use them for solving the following equation relating them to M, numerically yielding one subsonic and one supersonic solution depending on their initial conditions.
	$$
		\frac{A_i}{A^*} = \frac{1}{M_i} \left[ \frac{2}{\gamma + 1} \left( 1 + \frac{\gamma - 1}{2} M^2 \right) \right]^{\frac{\gamma + 1}{2(\gamma - 1)}}
		\qquad \eqref{eq:area_ratio_mach}
	$$
	Afterward the ratio of the local state variables to the total conditions can be determined, which can be used after, defining total conditions for the system, to calculate the local variables for every point.
	\cite{hall_isentropic_nodate}

	\begin{alignat*}{2}
	    \frac{T_i}{T_0}   & = \left( 1 + \frac{\gamma - 1}{2} M_i^2 \right)^{-1}
	    & \qquad & \eqref{eq:total_relation_T} \\
	    \frac{p_i}{p_0}   & = \left( 1 + \frac{\gamma - 1}{2} M_i^2 \right)^{-\frac{\gamma}{\gamma - 1}}
	    & \qquad & \eqref{eq:total_relation_p} \\
	    \frac{\rho_i}{\rho_0} & = \left( 1 + \frac{\gamma - 1}{2} M_i^2 \right)^{-\frac{1}{\gamma - 1}}
	    & \qquad & \eqref{eq:total_relation_rho}
	\end{alignat*}
	Where $p,\; \rho,\; T$ are the local gas conditions, $p_0,\; \rho_0,\; T_0$ the total gas conditions, $\gamma$ the specific heat ratio and $M_i$ the local mach number.
	This leads to the following solutions:
	\begin{table}[H]
\centering
\renewcommand{\arraystretch}{1.4} % Moderate row spacing
\begin{tabular}{|c|c|c|c|c|c|}
\hline
$i$                  & $\frac{A_i}{A^*}$       & M         & $\frac{p_i}{p_t}$        & $\frac{\rho_i}{\rho_t}$   & $\frac{T_i}{T_t}$      \\ \hline
1                    & 4                       & 0         & 1                        & 1                         & 1                      \\ \hline
2                    & 1                       & 1         & 0.52                     & 0.64                      & 0.81                   \\ \hline
\multirow{2}{*}{3}   & \multirow{2}{*}{318.31} & $\sim$0   & $\sim$1                  & $\sim$1                   & $\sim$1                \\ \cline{3-6} 
                     &                         & 10.55     & $3.28 \cdot 10^{-5}$     & $8.90 \cdot 10^{-4}$      & $3.68 \cdot 10^{-2}$   \\ \hline
4                    & 1                       & 1         & 0.52                     & 0.64                      & 0.81                   \\ \hline
\multirow{2}{*}{5}   & \multirow{2}{*}{4}      & 0.15      & 0.983                    & 0.988                     & 0.994                  \\ \cline{3-6} 
                     &                         & 3.06      & 0.02629                  & 0.08415                   & 0.31245                \\ \hline
\end{tabular}
\caption{Isentropic flow properties for different conditions.}
\label{tab:isentropic_flow}
\end{table}

	Mass flow is conserved along the flow and can be calculated using following equation, which derives from the equation for mass-flow in a steady one-dimensional flow \eqref{eq:1-d-massflow}, the isentropic relations [\eqref{eq:total_relation_T} - \eqref{eq:total_relation_p}] and the ideal gas law \eqref{eq:ideal_gas}.
	\cite{benson_mass_nodate}
	$$
		\dot{m} = A \cdot P_0 \cdot \sqrt{\frac{\gamma}{R T_0}} \cdot M \cdot \left(1 + \frac{\gamma - 1}{2} M^2\right)^{-\frac{\gamma + 1}{2(\gamma - 1)}}
	$$
	Where $A$ is the local cross-sectional area, $P_0$ the total pressure, $T_0$ the total temperature, $\gamma$ the specific heat ratio, $R$ the specific gas constant and $M$ the mach number.
	\cite{Cantwell_AA210A}

	Now knowing the approximate way state variables change inside the system, the assumption taken in chapter \ref{sec:expected-flow-regimes} regarding the Reynolds and Knudsen number can be tested for the following values:
	\begin{align*}
		p_{0,\;1} = 1.5\;\text{bar} &\qquad T_{0,\;1} = 300\;\text{K}\\
		p_{0,\;2} = 1.5\;\text{bar} &\qquad T_{0,\;2} = 500\;\text{K}
	\end{align*}
	Additionally, the lower limit for the reservoir pressure $p_{0,\;min}$ where some part of the system transitions to the molecular regime, will be determined.
	Again for temperatures $T_{0,\;1},\;T_{0,\;2}$
	

\subsubsection*{Interpretation}
\note{
	Go more into the solutions. Especially into the non physicality of the high mach numbers inside. Also mention non-isentropic processes. This is a idealized situation this will always lead to maximum mass flow in comparison to other formulation.
}

	\newpage

\subsection{Flow behaviors in micro channels}\label{sec:micro-channels}
	\cite{agrawal_comprehensive_2011}


	\newpage

\subsection{Including non-isentropic behaviors}\label{sec:disconnected-reservoirs}
	{\color{greenColor}\itshape
There is no reasonable analytical solution for the leakage.
Use measurements of pressure drop inside the chamber as reference and interpolate it to get solutions for higher pressure values.
This forces the reactor to be treated as a reservoir with some static pressure.
Does this still make sense in terms of conservation of mass?
Do we need to create some region of expansion and contraction around inlet and outlet to make the equations work?
}

	\newpage

\subsection{Under-expanded nozzle plume at outlet}\label{sec:outlet_plume}
	Once the gas emerges from the reactor and enters the outlet nozzle, it undergoes a rapid expansion into a low-pressure or near-vacuum environment.
This regime differs significantly from the upstream flow: velocity increases dramatically, density drops, and shock waves can appear.
The one-dimensional isentropic models used until now cannot fully capture these effects, especially if there is phase change (e.g., condensation) or a transition to free-molecular flow.
Hence, more advanced modeling approaches are required.
\cite{jousten_handbook_2016, anderson2021modern}

\subsubsection*{Method of characteristics}

	\paragraph*{Ideal nozzle design}
		The method of characteristics is a mathematical technique used to design supersonic nozzles so that gas flows expand smoothly from sonic to supersonic speeds without generating internal shocks.
		It works by tracing characteristic lines—paths along which flow properties remain constant—through the nozzle region.
		By aligning the walls with these lines it ensures that each incremental flow turn occurs through a series of controlled expansion waves, rather than abrupt angle changes that could cause shocks.
		If fully expanded this leads to a straight column of gas leaving the nozzle, where all of its energy is converted into kinetic energy without significant losses due to shocks.
		This approach is not feasible in this case since the length of the nozzle grows as the back pressure gets lower, the ideal nozzle for the high vacuum found in the surrounding chamber would not be able to fit due to space constraints.
		\cite{khare_rocket_2021, fernandes_shape_2023}

	\paragraph*{Determining flow field for free expansion}
		Mach lines are a concept usually encountered in two-dimensional supersonic flows. 
		These lines occur in pairs and are oriented at an angle $\mu = \arcsin\!\bigl(\frac{1}{M}\bigr)$ called the Mach angle, with respect to the direction of motion.
		In a 3-D flow field, these lines form a Mach cone, whose half-angle is the Mach angle itself.
		In an under-expanded jet, the method of characteristics can also be applied immediately downstream of the nozzle exit to predict the initial spread of the flow.
		It does this by mapping local Mach numbers and density profiles along characteristic lines, thus giving 
		a first approximation for velocity and pressure distributions.
		In supersonic expansions, Prandtl–Meyer expansion fans define the maximum turning angle $\theta_{max}$ the gas can undergo around a sharp corner.
		Which can be calculated using the following equation:
		\begin{equation*}
			\theta_{max} = \nu_{max} - \nu(Ma)
		\end{equation*}
		\begin{equation}
			\theta_{max} = \frac{\pi}{2} \left(\sqrt{\frac{\gamma + 1}{\gamma - 1}} - 1\right) - \sqrt{\frac{\gamma + 1}{\gamma - 1}}\arctan{\sqrt{\frac{\gamma - 1}{\gamma + 1}(Ma^2 - 1)}} + \arctan{\sqrt{Ma^2 -1}}
			\label{eq:max-turning-angle}
		\end{equation}
		Where $\nu_{max},\nu(Ma)$ are the Prandl-Meyer functions, $Ma$ is the mach number before the turn and $\gamma$ is the ratio of heats for the given gas.
		By incorporating these fans, the method of characteristics predicts the outer boundary of the flow—effectively identifying the angle outside which the density and mass flux become negligible.
		This angle is critical for determining the lateral extent of the under-expanded plume and for estimating where essentially all of mass flow is confined.
		Using equation \eqref{eq:max-turning-angle} for the mach number at the outlet nozzle exit calculated in section \ref{sec:one-dim-isentropic} the maximum turning angle yields:
		$$
			\theta_{max} = 68.9^\circ
			\qquad \text{with} \qquad
			Ma = 3.06\;,\quad \gamma = 1.47
		$$
		Notably, findings from Cassanova and Stephenson \cite{Cassanova1965} — even though the nozzle half-angle does not match — still offer valuable reference points regarding the resulting velocity field and associated flow features.
		However, a key limitation arises if condensation or other non-ideal processes occur.
		Experiments show that for working fluids like nitrogen, condensation in the expansion region can alter the thermodynamic state and produce deviations from these idealized solutions.
		Consequently, while the method of characteristics remains valuable for nozzle design and near-exit expansions, it cannot fully capture condensation effects or strong rarefaction phenomena further downstream.
		\cite{jousten_handbook_2016, robertson_investigation_1970, noauthor_zucrow_nodate}

\subsubsection*{Navier-Stokes Equations}
	A second option is to solve the Navier–Stokes equations numerically, using no-slip or first/second-order slip boundary conditions as appropriate.
	Near the nozzle exit—where the flow is still predominantly continuum—Navier–Stokes simulations can provide detailed velocity and pressure fields, including shock structures and boundary-layer effects.
	As the gas expands and its density decreases, however, continuum assumptions start to break down.
	While slip or transitional models can extend the validity range, ultimately they cannot fully describe the molecular-level interactions that dominate in highly rarefied regions.
	Consequently, beyond a certain distance from the nozzle exit or under conditions favorable to partial condensation, Navier–Stokes solutions must be supplemented by more advanced kinetic-based methods.
	\cite{anderson_fundamentals_2017, anderson2021modern}

\subsubsection*{Direct simulation Monte Carlo (DSMC)}
	To address the shortcomings of both the method of characteristics (under non-ideal conditions) and Navier–Stokes (when rarefaction becomes significant), the Direct Simulation Monte Carlo (DSMC) method is indispensable.
	DSMC performs a stochastic simulation of molecular motion and collisions, making it particularly well suited for:
	\begin{itemize}
		\item Transitional Regimes: Where the mean free path is comparable to or larger than the nozzle scale, DSMC accurately captures Knudsen-layer effects and velocity-slip phenomena.
		\item Condensation Phenomena: By including collision physics and intermolecular potentials, DSMC can handle the onset of condensation or two-phase flow more reliably than continuum or purely isentropic approaches. \cite{EMMONS1958}
		\item Free Expansion: As the jet spreads into the vacuum and transitions toward free-molecular flow, DSMC seamlessly handles extremely high Mach numbers and very low densities, maintaining physical fidelity across a wide range of Knudsen numbers.
	\end{itemize}
	Although more computationally expensive than continuum-based methods, DSMC provides the most robust depiction of outlet flows—particularly for condensable gases like nitrogen—and ensures high accuracy from near-nozzle regions all the way to the fully rarefied state.
	\cite{putignano2012supersonic, liu_study_2006}
	\newpage
\subsubsection*{Summary}
	Overall, while the method of characteristics remains a cornerstone of ideal nozzle design and near-exit expansions (including the prediction of Prandtl–Meyer fans and maximum turning angles), condensation and severe rarefaction invalidate its purely isentropic assumption further downstream.
	Navier–Stokes simulations can capture intermediate regimes near the exit, but eventually fail in the highly rarefied or partially condensed flow domain.
	For a comprehensive treatment of under-expanded plumes—particularly with nitrogen or other condensable working fluids—the DSMC technique offers the necessary level of detail, bridging the gap from continuum flow at the nozzle exit to free-molecular flow far downstream.

	\newpage
