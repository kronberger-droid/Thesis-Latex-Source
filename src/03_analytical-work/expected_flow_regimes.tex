\subsubsection*{Continuum Regime}
	The whole Theory relies heavily on the continuum model.
	For every location of the flow, but the vacuum, it will be assumed that it suffices to approximate the state variables ($T,\; p, \rho$) at certain locations using a simple continuum model presented in chapter \ref{sec:one-dim-isentropic} and use these state variables to calculate the Knudsen number.\\
	A sensible question to ask at this point is, where in the flow is it most probable to encounter the highest Knudsen number.
	Because this will reduce the locations to check for high Knudsen numbers when other formulations to calculate the state variables are used and therefore will simplify the process.
	To find such most probable Location, starting from the Definition of the Knudsen number.
	$$
		Kn(p,T) = \frac{\lambda}{L_c} = \frac{\mu(T)}{pL_c}\sqrt{\frac{\pi M_m R T}{2}}
	$$
	Where $\lambda$ is the mean free path, $L_c$ is the characteristic length, $k_B$ is the Boltzmann constant, $R$ is the specific gas constant, $T$ is the temperature of the fluid, $p$ is the pressure of the fluid, $M_m$ is the molar mass and $\mu$ is the dynamic viscosity.\\
	The dynamic viscosity can be calculated using Sutherland's formula.
	$$
		\mu(T) = \mu_0 \left(\frac{T}{T_0}\right)^{3/2} \frac{T_0 + S_\mu}{T + S_\mu}
	$$
	Where $mu_0$ is the reference viscosity at the reference temperature $T_0$ and $S_\mu$ representing the Sutherland constant, whose values are dependent on the chosen gas. For nitrogen these have the following values \cite{kim2004numericalanalysisflowcharacteristics}
	$$
		S_\mu = 111\;K\;,\quad T_0 = 300.55\;K\;,\quad\mu_0 = 17,81\; \text{Pa s} 
	$$
	Following plot shows the exact values using the Sutherland formula, plus two mean linear approximations in the range of 200-600 K, for the given values.
	One of them forcing the intercept to be zero.
	This will shift the values where Knudsen number reaches the continuum limit but won't have much influence on the behavior of the Knudsen number, thus version of the best-fit will be used for the following argument.
	\begin{figure}[H]
\centering
    \begin{tikzpicture}[scale=0.85]
        \begin{axis}[
            width=\textwidth,
            height=0.6\textwidth,
            xlabel={Temperature $T$ (K)},
            ylabel={Dynamic Viscosity $\mu$ (N·s/m$^2$)},
            xmin=200, xmax=600,
            ymin=0, ymax=4e-5,
            domain=200:600,
            grid=both,
            legend pos=north west,
        ]
            % Sutherland's formula for Air
            \addplot [
                thick,
                blueColor,
                samples=200
            ]
            {(1.716e-5) * (x/273)^(3/2) * (273+111)/(x+111)};
            \addlegendentry{Nitrogen ($\text{N}^2$) using Sutherlands formula}

            % Linear interpolation of the above function (dashed red)
            \addplot [
                thick,
                dashed,
                redColor
            ]
            {(4.18e-8)*x + 5.75e-6};
            \addlegendentry{\shortstack{Best fit: $\mu(T)\approx 4.18\cdot 10^{-8}\; T + 5.75\cdot 10^{-6}$}}

            % Linear interpolation of the above function (dashed red)
            \addplot [
                thick,
                dashed,
                greenColor
            ]
            {(5.51e-8)*x};
            \addlegendentry{\shortstack{Best fit forcing intercept zero: $\mu(T)\approx 5.51\cdot 10^{-8}\; T$}}
        \end{axis}
    \end{tikzpicture}
    \caption[Values for the dynamic viscosity of nitrogen using the Sutherland formula:]{
        \textbf{Values for the dynamic viscosity of nitrogen using the Sutherland formula:}
        with addition of two square-error, best-fits of the Sutherland formula in the range of $200 < T < 600$, one with intercept being forced to zero.
    }
    \label{plt:sutherland}
\end{figure}

	Assuming $L_c$ to be the throat diameter $A_{2,\;4}$ since this will cause the highest Knudsen number therefore a part of the system with a lower value of $L_c$ will just reach transient flow sooner.
	Also by using the best-fit line with intercept forced zero with a slope of $k = 5.51/cdot 10^{-8}$ from plot \ref{plt:sutherland} the Knudsen number can be reduced to a simple proportionality relation:
	$$
		Kn(p,T) \approx
		\frac{ k \cdot T }{ L_c } \sqrt{ \frac{ \pi M_m R}{ 2 } } \cdot \frac{ \sqrt{ T }}{ p }
		\coloneqq \alpha \cdot \frac{ T^{ 3/2 } }{ p }
		\quad \rightarrow \quad
		Kn \propto \frac{ T^{ 3/2 } }{ p }
	$$
	This right now doesn't just jet give a clear picture where in the assembly to expect the highest Knudsen number.
	But it already gives a clue that at a given temperature $T$, the only way to be able to reach low Knudsen numbers and be able to use continuum flow formulations, is to lower the pressure.

	To now pin down the location in a flow from high pressure to low pressure like in our case, the relation between temperature and pressure must be examined.

	\begin{figure}[ht]
\centering
\begin{tikzpicture}[scale=0.85]
    \begin{axis}[
        width=\textwidth,
        height=0.6\textwidth,
        xlabel={Mach number $Ma$},
        ylabel={Ratio},
        xmin=0, xmax=3.5,
        ymin=0, ymax=1.2,
        grid=both,
        legend style={legend pos=north east}
    ]
    % Parameter: gamma
    \def\gamma{1.47}

    % T_i/T_0
    \addplot[
        domain=0:3.5,
        samples=200,
        thick,
        blueColor
    ]
    {
        (1 + 0.5*(\gamma-1)*x^2)^(-1)
    };
    \addlegendentry{$T/T_0$}

    % p_i/p_0
    \addplot[
        domain=0:3.5,
        samples=200,
        thick,
        redColor
    ]
    {
        (1 + 0.5*(\gamma-1)*x^2)^(-\gamma/(\gamma-1))
    };
    \addlegendentry{$p/p_0$}

    % rho_i/rho_0
    \addplot[
        domain=0:3.5,
        samples=200,
        thick,
        greenColor
    ]
    {
        (1 + 0.5*(\gamma-1)*x^2)^(-1/(\gamma-1))
    };
    \addlegendentry{$\rho/\rho_0$}

    \end{axis}
\end{tikzpicture}
\caption{Isentropic temperature, pressure, and density ratios (equations \eqref{eq:total_relation_T} - \eqref{eq:total_relation_rho}) as a function of Mach number for $\gamma=1.47$.}
\end{figure}


	\note{
		Finish the argument: Change of pressure is more rapid than change in temperature. Also approximate $\alpha$ and then make a statement!
	}


	\paragraph{Knudsen Number in low pressure Zones}
		As the gas is leaving the outlet geometry and expands into the vacuum the characteristic length looses its significance.
		This is because the walls of the vacuum chamber are very far away in comparison to the length-scales of the flow geometry, while the gas expands it will lose pressure to conform to the vacuum and in that process will transition into free molecular flow.
		This leads to formulations using the mach number also loosing its significance.
		Therefore, it makes sense to identify a much more elegant way of calculating the local Knudsen number $Kn_L$ which will be much more applicable in this situation.
		$$
			K n_L = \frac{\lambda}{\phi} \left| \frac{d\phi}{dx} \right|
		$$
		Where $\lambda$ is the mean free path, $\phi$ is some state variable of the flow.
		This way the Knudsen number can be calculated throughout the expansion of the gas and can be used to find the contour lines where the transition between continuum and molecular flow will happen. 
		\cite{bird_dsmc_2013, Grabe2008, LiLam1964}

\subsubsection*{Laminar Flow}
	The Reynolds number itself does not bare any real significance for the applicability of the formulations used. Nonetheless, deviations in Reynolds number can in reality shape great parts of how the flow behaves and therefore effects important state variables.\\
	Reynolds numbers per unit length for isentropic expansion range $10^{-2} < Re/l < 1$, whereas the characteristic length is constant at $L_c = 20\cdot 10^{-6}$ and will therefore dominate the equation forcing low Reynolds numbers. \cite{ames1953compressible}\\
	Therefore the flow will stay laminar as long as it is contained inside the assembly and therefore mixing will be most likely diffusion mediated.
	\cite{comsol_microfluidics_guide}
	
\subsubsection*{Dimension of the Flow}

	\paragraph{Nozzle flow - pseudo 1D}
		In this section of the flow a further simplification of the two-dimensional flow through a duct will be used called "quasi one-dimensional" flow.
		This is possible by reducing the velocity distribution present at any point in a duct to its mean velocity.
		Therefore, reducing the velocity from a distribution $V(r)$ at every point in the duct to a scalar value $V$ at every point of the duct.
		This is a general simplification which can be made when using continuum flow analysis inside of ducts and is only bound by the applicability of continuum flow.
		This makes sense, because approximations of state variables and mass flow is the main goal here.
		Taking mean values still yield sensible solutions and won't leave us with less information.

	\paragraph*{Flow through the reactor - 3D}
		After the gas flow leaves the inlet nozzle and streams into the reactor chamber, the geometry of the assembly has no symmetry to reduce the dimension of the velocity field.
		Additionally, there is rapid expansion of the gas since the constraints given by the walls of the nozzle are now gone and the gas flows over a sharp corner.
		Therefore, the flow field has to be dependent on all spacial dimensions and essentially means the only way to get an accurate representation of the flow field the navier-stokes equations have to be solved.
		This won't be possible using only analytical tools, therefore in the following sections two simplifications of the flow through the reactor will be used.
		At first, assuming pseudo one-dimensional flow throughout the reactor, which is definitely wrong, since the flux area is neither expanding very slowly, nor the flow will be isentropic, due to the expansion into free space.
		To address this issue and still be able to use analytical tools to solve the system, in chapter \ref{sec:disconnected-reservoirs} the reactor will be assumed to be a reservoir.

	\paragraph{Free jet into vacuum - 2D}
		Similar to the flow into the reactor, the flow out of the system into the vacuum chamber won't be reducible to pseudo one-dimensional flow anymore.
		But this time the fact that the outlet nozzle is radially symmetric and there is no further geometry constraining the flow after it leaves the outlet.
		The flow can be at least assumed to inherit the radial symmetry from the outlet nozzle, thus reducing the spacial parameters of the flow field to the distance from the nozzle and the distance from the axis of symmetry of the outlet nozzle $r$.
		Furthermore, due to the gas expanding into vacuum and therefore the pressure dropping rapidly the gas will go through the process of rarefication, thus after some distance from the nozzle even continuum formulations break down.
		This will be discussed in more detail in chapter \ref{sec:outlet_plume} where possible ways to formulate solutions using numerical techniques.

	\cite{anderson2021modern}

\subsubsection*{Steady Flow}

	Steadiness of the flow is essentially given by the fact that the temperature $T_0$ and the pressure $p_0$ of the reservoir will be held constant during measurements.
	Thus, the flow will be driven by the pressure differential between the vacuum and the reservoir alone.
	So there is no reason for the flow to establish any dynamic behaviors after reaching some equilibrium state, after gas is first released into the system.
	This is true for regions inside as well as outside the assembly.
