\subsubsection*{Continuum Regime}
	The whole Theory relies heavily on the continuum model.
	For every location of the flow, but the vacuum, it will be assumed that it suffices to approximate the state variables (T, p, ρ) at certain locations using a simple one-dimensional continuum model presented in more detail in Section~\ref{sec:one-dim-isentropic} and use these state variables to calculate the Knudsen number.

	The plot in Figure~\ref{fig:knudsen-reynolds-plot} was created by calculating local temperatures and pressures using Equations~\eqref{eq:total_relation_T} and \eqref{eq:total_relation_p} for the reference total conditions given in Section~\ref{sec:geometry}.
	Further, calculating Knudsen number using Equation~\eqref{eq:knudsen-number}, where dynamic viscosity is expressed using Sutherland's formula~\eqref{eq:sutherland} with the following empirical constants \cite{kim2004numericalanalysisflowcharacteristics}:
	$$
		S_\mu = 111\;\text{K} \qquad T_{ref} = 300.55\;\text{K} \qquad \mu_{ref} = 17.81\; \text{sPa}
	$$
	Now the corresponding Reynolds numbers are calculated using the known relation between dimensionless numbers as stated in Equation~\eqref{eq:nondim-relation}.
	The values are plotted for two different characteristic lengths, $L_c = 20\;\mu m$ corresponding to throats of the inlet and outlet nozzles and $L_c = 40\;\mu m$ corresponding to the reactor volume and the maximum diameters of the inlet and outlet nozzles.

	Figure~\ref{fig:knudsen-reynolds-plot} clearly shows that for a flow with Mach numbers under 3.5 within the assembly, Knudsen numbers typically remain below 0.1, validating the continuum assumption.
	Although some regions may fall within the slip regime -- where slip boundary conditions have to be introduced to yield more accurate solutions -- continuum-based models remain applicable. 
	At the same time, the Reynolds number remains well under 2000, indicating laminar flow, which points to minimal mixing inside the assembly \cite{ames1953compressible, comsol_microfluidics_guide}.

	\begin{figure}[H]
		\centering
		\begin{center}
\begin{tikzpicture}
  \begin{groupplot}[
    group style={
      group size=2 by 1,
      horizontal sep=0.2cm,   % very tight spacing
    },
    width=7cm,
    height=6cm,
    xmin=0, xmax=3.5,
    xlabel={\footnotesize Mach number},
    tick label style={font=\footnotesize},
    minor x tick num=4,
    minor y tick num=4,
    clip=true,
    % A single legend above both plots:
    legend style={
      at={(1,1.1)},
      anchor=south,
      legend columns=2,
      draw=none,
      font=\scriptsize,
    },
    % turn off automatic legend entries from plots
    every axis plot/.append style={forget plot},
  ]

  %────────────────────────────────────────────────────────────
  % LEFT: Reynolds
  %────────────────────────────────────────────────────────────
  \nextgroupplot[
    ymode=log,
    ymin=1, ymax=2000,
    % draw bottom & left borders only:
    axis x line=bottom,
    axis y line=box,
    axis line style={-},
    % left ticks only:
    ytick pos=left,
    xtick pos=bottom,
    ylabel={\footnotesize Reynolds number},
    % after finishing this axis, draw the right border:
    after end axis={
      \draw[black,line width=0.8pt]
        (rel axis cs:1,0) -- (rel axis cs:1,1);
    },
  ]    % Manual legend: colours = temperature
    \addlegendimage{redColor,  solid, line width=1pt}
    \addlegendentry{\(T_0 = 600\,\mathrm K\)}
    \addlegendimage{black, solid, line width=1pt}
    \addlegendentry{\(L_c = 40\,\mu\mathrm m\)}
    % Manual legend: line‐styles = characteristic length
    \addlegendimage{blueColor,   solid, line width=1pt}
    \addlegendentry{\(T_0 = 300\,\mathrm K\)}
    \addlegendimage{black!50, solid, line width=1pt}
    \addlegendentry{\(L_c = 20\,\mu\mathrm m\)}

    % now the four Reynolds‐curves
    \addplot[blueColor!50, solid, line width=1pt] table[x=Ma,y=Re300_20] {knre_data.dat};
    \addplot[blueColor, solid, line width=1pt] table[x=Ma,y=Re300_40] {knre_data.dat};
    \addplot[redColor!50, solid, line width=1pt] table[x=Ma,y=Re600_20] {knre_data.dat};
    \addplot[redColor, solid, line width=1pt] table[x=Ma,y=Re600_40] {knre_data.dat};
    \addplot[greenColor, dashed, line width=1.5pt] table[x=Ma,y expr=2000] {knre_data.dat};

  %────────────────────────────────────────────────────────────
  % RIGHT: Knudsen
  %────────────────────────────────────────────────────────────
   \nextgroupplot[
     ymode=log,
     ymin=1e-3, ymax=1e-1,
     % draw bottom & left borders only:
     axis x line=bottom,
     axis y line=box,
     axis line style={-},
     % right ticks:
     ytick pos=right,
     xtick pos=bottom,
     ylabel={\footnotesize Knudsen number},
     % after finishing this axis, draw the right border:
     after end axis={
          \draw[black,line width=0.8pt]
            (rel axis cs:0,0) -- (rel axis cs:0,1);
        },
      ]
    % the same four style combinations
    \addplot[blueColor!50, solid, line width=1pt] table[x=Ma,y=Kn300_20] {knre_data.dat};
    \addplot[blueColor, solid, line width=1pt] table[x=Ma,y=Kn300_40] {knre_data.dat};
    \addplot[redColor!50, solid,  line width=1pt] table[x=Ma,y=Kn600_20] {knre_data.dat};
    \addplot[redColor, solid, line width=1pt] table[x=Ma,y=Kn600_40] {knre_data.dat};
    % reference line (omit from legend)
    \addplot[greenColor, dashed, line width=1.5pt] table[x=Ma,y expr=0.1] {knre_data.dat};

  \end{groupplot}
\end{tikzpicture}
\end{center}

		\caption[Reynolds and Knudsen numbers as functions of Mach number on logarithmic scales.]{
			\textbf{Reynolds and Knudsen numbers as functions of Mach number on logarithmic scales:}
			The plot shows values of $\mathrm{Re}$ and $\mathrm{Kn}$ for two temperatures ($T_0 = 300\,\mathrm{K}$ and $T_0 = 600\,\mathrm{K}$) and two characteristic lengths ($L_c = 20\,\mu\mathrm{m}$ and $L_c = 40\,\mu\mathrm{m}$), with each combination encoded via line style and marker shape.
			The reference pressure $p_\mathrm{ref}$ is fixed and chosen such that it corresponds to the limiting cases of the considered $T_0$ and $L_c$.
			The horizontal dotted red line indicates the threshold ($\mathrm{Kn} = 0.1$, $\mathrm{Re} = 2000$) beyond which transitional flow behavior is expected.
		}
		\label{fig:knudsen-reynolds-plot}
	\end{figure}

	\paragraph{Transition to Free Molecular Flow.}
		Upon exiting the nozzle into vacuum, the gas pressure drops rapidly, leading to a potentially large mean free path.
		Here, the continuum approximation no longer holds, and using a single characteristic length becomes meaningless.
		Instead, a local Knudsen number is more appropriate:
		\begin{equation}
		  Kn_L
		  = \frac{\lambda}{\phi}\left|\frac{d\phi}{dx}\right|,
		\end{equation}
		where \(\phi\) is a flow property (e.g., \(p\), \(\rho\), \(T\), or \(U\)).
		This definition enables identifying where the flow transitions from continuum to free molecular behavior \cite{bird_dsmc_2013,Grabe2008,LiLam1964}.
		Since the focus here is on internal assembly flow, we do not account for rarefied effects outside the nozzle and omit slip-layer modeling for simplicity.

\subsubsection*{Steady Flow}
	The assumption of steady flow is justified by the boundary conditions of the system.
	The reservoir pressure $p_0$ and temperature $T_0$ are held constant during operation.
	Thus, the flow is driven solely by the pressure differential between the reservoir and the vacuum.

	Once the initial unsteady effects have decayed, the system reaches an equilibrium state in which the flow remains steady over time.
	This assumption is valid both within the assembly and in the downstream expansion into the vacuum.
	\cite{LiLam1964}
