\subsubsection*{Continuum Regime}
	The whole Theory relies heavily on the continuum model.
	For every location of the flow, but the vacuum, it will suffice to approximate the state variables at certain locations using a simple continuum model and use these state variables to calculate the Knudsen number.\\
	A sensible question to ask at this point is: Where in the flow it is the most probable encounter the highest Knudsen number.
	Because this will reduce the locations to check for high Knudsen numbers and therefore will simplify the process.
	To find such most probable Location, given the Definition of the Knudsen number:
	$$
		Kn(p,T) = \frac{\lambda}{L_c} = \frac{\mu(T)R}{pL_c}\sqrt{\frac{\pi m T}{2k_B}}
	$$
	Where $\lambda$ is the mean free path, $L_c$ is the characteristic length, $k_B$ is the Boltzmann constant, $R$ is the specific gas constant, $T$ is the temperature of the fluid, $p$ is the pressure of the fluid, $m$ is the molecular mass and $\mu$ is the dynamic viscosity.\\
	The dynamic viscosity can be calculated using Sutherland's formula.
	$$
		\mu(T) = \mu_0 \left(\frac{T}{T_0}\right)^{3/2} \frac{T_0 + S_\mu}{T + S_\mu}
	$$
	\begin{figure}[H]
\centering
    \begin{tikzpicture}[scale=0.85]
        \begin{axis}[
            width=\textwidth,
            height=0.6\textwidth,
            xlabel={Temperature $T$ (K)},
            ylabel={Dynamic Viscosity $\mu$ (N·s/m$^2$)},
            xmin=200, xmax=600,
            ymin=0, ymax=4e-5,
            domain=200:600,
            grid=both,
            legend pos=north west,
        ]
            % Sutherland's formula for Air
            \addplot [
                thick,
                blueColor,
                samples=200
            ]
            {(1.716e-5) * (x/273)^(3/2) * (273+111)/(x+111)};
            \addlegendentry{Nitrogen ($\text{N}^2$) using Sutherlands formula}

            % Linear interpolation of the above function (dashed red)
            \addplot [
                thick,
                dashed,
                redColor
            ]
            {(4.18e-8)*x + 5.75e-6};
            \addlegendentry{\shortstack{Best fit: $\mu(T)\approx 4.18\cdot 10^{-8}\; T + 5.75\cdot 10^{-6}$}}

            % Linear interpolation of the above function (dashed red)
            \addplot [
                thick,
                dashed,
                greenColor
            ]
            {(5.51e-8)*x};
            \addlegendentry{\shortstack{Best fit forcing intercept zero: $\mu(T)\approx 5.51\cdot 10^{-8}\; T$}}
        \end{axis}
    \end{tikzpicture}
    \caption[Values for the dynamic viscosity of nitrogen using the Sutherland formula:]{
        \textbf{Values for the dynamic viscosity of nitrogen using the Sutherland formula:}
        with addition of two square-error, best-fits of the Sutherland formula in the range of $200 < T < 600$, one with intercept being forced to zero.
    }
    \label{plt:sutherland}
\end{figure}

	$L_c$ can be assumed constant since the height of the reactor and the smallest diameter of the ducts match.
	And by assuming $\mu$ to be close to constant the Knudsen number becomes a simple proportionality relation.
	$$
		Kn(p,T)\approx\frac{\mu R}{L_c}\sqrt{\frac{\pi m}{2k_B}}\cdot\frac{\sqrt{T}}{p}=\alpha\cdot\frac{\sqrt{T}}{p}\quad\rightarrow\quad Kn\propto \frac{\sqrt{T}}{p}
	$$
	This makes it obvious that areas with low pressure will lead to higher Knudsen numbers and will come closest to the limit of $Kn=0.1$ where continuum regime formulations stop to yield sensible solutions.
	Therefore, calculating the Knudsen number where the gas leaves the outlet nozzle will be useful to identify the flow regime that governs the gas flow inside the whole assembly.

	\note{
		Refine this argument!
	}

	\paragraph{Knudsen Number in low pressure Zones}
		As the gas is leaving the outlet geometry and expands into the vacuum the characteristic length looses its significance.
		This is because the walls of the vacuum chamber are very far away in comparison to the length-scales of the flow geometry, while the gas expands it will lose pressure to conform to the vacuum and in that process will transition into free molecular flow.
		This leads to formulations using the mach number also loosing its significance.
		Therefore, it makes sense to identify a much more elegant way of calculating the local Knudsen number $Kn_L$ which will be much more applicable in this situation.
		$$
			K n_L = \frac{\lambda}{\phi} \left| \frac{d\phi}{dx} \right|
		$$
		Where $\lambda$ is the mean free path, $\phi$ is some state variable of the flow.
		This way the Knudsen number can be calculated throughout the expansion of the gas and can be used to find the contour lines where the transition between continuum and molecular flow will happen. 
		\cite{bird_dsmc_2013, Grabe2008, LiLam1964}

\subsubsection*{Laminar Flow}
	The Reynolds number itself does not bare any real significance for the applicability of the formulations used. Nonetheless, deviations in Reynolds number can in reality shape great parts of how the flow behaves and therefore effects important state variables.\\
	Reynolds numbers per unit length for isentropic expansion range $10^{-2} < Re/l < 1$, whereas the characteristic length is constant at $L_c = 20\cdot 10^{-6}$ and will therefore dominate the equation forcing low Reynolds numbers. \cite{ames1953compressible}\\
	Therefore the flow will stay laminar as long as it is contained inside the assembly and therefore mixing will be most likely diffusion mediated.
	\cite{comsol_microfluidics_guide}
	
\subsubsection*{Dimension of the Flow}

	\paragraph{Nozzle flow - pseudo 1D}
		In this section of the flow a further simplification of the two-dimensional flow through a duct will be used called "quasi one-dimensional" flow.
		This is possible by reducing the velocity distribution present at any point in a duct to its mean velocity.
		Therefore, reducing the velocity from a distribution $V(r)$ at every point in the duct to a scalar value $V$ at every point of the duct.
		This is a general simplification which can be made when using continuum flow analysis inside of ducts and is only bound by the applicability of continuum flow.
		This makes sense, because approximations of state variables and mass flow is the main goal here.
		Taking mean values still yield sensible solutions and won't leave us with less information.

	\paragraph*{Flow through the chamber - 3D}

	\paragraph{Free jet into vacuum - 2D}

	\note{
		Refine sections above. They are not good!
	}

	\cite{anderson2021modern}

\subsubsection*{Steady Flow}

	Steadiness of the flow is essentially given by the fact that the temperature $T_0$ and the pressure $p_0$ of the reservoir will be held constant during measurements.
	Thus, the flow will be driven by the pressure differential between the vacuum and the reservoir alone.
	So there is no reason for the flow to establish any dynamic behaviors after reaching some equilibrium state, after gas is first released into the system.
	This is true for regions inside as well as outside the assembly.
