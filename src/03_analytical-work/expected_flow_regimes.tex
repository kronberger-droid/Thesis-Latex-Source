\subsubsection*{Continuum Regime}
	The theoretical framework developed in this thesis relies heavily on the continuum flow assumption.
	For every location along the flow path, except in the vacuum region, it is assumed that the state variables ($T$, $p$, $\rho$) can be approximated using the continuum model introduced in Chapter~\ref{sec:one-dim-isentropic}.
	These state variables will then be used to evaluate the Knudsen number.

	A sensible question at this point is: Where in the flow is the Knudsen number expected to be highest?
	Identifying this location helps to limit the analysis to specific regions when assessing the validity of the continuum assumption and simplifies the evaluation.

	To answer this, we start from the definition of the Knudsen number.
	\cite{halwidl_development_2016, anderson2021modern}
	$$
		Kn(p,T) = \frac{\lambda}{L_c} = \frac{\mu(T)}{pL_c}\sqrt{\frac{\pi R T}{2}}
	$$
	Here, $\lambda$ is the mean free path, $L_c$ is the characteristic length, $R$ is the specific gas constant, $T$ is the fluid temperature, $p$ is the fluid pressure, $M_m$ is the molar mass, and $\mu$ is the dynamic viscosity.

	The dynamic viscosity is evaluated using Sutherland's formula.
	\cite{Hirschfelder1954MolecularTO}
	\begin{equation}
		\mu(T) = \mu_0 \left(\frac{T}{T_0}\right)^{3/2} \frac{T_0 + S_\mu}{T + S_\mu}
		\label{eq:sutherland}
	\end{equation}
	Here, $\mu_0$ is the reference viscosity at the reference temperature $T_0$, and $S_\mu$ is the Sutherland constant, both dependent on the gas species.
	For nitrogen, the following values apply:
	\cite{kim2004numericalanalysisflowcharacteristics}
	$$
		S_\mu = 111\;\text{K}, \quad T_0 = 300.55\;\text{K}, \quad \mu_0 = 17.81\; \text{sPa}
	$$
	The following plot shows the dynamic viscosity over temperature using Sutherland's formula, along with two linear approximations over the temperature range of 200–600~K.
	One of the approximations is constrained to pass through the origin.
	While this shifts the transition point to molecular flow slightly, it does not significantly affect the general behavior of the Knudsen number.
	For simplicity, the best-fit line with zero intercept will be used in the following argument.
	\newpage
	\begin{figure}[H]
\centering
    \begin{tikzpicture}[scale=0.85]
        \begin{axis}[
            width=\textwidth,
            height=0.6\textwidth,
            xlabel={Temperature $T$ (K)},
            ylabel={Dynamic Viscosity $\mu$ (N·s/m$^2$)},
            xmin=200, xmax=600,
            ymin=0, ymax=4e-5,
            domain=200:600,
            grid=both,
            legend pos=north west,
        ]
            % Sutherland's formula for Air
            \addplot [
                thick,
                blueColor,
                samples=200
            ]
            {(1.716e-5) * (x/273)^(3/2) * (273+111)/(x+111)};
            \addlegendentry{Nitrogen ($\text{N}^2$) using Sutherlands formula}

            % Linear interpolation of the above function (dashed red)
            \addplot [
                thick,
                dashed,
                redColor
            ]
            {(4.18e-8)*x + 5.75e-6};
            \addlegendentry{\shortstack{Best fit: $\mu(T)\approx 4.18\cdot 10^{-8}\; T + 5.75\cdot 10^{-6}$}}

            % Linear interpolation of the above function (dashed red)
            \addplot [
                thick,
                dashed,
                greenColor
            ]
            {(5.51e-8)*x};
            \addlegendentry{\shortstack{Best fit forcing intercept zero: $\mu(T)\approx 5.51\cdot 10^{-8}\; T$}}
        \end{axis}
    \end{tikzpicture}
    \caption[Values for the dynamic viscosity of nitrogen using the Sutherland formula:]{
        \textbf{Values for the dynamic viscosity of nitrogen using the Sutherland formula:}
        with addition of two square-error, best-fits of the Sutherland formula in the range of $200 < T < 600$, one with intercept being forced to zero.
    }
    \label{plt:sutherland}
\end{figure}

	
	To determine the location of highest Knudsen number, the characteristic length is assumed to be the throat diameter of the nozzles: $L_c = d_{2,4} = 2 \cdot 10^{-6}\;\text{m}$.
	
	This choice yields the largest Knudsen number in the system.
	Any part of the system with a larger $L_c$ will reach the continuum limit later.

	Additionally, by using the best-fit line with zero intercept and a slope of $k = 5.51 \cdot10^{-8}\; \frac{\text{sPa}}{\text{K}}$ from Figure~\ref{plt:sutherland}, the Knudsen number can be simplified to a proportionality relation:
	\begin{equation}
		Kn(p,T) \approx
		\frac{ k \cdot T }{ L_c } \sqrt{ \frac{ \pi R }{ 2 } } \cdot \frac{ \sqrt{ T } }{ p }
		\coloneqq \alpha \cdot \frac{ T^{ 3/2 } }{ p }
		\quad \rightarrow \quad
		Kn \propto \frac{ T^{ 3/2 } }{ p }
	\end{equation}

	This relation allows for an approximate evaluation of the constant $\alpha$ for the present case:
	$$
		\alpha = \frac{k}{L_c}\sqrt{\frac{\pi R}{2}}
	$$
	Using the system parameters:
	$$
		\alpha = \frac{5.51 \cdot 10^{-8}\; \frac{\text{sPa}}{\text{K}}}{2.0 \cdot 10^{-6}\;\text{m}} \sqrt{\frac{\pi \cdot 296\; \frac{\text{J}}{\text{kgK}}}{2}}
		\approx 0.06\; \frac{\text{Pa}}{\text{K}^{3/2}}
	$$

	With $\alpha$ known, the pressure at which the flow transitions to molecular behavior can be estimated for a given temperature:
	$$
		p_\text{trans} = \alpha \frac{T^{3/2}}{Kn} \quad \text{where} \quad Kn = 0.1
	$$
	For two representative temperatures:
	\begin{align*}
		T_{0,1} &= 300\;\text{K} \quad \to \quad p_{\text{trans},1} = 0.03\;\text{bar}\\
		T_{0,2} &= 500\;\text{K} \quad \to \quad p_{\text{trans},2} = 0.07\;\text{bar}
	\end{align*}

	These results show that extremely low pressures are required to force the flow into the molecular regime.
	Since the reactor is intended to operate at higher pressures, close to ambient conditions, the continuum assumption can be considered valid throughout the assembly.

	However, this analysis does not yet identify the specific location within the assembly where the Knudsen number is expected to be highest.
	The simplified relation already suggests that, for a given temperature, the Knudsen number increases as the local pressure decreases.
	To pinpoint the most probable location for a transition to molecular flow, the relationship between pressure, temperature, and Mach number must be examined.

	Since the flow is assumed to be isentropic and in the continuum regime, the pressure and temperature distribution can be related to the local Mach number.
	This is illustrated in the following plot:

	\begin{figure}[ht]
\centering
\begin{tikzpicture}[scale=0.85]
    \begin{axis}[
        width=\textwidth,
        height=0.6\textwidth,
        xlabel={Mach number $Ma$},
        ylabel={Ratio},
        xmin=0, xmax=3.5,
        ymin=0, ymax=1.2,
        grid=both,
        legend style={legend pos=north east}
    ]
    % Parameter: gamma
    \def\gamma{1.47}

    % T_i/T_0
    \addplot[
        domain=0:3.5,
        samples=200,
        thick,
        blueColor
    ]
    {
        (1 + 0.5*(\gamma-1)*x^2)^(-1)
    };
    \addlegendentry{$T/T_0$}

    % p_i/p_0
    \addplot[
        domain=0:3.5,
        samples=200,
        thick,
        redColor
    ]
    {
        (1 + 0.5*(\gamma-1)*x^2)^(-\gamma/(\gamma-1))
    };
    \addlegendentry{$p/p_0$}

    % rho_i/rho_0
    \addplot[
        domain=0:3.5,
        samples=200,
        thick,
        greenColor
    ]
    {
        (1 + 0.5*(\gamma-1)*x^2)^(-1/(\gamma-1))
    };
    \addlegendentry{$\rho/\rho_0$}

    \end{axis}
\end{tikzpicture}
\caption{Isentropic temperature, pressure, and density ratios (equations \eqref{eq:total_relation_T} - \eqref{eq:total_relation_rho}) as a function of Mach number for $\gamma=1.47$.}
\end{figure}


	From Figure~\ref{plt:isentropic-change-over-Ma}, it becomes clear that temperature decreases less rapidly with increasing Mach number compared to pressure.
	Consequently, in isentropic continuum flow, a reduction in pressure—and thus an increase in Knudsen number—is primarily driven by acceleration of the flow.

	In confined geometries, the flow is typically subsonic, only reaching sonic or supersonic velocities at the outlet, when discharging into a sufficiently low ambient pressure.
	Therefore, the location where the Knudsen number is highest, and where a transition to molecular behavior is most probable, is after the exit plane of the outlet nozzle.
	As stagnation pressure decreases, this transition may propagate further upstream into the assembly.

	\paragraph{Knudsen Number in Low-Pressure Zones}
		As the gas leaves the outlet nozzle and expands into the vacuum, the concept of a characteristic length loses its significance.
		This is because the walls of the vacuum chamber are far away compared to the geometric length scales inside the assembly.
		During expansion, the gas will continuously lose pressure to conform to the vacuum environment, eventually transitioning into free molecular flow.
		Consequently, formulations relying on the Mach number also become inapplicable.

		For this reason, a more general and elegant expression for the local Knudsen number $Kn_L$ is introduced, which is better suited for describing flow behavior in this regime:
		\begin{equation}
			Kn_L = \frac{\lambda}{\phi} \left| \frac{d\phi}{dx} \right|
		\end{equation}
		Here, $\lambda$ is the local mean free path, and $\phi$ is an arbitrary state variable of the flow.
		This formulation allows for the calculation of the Knudsen number throughout the expansion and can be used to identify contours where the transition from continuum to molecular flow occurs.
		\cite{bird_dsmc_2013, Grabe2008, LiLam1964}

\subsubsection*{Laminar Flow}
	While the Reynolds number does not directly affect the validity of the analytical formulations used, it plays a crucial role in shaping the actual flow behavior.
	In particular, it influences mixing and the development of turbulence, both of which affect key state variables as well as interactions of the gas with the reactive surface.

	For isentropic expansions, typical Reynolds numbers per unit length fall within the range $10^{-2} < Re/l < 1$.
	Given the small characteristic length of $L_c = 20 \cdot 10^{-6}\;\text{m}$, the flow is strongly constrained to low Reynolds numbers.
	\cite{ames1953compressible}
	As a result, the flow remains laminar throughout the assembly, with mixing primarily governed by molecular diffusion.
	\cite{comsol_microfluidics_guide}

\subsubsection*{Steady Flow}
	The assumption of steady flow is justified by the boundary conditions of the system.
	The reservoir pressure $p_0$ and temperature $T_0$ are held constant during operation.
	Thus, the flow is driven solely by the pressure differential between the reservoir and the vacuum.

	Once the initial unsteady effects have decayed, the system reaches an equilibrium state in which the flow remains steady over time.
	This assumption is valid both within the assembly and in the downstream expansion into the vacuum.
	\cite{LiLam1964}
