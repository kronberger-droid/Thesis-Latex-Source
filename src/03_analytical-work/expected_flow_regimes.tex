\subsubsection{Continuum Regime}
The Knudsen number (\(Kn\)) serves as a crucial indicator for the validity of continuum theory: low values (\(Kn < 0.1\)) justify treating the fluid as a continuous medium, while higher values signal a transition into rarefied (transitional or free-molecular) flow \cite{halwidl_development_2016,anderson2021modern}. Since the present work relies on the continuum assumption, we must ensure that \(Kn\) remains sufficiently small throughout the flow domain. 

However, due to the micro-scale geometry involved, parts of the flow may enter the slip regime (\(0.001 \lesssim Kn < 0.1\)). In this range, continuum models remain largely valid, but additional slip-layer boundary conditions are required to fully capture near-wall effects. Implementing such slip boundary conditions can be complex, and therefore we omit them here.

\paragraph{Knudsen Number.}
	The Knudsen number is given by
	\begin{equation}
	  Kn(p,T) 
	  = \frac{\lambda}{L_c} 
	  = \frac{\mu(T)}{p\,L_c}\,\sqrt{\frac{\pi\,T\,R_u}{2\,M_m}},
	\end{equation}
	where \(\lambda\) is the mean free path, \(L_c\) a characteristic length, \(\mu(T)\) the dynamic viscosity, \(p\) the local pressure, \(T\) the temperature, \(M_m\) the molar mass, and \(R_u\) the universal gas constant. The temperature dependence of \(\mu\) is described by Sutherland’s law:
	\begin{equation}
	  \mu(T)
	  = \mu_{\text{ref}}
	    \left(\frac{T}{T_{\text{ref}}}\right)^{3/2}
	    \frac{T_{\text{ref}} + S_\mu}{T + S_\mu},
	\end{equation}
	with \(S_\mu=111\,\mathrm{K}\), \(T_{\text{ref}}=300.55\,\mathrm{K}\), and \(\mu_{\text{ref}}=17.81 \times 10^{-6}\,\mathrm{Pa\,s}\) for nitrogen \cite{kim2004numericalanalysisflowcharacteristics}.

	\paragraph{Reynolds Number.}
	To assess the potential for turbulence, we compute the Reynolds number,
	\begin{equation}
	  Re = \frac{\rho\,U\,L_c}{\mu(T)},
	\end{equation}
	where \(\rho\) is the density and \(U\) the flow velocity.

	\begin{center}
\begin{tikzpicture}
  \begin{groupplot}[
    group style={
      group size=2 by 1,
      horizontal sep=0.2cm,   % very tight spacing
    },
    width=7cm,
    height=6cm,
    xmin=0, xmax=3.5,
    xlabel={\footnotesize Mach number},
    tick label style={font=\footnotesize},
    minor x tick num=4,
    minor y tick num=4,
    clip=true,
    % A single legend above both plots:
    legend style={
      at={(1,1.1)},
      anchor=south,
      legend columns=2,
      draw=none,
      font=\scriptsize,
    },
    % turn off automatic legend entries from plots
    every axis plot/.append style={forget plot},
  ]

  %────────────────────────────────────────────────────────────
  % LEFT: Reynolds
  %────────────────────────────────────────────────────────────
  \nextgroupplot[
    ymode=log,
    ymin=1, ymax=2000,
    % draw bottom & left borders only:
    axis x line=bottom,
    axis y line=box,
    axis line style={-},
    % left ticks only:
    ytick pos=left,
    xtick pos=bottom,
    ylabel={\footnotesize Reynolds number},
    % after finishing this axis, draw the right border:
    after end axis={
      \draw[black,line width=0.8pt]
        (rel axis cs:1,0) -- (rel axis cs:1,1);
    },
  ]    % Manual legend: colours = temperature
    \addlegendimage{redColor,  solid, line width=1pt}
    \addlegendentry{\(T_0 = 600\,\mathrm K\)}
    \addlegendimage{black, solid, line width=1pt}
    \addlegendentry{\(L_c = 40\,\mu\mathrm m\)}
    % Manual legend: line‐styles = characteristic length
    \addlegendimage{blueColor,   solid, line width=1pt}
    \addlegendentry{\(T_0 = 300\,\mathrm K\)}
    \addlegendimage{black!50, solid, line width=1pt}
    \addlegendentry{\(L_c = 20\,\mu\mathrm m\)}

    % now the four Reynolds‐curves
    \addplot[blueColor!50, solid, line width=1pt] table[x=Ma,y=Re300_20] {knre_data.dat};
    \addplot[blueColor, solid, line width=1pt] table[x=Ma,y=Re300_40] {knre_data.dat};
    \addplot[redColor!50, solid, line width=1pt] table[x=Ma,y=Re600_20] {knre_data.dat};
    \addplot[redColor, solid, line width=1pt] table[x=Ma,y=Re600_40] {knre_data.dat};
    \addplot[greenColor, dashed, line width=1.5pt] table[x=Ma,y expr=2000] {knre_data.dat};

  %────────────────────────────────────────────────────────────
  % RIGHT: Knudsen
  %────────────────────────────────────────────────────────────
   \nextgroupplot[
     ymode=log,
     ymin=1e-3, ymax=1e-1,
     % draw bottom & left borders only:
     axis x line=bottom,
     axis y line=box,
     axis line style={-},
     % right ticks:
     ytick pos=right,
     xtick pos=bottom,
     ylabel={\footnotesize Knudsen number},
     % after finishing this axis, draw the right border:
     after end axis={
          \draw[black,line width=0.8pt]
            (rel axis cs:0,0) -- (rel axis cs:0,1);
        },
      ]
    % the same four style combinations
    \addplot[blueColor!50, solid, line width=1pt] table[x=Ma,y=Kn300_20] {knre_data.dat};
    \addplot[blueColor, solid, line width=1pt] table[x=Ma,y=Kn300_40] {knre_data.dat};
    \addplot[redColor!50, solid,  line width=1pt] table[x=Ma,y=Kn600_20] {knre_data.dat};
    \addplot[redColor, solid, line width=1pt] table[x=Ma,y=Kn600_40] {knre_data.dat};
    % reference line (omit from legend)
    \addplot[greenColor, dashed, line width=1.5pt] table[x=Ma,y expr=0.1] {knre_data.dat};

  \end{groupplot}
\end{tikzpicture}
\end{center}

	Figure~ that for subsonic flow within the micro-channels, \(Kn\) typically remains below \(0.1\), validating the continuum assumption. At the same time, the Reynolds number remains well under \(100\), indicating laminar flow \cite{ames1953compressible, comsol_microfluidics_guide}. Although some regions may fall within the slip regime, continuum-based models remain largely applicable. 

	\paragraph{Transition to Free Molecular Flow.}
		Upon exiting the nozzle into vacuum, the gas pressure drops rapidly, leading to a potentially large mean free path.
		Here, the continuum approximation no longer holds, and using a single characteristic length becomes meaningless.
		Instead, a local Knudsen number is more appropriate:
		\begin{equation}
		  Kn_L
		  = \frac{\lambda}{\phi}\left|\frac{d\phi}{dx}\right|,
		\end{equation}
		where \(\phi\) is a flow property (e.g., \(p\), \(\rho\), \(T\), or \(U\)).
		This definition enables identifying where the flow transitions from continuum to free molecular behavior \cite{bird_dsmc_2013,Grabe2008,LiLam1964}.
		Since the focus here is on internal assembly flow, we do not account for rarefied effects outside the nozzle and omit slip-layer modeling for simplicity.

\subsubsection*{Steady Flow}
	The assumption of steady flow is justified by the boundary conditions of the system.
	The reservoir pressure $p_0$ and temperature $T_0$ are held constant during operation.
	Thus, the flow is driven solely by the pressure differential between the reservoir and the vacuum.

	Once the initial unsteady effects have decayed, the system reaches an equilibrium state in which the flow remains steady over time.
	This assumption is valid both within the assembly and in the downstream expansion into the vacuum.
	\cite{LiLam1964}
