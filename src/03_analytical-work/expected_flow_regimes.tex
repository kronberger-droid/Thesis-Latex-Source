
\subsubsection{Continuum Regime}
	The whole Theory relies heavily on the continuum model.
	For every location of the flow, but the vacuum, it will suffice to approximate the state variables at certain locations using a simple continuum model and use these state variables to calculate the Knudsen number.\\
	A sensible question to ask at this point is: Where in the flow it is the most probable encounter the highest Knudsen number.
	Because this will reduce the locations to check for high Knudsen numbers and therefore will simplify the process.
	To find such most probable Location, given the Definition of the Knudsen number:
	$$
		Kn(p,T) = \frac{\lambda}{L_c} = \frac{\mu(T)R}{pL_c}\sqrt{\frac{\pi m T}{2k_B}}
	$$
	Where $\lambda$ is the mean free path, $L_c$ is the characteristic length, $k_B$ is the Boltzmann constant, $R$ is the specific gas constant, $T$ is the temperature of the fluid, $p$ is the pressure of the fluid, $m$ is the molecular mass and $\mu$ is the dynamic viscosity.
	$L_c$ can be assumed constant since the height of the reactor and the smallest diameter of the ducts match.
	And by assuming $\mu$ to be close to constant the Knudsen number becomes a simple proportionality relation.
	$$
		Kn(p,T)\approx\frac{\mu R}{L_c}\sqrt{\frac{\pi m}{2k_B}}\cdot\frac{\sqrt{T}}{p}=\alpha\cdot\frac{\sqrt{T}}{p}\quad\rightarrow\quad Kn\propto \frac{\sqrt{T}}{p}
	$$
	This makes it obvious that areas with low pressure will lead to higher Knudsen numbers and will come closest to the limit of $Kn=0.1$ where continuum regime formulations stop to yield sensible solutions.
	Therefore, calculating the Knudsen number where the gas leaves the outlet nozzle will be useful to identify the flow regime that governs the gas flow inside the whole assembly.

\paragraph{Knudsen Number in low pressure Zones}

	When the gas is leaving the outlet the pressure will steadily drop until the method of determining Pressure and Temperature themselves will fail and force us to rely on numerical calculations.
	Therefore, it makes sense to identify a much more elegant way of calculating the local Knudsen number $Kn_L$ which will be much more applicable in this situation.
	$$
		K n_L = \frac{\lambda}{\phi} \left| \frac{d\phi}{dx} \right|
	$$
	Where $\lambda$ is the mean free path, $\phi$ is some state variable of the flow

{\color{greenColor}\itshape
Add reference!
}

\subsubsection{Laminar Flow}

	The Reynolds number itself does not bare any real significance for the applicability of the formulations used. Nonetheless, deviations in Reynolds number can in reality shape great parts of how the flow behaves and therefore effects important state variables.\\
	Reynolds numbers per unit length for isentropic expansion range $10^{-2} < Re/l < 1$, whereas the characteristic length is constant at $L_c = 20\cdot 10^{-6}$ and will therefore dominate the equation forcing low Reynolds numbers. \cite{ames1953compressible}\\
	Therefore the flow will stay laminar as long as it is contained inside the assembly and therefore mixing will be most likely diffusion mediated.
	\cite{comsol_microfluidics_guide}
	
\subsubsection{Dimension of the Flow}

	\paragraph{Inside - pseudo 1D}

		In this section of the flow a further simplification of the two-dimensional flow through a duct will be used called "quasi one-dimensional" flow.
		This is possible by reducing the velocity distribution present at any point in a duct to its mean velocity.
		Therefore, reducing the velocity from a distribution $V(r)$ at every point in the duct to a scalar value $V$ at every point of the duct.
		This is a general simplification which can be made when using continuum flow analysis inside of ducts and is only bound by the applicability of continuum flow.
		This makes sense, because approximations of state variables and mass flow is the main goal here.
		Taking mean values still yield sensible solutions and won't leave us with less information.

	\paragraph{Outside - radial symmetric 3D}

		Any further simplification from the radially symmetric expansion into vacuum described in section would not make sense since it would eliminate important information on how the flow behaves when leaving the outlet.
		Like the amount of gas which will reach the mass spec and the velocity distribution.
\cite{anderson2021modern}

\subsubsection{Steady Flow}

	Steadiness of the flow is essentially given by the fact that the temperature $T_0$ and the pressure $p_0$ of the reservoir will be held constant during measurements.
	Thus, the flow will be driven by the pressure differential between the vacuum and the reservoir alone.
	So there is no reason for the flow to establish any dynamic behaviors after reaching some equilibrium state, after gas is first released into the system.
	This is true for regions inside as well as outside the assembly.
