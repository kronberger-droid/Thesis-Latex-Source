	The geometry can be explained in three simple sections: gas from a reservoir flows over the inlet into the micro-reactor where it leaves through the outlet into a vacuum. 
	This is a stark simplification, but for a great part of this thesis, this is how we will imagine our flow path.
	This is because the only thing we left out is any kind of leak in the system.
	Those leaks will be the most influential around the reactor, since this is the only part that is not held at a constant pressure by any external part of the system.\\
\subsubsection*{Reservoir}

	It is kept at constant pressure $P_0$ and constant temperature $T_0$, out of which the density $\rho$ can be calculated.
	These values will represent the stagnation or total conditions of the system.
	For simplicity, it contains only one gas which is defined by its specific heat ratio $\gamma$ and by its molar mass $M_m$.
	The Formulations used yield the changes in pressure, temperature and density as ratios between the local conditions and the stagnation conditions
	Therefore, there is no need to define stagnation conditions explicitly beforehand, since local conditions can be calculated for by multiplying chosen stagnation conditions with the calculated ratios.
	Therefore, only the parameters of the gas have to be defined up front, which will be defined as:
	$$
		\gamma = 1.47\;, \qquad M_m = 28.013\; \frac{\text{g}}{\text{mol}}
	$$
	representing nitrogen gas ($\text{N}_2$), since it is comparable to the gases planned to use in experiments.
\subsubsection*{Inlet nozzle}

	\begin{figure}[H]
	    \centering
	    \includegraphics[width=0.9\textwidth]{src/03_analytical-work/fig_inlet-nozzle-geometry.pdf}
	    \caption{Non-technical drawing of the inlet nozzle geometry (not in proportion)}
	    \label{fig:geometry-inlet-nozzle}
	\end{figure}
	The duct connecting the inlet reservoir with the reactor will be a slightly converging duct, due to production constraints.
	Therefore, it will act like a Nozzle, accelerating the gas until it expands into the reactor.
	
\subsubsection*{Reactor}

	\begin{figure}[H]
	    \centering
	    \includegraphics[width=1.0\textwidth]{src/03_analytical-work/fig_reactor-geometry.pdf}
	    \caption{Non-technical drawing of the reactor geometry (not in proportion)}
	    \label{fig:geometry-reactor}
	\end{figure}
	The reactor resembles a very small but broad cylinder shape which is opened at the bottom.
	The sample will be pressed into the opening, which will lead to some leakage out of the system.
	
\subsubsection*{Outlet nozzle}
	
	\begin{figure}[H]
	    \centering
	    \includegraphics[width=0.9\textwidth]{src/03_analytical-work/fig_outlet-nozzle-geometry.pdf}
	    \caption{Non-technical drawing of the outlet nozzle geometry (not in proportion)}
	    \label{fig:geometry-outlet-nozzle}
	\end{figure}
	With the same geometry as the inlet, but the gas flowing in opposite directions, one would suspect the outlet to act as a subsonic nozzle, which could logically be the case, since without a converging section in front it would be impossible to reach sonic velocities and therefore will choke the flow and keep them at subsonic velocities.
	However, it is actually possible for the flow to create a converging section by itself, which will force the flow to be sonic at the beginning of the outlet and will further accelerate into the supersonic regimes, creating a supersonic nozzle.
	Which of these two possibilities is most likely will be discussed at a later point.
	
\subsubsection*{Vacuum}

	After leaving the outlet, the gas will first expand into a small cylindrical section after which it will expand freely into the vacuum.
	The exact pressure left in the vacuum chamber does not influence the gas flow inside the assembly since the large pressure ratio between reactor and vacuum will force the flow to be chocked, and thus the back pressure looses its influence.
	Due to the sharp change in pressure and the high velocity of the gas particles the gas will transition to the molecular regime after leaving the outlet nozzle.
