	The geometry can be explained in three simple sections: gas from a reservoir flows over the inlet into the micro-reactor where it leaves through the outlet into a vacuum. 
	This is a stark simplification, but for a great part of this thesis, this is how we will imagine our flow path.
	This is because the only thing we left out is any kind of leak in the system.
	Those leaks will be the most influential around the reactor, since this is the only part that is not held at a constant pressure by any external part of the system.\\

\subsubsection*{Reservoir}
	It is kept at constant pressure $P_0$ and constant temperature $T_0$, out of which the density $\rho$ can be calculated.
	These values will represent the stagnation or total conditions of the system, where the flow velocity is zero. The parts of the flow where stagnation conditions are assumed will be represented using a gray hue.
	For simplicity, it contains only one gas which is defined by its specific heat ratio $\gamma$ and by its molar mass $M_m$.
	The Formulations used yield the changes in pressure, temperature and density as ratios between the local conditions and the stagnation conditions
	Therefore, there is no need to define stagnation conditions explicitly beforehand, since local conditions can be calculated for by multiplying chosen stagnation conditions with the calculated ratios.
	Therefore, only the parameters of the gas have to be defined up front, which will be defined as:
	$$
		\gamma = 1.47\;, \qquad M_m = 28.013\; \frac{\text{g}}{\text{mol}}
	$$
	representing nitrogen gas ($\text{N}_2$), since it is comparable to the gases planned to use in experiments.
	\newpage

\subsubsection*{Inlet/Outlet Nozzles}
	\paragraph{Inlet}
		The duct connecting the inlet reservoir with the reactor will be a slightly converging duct, due to production constraints.
		Therefore, it will act like a Nozzle, accelerating the gas until it expands into the reactor.
	
	\paragraph{Outlet}
		With the same geometry as the inlet, but the gas flowing in opposite directions, one would suspect the outlet to act as a subsonic nozzle, which could logically be the case, since without a converging section in front it would be impossible to reach sonic velocities and therefore will choke the flow and keep them at subsonic velocities.
		However, it is actually possible for the flow to create a converging section by itself, which will force the flow to be sonic at the beginning of the outlet and will further accelerate into the supersonic regimes, creating a supersonic nozzle.
		\cite{jousten_handbook_2016}
	\begin{figure}[H]
	    \centering
	    \includegraphics[width=\textwidth]{src/03_analytical-work/fig_nozzle-geometries.pdf}
	    \caption{
			Schematic of the inlet and outlet nozzle assembly.
			Yellow-shaded regions denote domains where one-dimensional flow is assumed.
			Flow is indicated by colored arrows: green arrows represent subsonic flow and red arrows indicate supersonic flow.
			In the downstream expansion region, the flow is modeled as two-dimensionally rotationally symmetric (green-shaded).
			This schematic is not drawn to scale
		}
	    \label{fig:geometry-nozzles}
	\end{figure}

	\paragraph{Nozzle flow - pseudo 1D}
		In these sections of the flow a simplification of the two-dimensional flow through a duct will be used called "quasi one-dimensional" flow.
		This is possible by reducing the velocity distribution present at any point along the length of a duct to its mean velocity.
		Therefore, reducing the velocity from a distribution $V(r)$ at every point in the duct to a scalar value $V$ at every point of the duct.
		This is a general simplification which can be made when using continuum flow analysis inside of ducts with reasonable small change in cross-sectional area and is only bound by the applicability of continuum flow as described in chapter \ref{sec:flow-dimension-foundations}.
		One-dimensional flow will be represented using a yellow hue, as seen in the volume of the nozzles in figure \ref{fig:geometry-nozzles}.
		\cite{anderson2021modern}
		\newpage
\subsubsection*{Micro-Reactor Volume}
	The reactor resembles a very small but broad cylinder shape which is opened at the bottom.
	The sample will be pressed into the opening, which will lead to some leakage out of the system.
	\begin{figure}[H]
	    \centering
	    \includegraphics[width=\textwidth]{src/03_analytical-work/fig_reactor-geometry.pdf}
	    \caption{
			Schematic of the reactor geometry.
			The reactor volume is shown in blue to represent the region where three-dimensional flow occurs.
			The sample surface is indicated in red, and the leak is marked by a blue outline along the reactor perimeter.
			Flow characteristics are depicted by colored arrows: yellow arrows for slow-moving flow, green arrows for subsonic flow, and red arrows for supersonic flow.
			This diagram is not drawn to scale.
		}
	    \label{fig:geometry-reactor}
	\end{figure}

	\paragraph*{Flow through the reactor - 3D}
		After the gas flow leaves the inlet nozzle and streams into the reactor chamber, the geometry of the assembly has no symmetry to reduce the dimension of the velocity field.
		Additionally, there is rapid expansion of the gas since the constraints given by the walls of the nozzle are now gone and the gas flows over a sharp corner.
		Therefore, the flow field has to be dependent on all spacial dimensions and essentially means the only way to get an accurate representation of the flow field the Navier-Stokes equations have to be solved.
		This won't be possible using only analytical tools, therefore in the following sections two simplifications of the flow through the reactor will be used.
		At first, assuming pseudo one-dimensional flow throughout the reactor, which is definitely wrong, since the flux area is neither expanding very slowly, nor the flow will be isentropic, due to the expansion into free space.
		To address this issue and still be able to use analytical tools to solve the system, in chapter \ref{sec:disconnected-reservoirs} the reactor will be assumed to be a reservoir.
		Three-dimensional flow will be represented using a blue hue as seen in the middle of the reactor in figure \ref{fig:geometry-reactor}.

\subsubsection*{Vacuum}
	After leaving the outlet, the gas will first expand into a small cylindrical section after which it will expand freely into the vacuum.
	The exact pressure left in the vacuum chamber does not influence the gas flow inside the assembly since the large pressure ratio between reactor and vacuum will force the flow to be chocked, and thus the back pressure looses its influence.
	Due to the sharp change in pressure and the high velocity of the gas particles the gas will transition to the molecular regime after leaving the outlet nozzle.

	\paragraph{Free jet into vacuum - 2D}
		Similar to the flow into the reactor, the flow out of the system into the vacuum chamber won't be reducible to pseudo one-dimensional flow anymore.
		But this time the fact that the outlet nozzle is radially symmetric and there is no further geometry constraining the flow after it leaves the outlet.
		The flow can be at least assumed to inherit the radial symmetry from the outlet nozzle, thus reducing the spacial parameters of the flow field to the distance from the nozzle and the distance from the axis of symmetry of the outlet nozzle $r$.
		Furthermore, due to the gas expanding into vacuum and therefore the pressure dropping rapidly the gas will go through the process of rarefication, thus after some distance from the nozzle even continuum formulations break down.
		This will be discussed in more detail in chapter \ref{sec:outlet_plume} where possible ways to formulate solutions using numerical techniques.
		\cite{anderson2021modern}
