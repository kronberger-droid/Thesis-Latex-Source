\subsubsection*{Geometry and components}
\addcontentsline{toc}{subsubsection}{Geometry and components}

The geometry can be explained in three simple sections: gas from a reservoir (1) flows over a duct into the reactor (2) where it leaves through another duct into a vacuum (3). This is a stark simplification, but for a great part of this thesis, this is how we will imagine our flow path. This is because the only important thing we left out is any kind of leaks in the system. Those leaks will be most influential around the reactor, since this is the only part not held at a constant pressure by any external part of the system.\\
We will now get into the specifics about all the important parts of the system:\\
\textbf{Inlet Reservoir}\\
It is held at a constant pressure \(P_0\) and a constant temperature \(T_0\) and contains only one gas which is defined by its specific heat ratio \(\gamma\) and by its molar mass \(M_m\). These are all parameters which are set by us in advance. None of these will change throughout a calculation and are to be decided in advance.

\subsubsection*{Motivation and goals}
\addcontentsline{toc}{subsubsection}{Motivation and goals}
