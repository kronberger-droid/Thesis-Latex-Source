\subsubsection{Idealized flow regimes}

	In gas dynamics a flow can be categorized by its particle interaction using the Knudsen number, which represents the ratio between the mean-free-path $\lambda$ of the gas and some characteristic length $L_c$.
	$$
		Kn=\frac{\lambda}{L_c}
	$$
	The characteristic length is usually chosen to be the smallest linear length in the system.
	For example the throat diameter of a nozzle. \cite{putignano2012supersonic}
\paragraph{Continuum regime (\(Kn \leq 0.001\))}
	
	In this regime, the interactions of particles in the medium are much more frequent than the interactions of particles with the boundaries of the duct.
	This makes it possible to describe the fluid itself as a continuous medium with the assumption of non-slip boundary conditions.
	The Navier-Stokes equations govern the calculations in this regime.

	\begin{figure}[H]
	\centering
		\begin{tikzpicture}[scale=1.2]

		% Pipe background with border
		\fill[baseLightGray] (-3,-1) rectangle (3,1);
		\draw[baseDarkerGray, thick] (-3,-1) rectangle (3,1);

		% Horizontal centerline
		\draw[dashed, baseDarkGray] (-3.2,0) -- (3.2,0);

		\draw[dashed, baseDarkGray, thick] (-2,-1.2) node[below] {x} -- (-2,1.4) ;

		% Vertical red dashed centerline
		\draw[redColor, dashed, thick] (-0.5,-1.2) -- (-0.5,1.4);

		% Parabolic velocity profile (zero velocity at the boundaries)
		\draw[blueColor, thick, domain=-1:1, smooth] plot ({1 - 3*\x*\x}, \x);

		% Flow lines (arrows reducing length at walls)
		\foreach \y in {-0.9,-0.6,-0.3,0,0.3,0.6,0.9}
		    \draw[blueColor, -{Stealth[length=5pt]}] (-2,\y) -- ({1 - 3*\y*\y},\y);

		% Average velocity arrow placed at the centerline
		\draw[greenColor, thick, -{Stealth[length=7pt]}] (-2,1.2) -- (-0.5,1.2);
		\node[greenColor] at (-1.2,1.5) {\large $V_{\text{avg}}$};

		% u(r) label at the profile
		\node[blueColor] at (1,0.5) {\large $u(r)$};

		% r-axis
		\draw[-{Stealth[length=6pt]}, baseDarkerGray] (-3,-1.2) -- (-3,1.5) node[above] {$r$};
		\draw (-3.2,0) node[left, baseDarkerGray] {0};
	
		\end{tikzpicture}
	\caption{Velocity distribution at a point $x$ inside a constant area duct with slip boundary conditions. \cite{Cengel2017}}
	\label{fig:non-slip-flow}
	\end{figure}

\paragraph{Slip regime (\(0.001 \leq Kn \leq 0.1\))}

	For increasing Knudsen numbers the mean free path becomes comparable to the characteristic length scale of the system.
	In this regime, the assumptions for continuum flow still hold, but there are deviations, especially near the boundaries.
	While continuum mechanics assumes no-slip conditions on the boundary, in this regime, slip on the boundary must be factored in.

	\begin{figure}[H]
	\centering
		\begin{tikzpicture}[scale=1.2]

		% Pipe background with border
		\fill[baseLightGray] (-3,-1) rectangle (3,1);
		\draw[baseDarkerGray, thick] (-3,-1) rectangle (3,1);

		% Horizontal centerline
		\draw[dashed, baseDarkGray] (-3.2,0) -- (3.2,0);

		\draw[dashed, baseDarkGray, thick] (-2,-1.2) node[below] {x} -- (-2,1.4) ;

		% Vertical red dashed centerline
		\draw[redColor, dashed, thick] (-0.5,-1.2) -- (-0.5,1.4);

		% Parabolic velocity profile (zero velocity at the boundaries)
		\draw[blueColor, thick, domain=-1:1, smooth] plot ({0.5 - 1.5*\x*\x}, \x);

		% Flow lines (arrows reducing length at walls)
		\foreach \y in {-0.9,-0.6,-0.3,0,0.3,0.6,0.9}
		    \draw[blueColor, -{Stealth[length=5pt]}] (-2,\y) -- ({0.5- 1.5*\y*\y},\y);

		% Average velocity arrow placed at the centerline
		\draw[greenColor, thick, -{Stealth[length=7pt]}] (-2,1.2) -- (-0.5,1.2);
		\node[greenColor] at (-1.2,1.5) {\large $V_{\text{avg}}$};

		% u(r) label at the profile
		\node[blueColor] at (1,0.5) {\large $u(r)$};

		% r-axis
		\draw[-{Stealth[length=6pt]}, baseDarkerGray] (-3,-1.2) -- (-3,1.5) node[above] {$r$};
		\draw (-3.2,0) node[left, baseDarkerGray] {0};
	
		\end{tikzpicture}
	\caption{Velocity distribution at a point $x$ inside a constant area duct with slip boundary conditions. \cite{Cengel2017}}
	\label{fig:slip-flow}
	\end{figure}
{\color{greenColor}\itshape
Correct the Figures! Add the Boundaries!
}
	\newpage

\paragraph{Transition regime (\(0.1 \leq Kn \leq 10\))}
	
	This regime is a middle ground between continuum and fully molecular flow.
	Neither the continuum assumptions of fluid dynamics nor the free molecular flow assumptions hold completely.
	The interactions between the gas molecules and the boundaries are significant, and the flow characteristics may vary widely.

\paragraph{Molecular regime (\(Kn \geq 10\))}

	In this regime, the mean free path is much larger than the dimensions of boundaries.
	This leads to particle interactions themselves becoming negligible in comparison to the interaction of particles with the boundary.
	\cite{rapp2017microfluidics}

\subsubsection{Flow behaviors in micro channels}

	

\subsubsection{Turbulence}

	The ratio between the viscous and inertial forces in a fluid can be used to determine the probability of turbulence in a flow.
	This ratio is called the Reynolds number:
	$$
		Re = \frac{\rho v L_c}{\mu}
	$$
	Where  $\rho$ is the density of the fluid, $v$ is the fluid velocity, $\mu$ is the dynamic viscosity of the fluid the  At which value a flow becomes turbulent varies for different flow conditions and geometries. For internal flow through circular pipes it is usually around $Re = 2300$. \cite{Cengel2017}
	
\paragraph{Laminar Flow (\(Re \le 2300\))}

	Laminar flow is characterized by many parallel layers of fluid, with minimal mixing between them.
	Friction at the boundaries leads to little energy loss since fluid at the boundary doesn't interact with neighboring fluid.
	
\paragraph{Turbulent Flow (\(Re > 2300\))}

	Turbulent flow leads to greater mixing and velocity fluctuations in the fluid.
	Which leads to greater momentum and heat transfer, and therefore greater pressure drops. 

\subsubsection{Dimensionality of the flow}

	The dimensionality of the flow describes how many spatial coordinates the values in a velocity field depend on.
	The flow through a constant area duct is usually described as a one-dimensional flow field only depending on the position $x$ along the length of the duct.
	In the case of variable area ducts the flow will be three-dimensional and has to be calculated for all spatial coordinates.
	But assuming only a slight change in area along the length of the duct the flow can be approximated using a one-dimensional flow field with enough precision.
	This is called quasi one dimensional flow. \cite{anderson2021modern}

\newpage

\subsubsection{Mach regimes}

	The Mach number is defined as the ratio between the local velocity $u$ and the local speed of sound $a$.
	$$
		Ma = \frac{u}{a}
	$$
	It is a very important metric when analyzing isentropic flow.

\paragraph{Low subsonic regime (\(Ma < 0.3\))}

	For low Mach numbers, compressibility effects of a gas can be neglected, and the gas can be treated as an incompressible fluid.

\paragraph{Subsonic regime (\(0.3 < Ma < 1.0\))} 

	Inside system of variable area ducts the gas flow generally stays subsonic.
	Once sonic speed is reached in a converging duct, the behavior reverses, and the velocity decreases, limiting the flow to subsonic or sonic speeds within converging ducts.\\
	
	\begin{figure}[H]
		\centering
		\begin{tikzpicture}[font=\small, scale=1.4]

			% Titles
			\node[baseBlack] at (1,1.3) {\large Subsonic Nozzle};
			\node[baseBlack] at (6,1.3) {\large Subsonic Diffuser};

			% Nozzle on the left
			\draw[baseBlack, thick] (0,1) -- (2,0.5) -- (2,-0.5) -- (0,-1) -- cycle;

			% Flow arrow into nozzle
			\draw[line width=1.2pt, blueColor, -{Stealth[length=6pt]}] (-1.2,0) -- (0,0) 
			  node[midway, above, sloped, baseBlack]{subsonic};

			% Parameters in the nozzle
			% Green parameters (top line)
			\node[greenColor] at (1.0,0.3) {Ma $\uparrow$ \quad V $\uparrow$};

			% Red parameters (below)
			\node[redColor] at (1.0,-0.05) {P $\downarrow$ \quad T $\downarrow$ \quad $\rho \downarrow$};
			% Diffuser on the right
			\draw[baseBlack, thick] (5,0.5) -- (7,1) -- (7,-1) -- (5,-0.5) -- cycle;

			% Flow arrow into diffuser
			\draw[line width=1.2pt, blueColor, -{Stealth[length=6pt]}] (3.8,0) -- (5,0) 
			  node[midway, above, sloped, baseBlack]{subsonic};

			% Parameters in the diffuser
			% Green parameters (top line)
			\node[greenColor] at (6.0,0.25) {T $\uparrow$ \quad P $\uparrow$ \quad $\rho \uparrow$};

			% Red parameters (below)
			\node[redColor] at (6.0,-0.05) {V $\downarrow$ \quad Ma $\downarrow$};

		\end{tikzpicture}

		\caption{Change of flow properties in subsonic nozzles and diffusers}
		\label{fig:nozzle_diffuser_super}
	\end{figure}

\paragraph{Sonic regime (\(Ma = 1\))}
	
	Sonic flow occurs at the exit of a converging duct, if a certain critical pressure ratio between the two reservoirs connected by the duct is reached.
	This ratio is defined as:

	$$
		\frac{P^*}{P_t}=\left(\frac{2}{\gamma + 1}\right)^{\gamma/(\gamma - 1)}
	$$ 
	It is derived from isentropic flow relations and can be expressed in any state variable.
	
\paragraph{Supersonic regime (\(Ma > 1\))} 

	If there are critical conditions at the end of a converging duct and a diverging duct follows.
	The flow continues to accelerate and reaches supersonic speeds.
	The location where the flow reaches critical condition is called throat and represents the minimal diameter of the duct.\\
	In supersonic flows, state variables change rapidly causing phenomenons like shock waves and expansion fans.
	
	\begin{figure}[H]
	\centering
		\begin{tikzpicture}[font=\small, scale=1.4]
			% Titles
			\node[baseBlack] at (1,1.3) {\large Supersonic Nozzle};
			\node[baseBlack] at (6,1.3) {\large Supersonic Diffuser};

			%%%%%%%%%%%%%%%%%%%%%%%%
			% Supersonic Nozzle on the left (diverging duct)
			%%%%%%%%%%%%%%%%%%%%%%%%
			% For a diverging nozzle:
			% Left side narrower, right side wider
			\draw[baseBlack, thick] (0,0.5) -- (2,1) -- (2,-1) -- (0,-0.5) -- cycle;

			% Flow arrow into nozzle
			\draw[line width=1.2pt, blueColor, -{Stealth[length=6pt]}] (-1.2,0) -- (0,0) 
			  node[midway, above, sloped, baseBlack]{supersonic};

			% Parameters in the supersonic nozzle
			% Green parameters (top line)
			\node[greenColor] at (1.0,0.3) {Ma $\uparrow$ \quad V $\uparrow$};

			% Red parameters (below)
			\node[redColor] at (1.0,-0.05) {P $\downarrow$ \quad T $\downarrow$ \quad $\rho \downarrow$};

			%%%%%%%%%%%%%%%%%%%%%%%%
			% Supersonic Diffuser on the right (converging duct)
			%%%%%%%%%%%%%%%%%%%%%%%%
			% For a converging diffuser:
			% Left side wider, right side narrower
			\draw[baseBlack, thick] (5,1) -- (7,0.5) -- (7,-0.5) -- (5,-1) -- cycle;

			% Flow arrow into diffuser
			\draw[line width=1.2pt, blueColor, -{Stealth[length=6pt]}] (3.8,0) -- (5,0) 
			  node[midway, above, sloped, baseBlack]{supersonic};

			% Parameters in the supersonic diffuser
			% Green parameters (top line)
			\node[greenColor] at (6.0,0.25) {T $\uparrow$ \quad P $\uparrow$ \quad $\rho \uparrow$};

			% Red parameters (below)
			\node[redColor] at (6.0,-0.05) {V $\downarrow$ \quad Ma $\downarrow$};
		\end{tikzpicture}
	\caption{Change in flow properties in supersonic nozzles and diffusers}
	\label{fig:supersonic_nozzle_diffuser}
	\end{figure}

\newpage

{\color{greenColor}\itshape
	Add references! There should be two good ones!
}
